\invisiblepart{DOSSIER : RIGGER}

\section{THE RIGGER}
\begin{multicols}{3}
\setlength{\parskip}{.05cm}

\texttt{>>>When it comes right down to it, I don’t really live any-
where. Unless you count the driver’s seat. My crew might call
me the “lookout” or the “getaway driver” but when things
have gone bad, I’ve never seen them not be happy that I own
an armored truck with a couple of Vindicators on it.}

\texttt{Seriously, have you seen it? Man, she’s sweet. Purrs like a
kitten, too.}

\texttt{Anyway, with all this Matrix-this and magic-that and
mass-transit-other, you’d think driving wasn’t such a big
thing. Well, that’s a load of bullshit. See, runners don’t take
the fuckin’ subway, choombatta. There ain’t a bus that goes
to the top of Ares Macrotech Tower. You want discreet tac-
tical insertion into a hot LZ? Or a luxury ride in a tricked
out limo? Or how about a good old fashioned \#18 (that one
involves crashing a cement truck through a wall to-- well,
anyway, good times...).}

\texttt{Long story short, you want a ride? You talk to me.<<<}

\textbf{The Rigger} is a cybered-up, shit-hot driving machine.
When a team needs transportation, recon, or a flying
drone to blow the enemy into bloody rags, they turn
to their rigger. Riggers have the capability to oper-
ate any vehicle at its peak, as well as operate drone
vehicles of various kinds. Getting into and out of an
op, and providing a little robotic fire support, is the
rigger’s specialty.



\subsection{CREATING A RIGGER}

\paragraph{1.  Choose your Metatype}

You may choose \textbf{Human}, \textbf{Dwarf}, \textbf{Elf}, \textbf{Ork}, or
\textbf{Troll}. Each metatype offers a selection of meta-
type moves. Choose one metatype move from
the options presented.

\paragraph{2.  Choose your look}

\textit{Goggles, alert eyes, obvious cybereyes}

\textit{Kaiser helmet, cowboy hat, pirate bandana}

\textit{Biker clothes, flight suit, street clothes, punk}

\textit{Heavy body, built body, lean body}

\paragraph{3.  Choose your name and street name}

Make up a name and street name or pick a real
name and street name from the lists and name
generators starting in the \textbf{GM Resources} section.

\paragraph{4.  Assign your stats}

You have 5 stats: Awareness, Combat, Stamina,
Craft, and Presence. Important stats for you are
Awareness, Craft, and Stamina.

You have 4 \textbf{Build Points} to distribute among
your stats. To increase a stat by 1 point costs 1
Build Point. You may increase a stat to a maxi-
mum of +2 as a starting character. If you wish,
you may lower 1 stat to -1 in order to have an
additional point to spend.

\paragraph{5.  Create your Vehicle and Drones}

Pick a mix of vehicles and drones (you may have
up either 2 drones and 1 vehicle or 1 drone and
2 vehicles) from those listed in the \textbf{Vehicles} sec-
tion on page 38, or build them according to
the \textbf{Gear Creation} rules on page 62.

\paragraph{6.  Choose your Equipment}

Choose from the lists below, or customize your
own gear using the rules in \textbf{Creating Gear} on
page 60.

\textbf{Armor:} \textit{ballistic vest, lined coat}

\textbf{Weapon (choose 2):} \textit{Enfield AS-7, Browning
Max Power, Ares Predator, AK-97K, combat
axe}


\paragraph{7.  Choose your cyberware}

You have a \textbf{Control Rig} installed. This allows you
to link to your vehicles and drones. The Control
Rig is always active, and includes a datajack.

You may choose one of the following two kits.
Costs below do not include the cost of the Con-
trol Rig ((descriptions of these items are on page
38):

\textbf{Kit 1 (2 essence):} \textit{cybereyes with flare com-
pensator and low-light, tactical computer}

\textbf{Kit 2 (3 essence):} \textit{cyberears with noise damp-
er and radio, bone lacing}


\paragraph{8.  Set your Essence and Edge.}

To determine your starting Essence, subtract the
essence cost of your cyberware (if any) from 4.

You start with 3 Edge.

\paragraph{9.  Choose 2 Contacts}

Chop shop worker, go ganger, fence, trucker,
arms dealer, mechanic, bartender, cargo pilot,
car thief


\paragraph{10.  Establish debts and favors}

Place one of your fellow runners’ names in at
least one of the blanks in the \textbf{Debts \& Favors}
section of your playbook. Each time a name
appears in a debt or favor, it counts as 1 Bond
with that character. The more people you have
Bond with, the better.

\paragraph{11.  Starting Funds}

You start play with 3d6 x 400¥ immediately
available.

\paragraph{12.  Starting Moves}

You know all the Core and Secondary Moves.
You also know the \textbf{Wheelman} or
\textbf{Drone Rigger} move and one other Rigger move.

\end{multicols}

\newpage



\begin{dossier}
\dossierstatbar{THE RIGGER}
\hspace{.5cm}%
\vrule width 2pt
\hspace{.3cm}%
\begin{dossiermovebar}
\fontsize{9pt}{1em}\selectfont
\setlength{\parskip}{.2cm}


\unselectedMove{ Wheelman:} while jacked into a vehicle you own, when you: 
\begin{moveoptions}
  \moveoption{Stay Frosty, roll+Craft}

  \moveoption{ Check the Situation, add your
    vehicle’s Sensor rating to the roll }

    \moveoption{ Fail a move involving the vehicle, mark off 1 Fuel. }
\end{moveoptions}
    \unselectedMove{ Drone Rigger:} while jacked into a drone, when you: 
\begin{moveoptions}
      \moveoption{Rock \& Roll or Stay Frosty, roll+Craft}

      \moveoption{Check the Situation, roll+the drone’s Sensor rating}

        \moveoption{Fail a move involving the drone, mark off 1 Fuel. }

        \moveoption{ Take an action of your own (not involving the drone), take -2. }
\end{moveoptions}
        \unselectedMove{ Autonomous Mode:} when you put a drone in autonomous mode, indicate which 
          mode setting you want, and roll+Craft. On 10+, hold 2 to be spent on the drone’s moves. 
            On 7-9, hold 1. Drone mode settings (and the rolls they use for moves) are: 
\begin{moveoptions}
            \moveoption{ Sentry: the drone can make the Rock \& Roll move; roll+Tactical }

            \moveoption{ Recon: the drone can make the Check the Situation move; roll+Sensor }

            \moveoption{ Evasion: the drone can make the Stay Frosty move; roll+Power }
\end{moveoptions}
            \unselectedMove{ Split Personality:} when you launch a drone, roll+Awareness. On 10+, you don’t take 
                the normal -2 penalty to non-drone moves while controlling it. On 7-9, the penalty is 
                reduced to -1. 

                \unselectedMove{ Feedback:} when a vehicle or drone you are currently jacked into takes damage, 
                  roll+Stamina. On 10+, the feedback is filtered out completely. On 7-9,  you get a little bit of 
                  a zap: take 1 stun. On a failure, you get a wallop: take 1 wound. 

                  \unselectedMove{ Fly, my pretties!:} You can control two drones at a time instead of one. 

                    \unselectedMove{ Jury Rig:} when you have to make fast repairs to a vehicle or machine, roll+Craft. On 
                      10+, you get it running again and fast. On 7-9, you get it running, but (choose 1): 
\begin{moveoptions}
  \moveoption{ it will only run for 1d10 minutes }

  \moveoption{ afterwards, it will be a total loss. }

  \moveoption{ one of its qualities is reduced by 1, permanently }
\end{moveoptions}
                      \unselectedMove{ Garage:} when you have downtime or legwork time, you can upgrade one of your 
                        vehicles or drones. For every day of time you spend upgrading, you can improve one of the 
                        vehicle’s quality by 1 point or add or change a weapon. You can only upgrade each quality 
                          once. 

                          \unselectedMove{ Percussive Maintenance:} when you smack the hell out of a recalcitrant device, 
                            roll+Craft. On 10+, the device springs to life. On 7-9, the device works for only a moment, 
                            but you know what you need to do to fix it. Take +1 forward to Jury Rig. 

                            \unselectedMove{ Paint the Target:} when you point out a drone or vehicle’s weakness to your team- 
                                mates, they take +1 forward to attacks against it. 


\end{dossiermovebar}%
\end{dossier}

%%% Local Variables: 
%%% mode: latex
%%% TeX-master: "sixth_world"
%%% End: 
