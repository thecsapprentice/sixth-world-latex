\documentclass[oneside,10pt]{article}
% reminder: US letter: 596pt x 795pt

\newlength{\pagewidthA}
\newlength{\pageheightA}
\setlength{\pagewidthA}{8.5in}
\setlength{\pageheightA}{11in}

\newlength{\pagewidthB}
\newlength{\pageheightB}
\setlength{\pagewidthB}{11in}
\setlength{\pageheightB}{8.5in}

\newlength{\stockwidth}
\newlength{\stockheight}

\usepackage{geometry}

% these seem to have NO effect in preamble!
% geometry first has effect after begin{document}!
\pdfpagewidth=\pagewidthA \pdfpageheight=\pageheightA % to enforce?
\paperwidth=\pagewidthA \paperheight=\pageheightA     % for TikZ
\stockwidth=\pagewidthA \stockheight=\pageheightA % hyperref (memoir)?!


\geometry{inner=1cm,outer=1cm,top=1cm,bottom=1cm}


\makeatletter
\newcommand{\printpagevalues}{%
  % from geometry.sty:
  * paper: \ifx\Gm@paper\@undefined<default>\else\Gm@paper\fi \\%
  * layout: \ifGm@layout<custom>\else<same size as paper>\fi \\%
  \@ifundefined{ifGm@layout}{}{%
  \ifGm@layout
  * layout(width,height): (\the\Gm@layoutwidth,\the\Gm@layoutheight) \\%
  \fi
  * layoutoffset:(h,v)=(\the\Gm@layouthoffset,\the\Gm@layoutvoffset) \\%
  }%
  \pagevalues % from package layouts
}
\makeatother

\newcommand{\generatePageLayouts}{%
  % this command must be called after \begin{document}!

  % geometry needs layoutwidth - cause it ignores the above paper sizes!
  % layoutwidth=148mm ok, layoutwidth=\paperwidth NOT ok
  % paperwidth gets reset again internally in newgeometry: in log: *geometry* verbose mode: * layout(width,height): (614.295pt,794.96999pt)
  % but by using \stockwidth, which here is just a custom length: * layout(width,height): (421.10078pt,597.50787pt)

  \newgeometry{layoutwidth=\pagewidthA,layoutheight=\pageheightA,left=2cm,right=2cm,bottom=2cm,top=2cm}
  \savegeometry{LayoutPageA}

  \newgeometry{layoutwidth=\pagewidthB,layoutheight=\pageheightB,inner=1cm,outer=1cm,top=1cm,bottom=1cm}
  \savegeometry{LayoutPageB}
}


\newcommand{\switchToLayoutPageA}{%
  % doesn't include page sizes; so page size too:
  \pdfpagewidth=\pagewidthA \pdfpageheight=\pageheightA % for PDF output
  \paperwidth=\pagewidthA \paperheight=\pageheightA     % for TikZ
  \stockwidth=\pagewidthA \stockheight=\pageheightA % hyperref (memoir)?!
  \loadgeometry{LayoutPageA} % note; \loadgeometry may reset paperwidth/h!
}
\newcommand{\switchToLayoutPageB}{%
  % doesn't include page sizes; so page size too:
  \pdfpagewidth=\pagewidthB \pdfpageheight=\pageheightB % for PDF output
  \paperwidth=\pagewidthB \paperheight=\pageheightB     % for TikZ
  \stockwidth=\pagewidthB \stockheight=\pageheightB % hyperref (memoir)?!
  \loadgeometry{LayoutPageB} % note; \loadgeometry may reset paperwidth/h!
}


\usepackage{sectsty}
\usepackage[explicit]{titlesec}
\usepackage{fancyhdr}
\usepackage[absolute]{textpos}
\usepackage{xltxtra,fontspec,xunicode}
\defaultfontfeatures{Scale=MatchLowercase}
\setmainfont[BoldFont=Shadowrun Bold,ItalicFont=Shadowrun Italic]{Shadowrun}
\newfontfamily\fatefont{Fate Core Glyphs}
\newfontfamily\orbitronfont{Orbitron}
\newfontfamily\oswaldfont{Oswald}
\newfontfamily\shadowrunfont{Shadowrun}
\newfontfamily\shadowrunbfont{Shadowrun Bold}
\setlength{\parindent}{0cm}
\setlength{\parskip}{.2cm}
\setcounter{tocdepth}{2}
\setcounter{secnumdepth}{0}


\usepackage{pdflscape}
\usepackage{atbegshi}
\usepackage[svgnames,table]{xcolor} 
\usepackage{lipsum}
\usepackage{tabu}
\usepackage{multicol}
\usepackage{changepage}
\usepackage{parcolumns}
\usepackage{array}
\usepackage{comment}
\usepackage{picture}
\usepackage[hidelinks]{hyperref}
\usepackage{float}
\usepackage{layouts}
%\usepackage{showframe}
\usepackage{ifthen}
\usepackage{booktabs}
\usepackage{multirow}
\usepackage{tcolorbox}
\usepackage{tikz}
\definecolor{lightgray}{gray}{0.90}


\newcommand\Mybox[1]{%
  \setlength\fboxsep{0pt}\fcolorbox{red}{white}{#1}%
}


\newsavebox{\statBox}
\savebox{\statBox}
(25,25)[bl]{
  \multiput(0,0)(0,25){2}
  {\line(1,0){25}}
  \multiput(0,0)(25,0){2}
  {\line(0,1){25}}
  %\put(20,20){\line(1,0){5}}
  %\put(20,20){\line(0,1){5}}
  \put(0,15){\line(1,1){10}}
}

\newsavebox{\statBoxInf}
\savebox{\statBoxInf}
  (25,25)[bl]{
     \multiput(0,0)(0,25){2}
    {\line(1,0){25}}
  \multiput(0,0)(25,0){2}
    {\line(0,1){25}}
  \put(20,20){\line(1,0){5}}
  \put(20,20){\line(0,1){5}}
}

\newsavebox{\statBoxPlain}
\savebox{\statBoxPlain}
  (25,25)[bl]{
     \multiput(0,0)(0,25){2}
    {\line(1,0){25}}
  \multiput(0,0)(25,0){2}
    {\line(0,1){25}}
}

\newsavebox{\optaccumBox}
\savebox{\optaccumBox}
(10,10)[bl]{
  
  \multiput(0,0)(0,10){2}
  {\line(1,0){10}}
  \multiput(0,0)(10,0){2}
  {\line(0,1){10}}
}

\newsavebox{\accumBox}
\savebox{\accumBox}
  (10,10)[bl]{
\multiput(0,0)(0,10){2}
    {\line(1,0){10}}
  \multiput(0,0)(10,0){2}
    {\line(0,1){10}}
}

\newsavebox{\accumBoxT}
\savebox{\accumBoxT}
  (12,12)[bl]{
\linethickness{2pt}
\multiput(0,0)(0,10){2}
    {\line(1,0){12}}
  \multiput(1,0)(10,0){2}
    {\line(0,1){11}}
}


\newcommand{\selectedMove}[1]{{\fatefont{}B} \textbf{#1}}
\newcommand{\unselectedMove}[1]{{\fatefont{}b} \textbf{#1}}


%\floatstyle{boxed}
\newfloat{dossier}{t}{dos}
\floatname{dossier}{Dossier}

\def\vrulefill{\leavevmode\leaders\vrule width 2pt\vfill\kern\z@}

\titleformat{\section}[block]
{\orbitronfont\fontsize{1.2cm}{1em}\selectfont\bfseries\filright#1}{}{0pt}{}[\vspace{0ex}\titlerule]

\titleformat{\subsection}[block]
{\orbitronfont\fontsize{.7cm}{1em}\selectfont\bfseries\filright#1}{}{0pt}{}[\vspace{0ex}\titlerule]

\titleformat{name=\subsubsection}[block]{\orbitronfont\fontsize{.5cm}{1em}\selectfont\bfseries\filright}{}{0em}{\underline{\MakeUppercase{#1}}}

\titleformat{name=\paragraph}[block]{\orbitronfont\fontsize{.45cm}{1em}\selectfont\bfseries\filright}{}{0em}{\MakeUppercase{#1}}

\titlespacing{\paragraph}{%
  0pt}{%              left margin
  1\baselineskip}{% space before (vertical)
  .5em}%  

\newcommand{\LeftBlockEntry}[2]{                     % Same as \EducationEntry
  \noindent
  \colorbox{Black}{
    \parbox[c][.64cm]{#1}{\color{White}{\orbitronfont\fontsize{18pt}{1em}\bfseries\selectfont
        {\raggedright{} #2}}}
}}
\newcommand{\CenterBlockEntry}[2]{                     % Same as \EducationEntry
  \noindent
  \colorbox{Black}{
    \parbox[c][.64cm]{#1}{\color{White}{\orbitronfont\fontsize{18pt}{1em}\bfseries\selectfont
        {\centering{#2}}}}
}}

\newsavebox{\dossierboxhelper}
\newenvironment{dossierbox}[3]
{
\begin{lrbox}{\dossierboxhelper}
\begin{minipage}[t][#2]{#1}
\CenterBlockEntry{#1}{#3}
\vspace{.1cm}
}
{
\end{minipage}
\end{lrbox}
\usebox{\dossierboxhelper}
}

\newenvironment{moveoptions}
{
\vspace{-.3cm}
\begin{adjustwidth*}{.5cm}{.5cm}
\setlength{\parskip}{.1cm}
\begin{multicols}{2}
}
{
\end{multicols}
\end{adjustwidth*}
\vspace{-.3cm}
}

\newcommand{\moveoption}[1]{\parbox{\linewidth}{\tcirc{} #1}}

\newsavebox{\dossierstatbarbox}
\newcommand{\dossierstatbar}{
\begin{lrbox}{\dossierstatbarbox}
\begin{minipage}[b][\textheight][t]{.215\textwidth}
\begin{dossierbox}{5.2cm}{5cm}{Profile}
\begin{tabular}{|>{\oswaldfont\fontsize{15pt}{1em}\selectfont\bfseries}m{1.8cm}|m{2.9cm}|}
\hline
Street Name& \\[.25cm]\hline
Real Name&  \\[.25cm]\hline
Metatype&  \\[.25cm]\hline
Look&  \\[1cm]
\hline
\end{tabular}
\end{dossierbox}
\begin{dossierbox}{5.2cm}{6.5cm}{STATS}
\begin{picture}(5cm,5cm)
\put( 0cm,4cm){\usebox{\statBox}}
\put( 1cm,4.1cm){ \oswaldfont\fontsize{16pt}{1em}\selectfont\bfseries Oomph }
\put( 0cm, 3cm){\usebox{\statBox}}
\put( 1cm,3.1cm){ \oswaldfont\fontsize{16pt}{1em}\selectfont\bfseries Twitch }
\put( 0cm,2cm){\usebox{\statBox}}
\put( 1cm,2.1cm){ \oswaldfont\fontsize{16pt}{1em}\selectfont\bfseries Mastery }
\put( 0cm,1cm){\usebox{\statBox}}
\put( 1cm,1.1cm){ \oswaldfont\fontsize{16pt}{1em}\selectfont\bfseries Flair }
\put( 0cm,0cm){\usebox{\statBox}}
\put( 1cm,0.1cm){ \oswaldfont\fontsize{16pt}{1em}\selectfont\bfseries Essence }
\put( 3cm,4cm){\usebox{\statBoxPlain}}
\put( 4cm,4.1cm){ \oswaldfont\fontsize{16pt}{1em}\selectfont\bfseries Edge }
\put( 3cm,2cm){\usebox{\statBoxPlain}}
\put( 4cm,2.1cm){ \oswaldfont\fontsize{16pt}{1em}\selectfont\bfseries Armor }
\end{picture}
\end{dossierbox}
\begin{dossierbox}{5.2cm}{3cm}{DAMAGE}
\begin{picture}(5cm,1.1cm)
\multiput(1.25cm,0.6cm)(15,0){7}{\usebox{\accumBox}}
\put(4.95cm,0.6cm){\usebox{\accumBoxT}}
\put(3.25cm,.14cm){\line(0,1){6}}
\put(3.25cm,0.35cm){\line(1,0){52}}
\put(5.08cm,0.35cm){\line(0,1){6}}
\put(3.35cm, 0cm){\oswaldfont\fontsize{12pt}{1em}\selectfont\bfseries GUT CHECK }
\end{picture}
\end{dossierbox}
\begin{dossierbox}{5.2cm}{3cm}{XP}
\begin{picture}(5cm,1.1cm)
\multiput(0.2cm,0.6cm)(15,0){9}{\usebox{\accumBox}}
\put(4.95cm,0.6cm){\usebox{\accumBoxT}}
\put(3.25cm,.14cm){\line(0,1){6}}
\put(3.25cm,0.35cm){\line(1,0){52}}
\put(5.08cm,0.35cm){\line(0,1){6}}
\put(3.35cm, 0cm){\oswaldfont\fontsize{12pt}{1em}\selectfont\bfseries ADVANCE }
\end{picture}
\end{dossierbox}
\end{minipage}%
\end{lrbox}
\usebox{\dossierstatbarbox}
}

\newsavebox{\dossiermovebarbox}
\newenvironment{dossiermovebar}
{
\begin{lrbox}{\dossiermovebarbox}
\begin{minipage}[b][\textheight][t]{0.50\linewidth}
\begin{dossierbox}{17.6cm}{14cm}{ARCHETYPE MOVES}
}
{
\vspace{.2cm}
\end{dossierbox}%
\begin{dossierbox}{17.6cm}{3cm}{WEAPONS}
\begin{adjustwidth*}{0cm}{.2cm}
\vspace{-.1cm}
\begin{tabu}{p{5cm}p{1.5cm}p{1.5cm}p{5cm}p{3cm}}
\rowfont{\oswaldfont\fontsize{16pt}{0em}\selectfont} Weapon & Range & Damage & Ammo & Tags\\
\end{tabu}

\vspace{.2cm}
\setlength{\baselineskip}{.5cm}
\underline{\hspace{17.9cm}}

\underline{\hspace{17.9cm}}

\underline{\hspace{17.9cm}}

\underline{\hspace{17.9cm}}

\end{adjustwidth*}
\end{dossierbox}
\end{minipage}%
\end{lrbox}
\usebox{\dossiermovebarbox}
}


\newsavebox{\dossierrightbarbox}
\newcommand{\dossierrightbar}[1]{
\begin{lrbox}{\dossierrightbarbox}
\ifthenelse{ \equal{#1}{standard} }{
\begin{minipage}[b][\textheight][t]{0.215\linewidth}
\begin{dossierbox}{5.2cm}{3cm}{DEBTS \& FAVORS}
\begin{adjustwidth*}{0cm}{.2cm}
\fontsize{9pt}{1em}\selectfont

\underline{\hspace{3cm}} taught me a valuable lesson in
self-control.

\underline{\hspace{3cm}} helped me find my place here in the
shadows.

I killed someone for \underline{\hspace{3cm}}.

I taught \underline{\hspace{3cm}} secrets normally reserved
for the masters.
\end{adjustwidth*}
\vspace{.5cm}
\end{dossierbox}% 
\begin{dossierbox}{5.2cm}{3cm}{ARMOR}
\begin{adjustwidth*}{0cm}{.2cm}
\vspace{-.2cm}
{\oswaldfont\fontsize{14pt}{0em}\selectfont Type \hspace{2.6cm} Armor Value }
\vspace{.2cm}
\underline{\hspace{5cm}}
\end{adjustwidth*}
\end{dossierbox}%
\begin{dossierbox}{5.2cm}{3cm}{GEAR}
\begin{adjustwidth*}{0cm}{.2cm}
\vspace{.2cm}
\setlength{\baselineskip}{.5cm}
\underline{\hspace{5cm}}
\underline{\hspace{5cm}}
\underline{\hspace{5cm}}
\underline{\hspace{5cm}}
\underline{\hspace{5cm}}
\underline{\hspace{5cm}}
\underline{\hspace{5cm}}
\underline{\hspace{5cm}}
\underline{\hspace{5cm}}
\underline{\hspace{5cm}}
\underline{\hspace{5cm}}
\underline{\hspace{5cm}}
\underline{\hspace{5cm}}
\underline{\hspace{5cm}}
\underline{\hspace{5cm}}
\underline{\hspace{5cm}}
\underline{\hspace{5cm}}
\underline{\hspace{5cm}}
\underline{\hspace{5cm}}
\underline{\hspace{5cm}}
\end{adjustwidth*}
\end{dossierbox}%
\end{minipage}
}{
  This is false.
}
\end{lrbox}
\usebox{\dossierrightbarbox}}



\newcommand{\dossiervehiclebox}{
\begin{dossierbox}{6.9cm}{7cm}{VEHICLE / DRONE }
\begin{tabular}{p{2cm}p{2cm}p{2cm}}
\multicolumn{3}{l}{Name:} \\[.15cm] \hline
\multicolumn{3}{l}{Type:} \\[.15cm] \hline
\multicolumn{3}{l}{Tags:} \\[.15cm] \hline
Power:&Armor:&Frame:\\[.15cm] \hline
Sensor:&Fuel:&Capacity:\\[.15cm] \hline
\multicolumn{3}{l}{Tactical(Drone):}\\[.15cm]\hline
\multicolumn{3}{l}{Notes:} \\[.15cm]
\hline \\[.15cm] 
\hline \\[.15cm]
\hline \\[.15cm]
\hline \\[.15cm]
\hline \\[.15cm]
\hline \\[.15cm]
\hline \\[.15cm]
\end{tabular}
\end{dossierbox}
}









%%% Local Variables: 
%%% mode: latex
%%% TeX-master: "sixth_world"
%%% End: 


\def\SW/{\textit{Sixth World}}
\def\DW/{\textit{Dungeon World}}
\def\SR/{\textit{Shadowrun}}
\newcommand{\tcirc}{\protect{$\bigcirc$}}
\def\oomph/{\textbf{Oomph}}
\def\twitch/{\textbf{Twitch}}
\def\mastery/{\textbf{Mastery}}
\def\flair/{\textbf{Flair}}
\def\essence/{\textbf{Essence}}


\newenvironment{gmtip}%
{\begin{tcolorbox}[title=GM Tips,sharp corners,rounded
    corners=downhill,parbox=false]
\setlength{\parindent}{0cm}
\setlength{\parskip}{.2cm}
\noindent
  }%
{\end{tcolorbox}}

\newenvironment{dent}%
{\begin{adjustwidth*}{.5cm}{.5cm}}%
{\end{adjustwidth*}}


\newcommand*\parttitle{}
\newcommand\invisiblepart[1]{%
  \refstepcounter{part}%
  \renewcommand*\parttitle{#1}%
}

\newcommand{\movename}[1]{\textbf{\MakeUppercase{#1}}}

\newcommand{\fluff}[1]{\texttt{\textgreater \textgreater \textgreater #1\textless \textless \textless}}

\title{Your Title}
\author{Your Name}
\date{}

\setlength{\TPHorizModule}{1cm}
\setlength{\TPVertModule}{1cm}



%%% Local Variables: 
%%% mode: latex
%%% TeX-master: "sixth_world"
%%% End: 


\begin{document}

\generatePageLayouts{}
\switchToLayoutPageA{}


\fancypagestyle{plain}
{
  \fancyhf{}
    \fancyhead{}
    \fancyfoot{}
}	% clear header and footer of plain page because of ToC

\begin{titlepage}

\begin{center}

% Upper part of the page. The '~' is needed because \\
% only works if a paragraph has started.


~\\[5cm]

{\fontsize{75pt}{1em}\selectfont\bfseries
  SIXTH WORLD}\\[.5cm]

{\orbitronfont\fontsize{22.5pt}{1em}\selectfont\bfseries
  A DUNGEON WORLD HACK}\\[0.5cm]

{\orbitronfont\fontsize{22.5pt}{1em}\selectfont\bfseries
  FOR SHADOWRUN}\\[0.5cm]



\vfill

% Bottom of the page
Version: 1.0, Codename: ARES

Authors: Nathan Mitchell

Past Version Authors: Chris Clouser, Tanner Yea

\end{center}

DISCLAIMER

Dungeon World is the property of Sage LaTorra and Adam Koebel, and is available under the Creative Commons Attribution 3.0 Unported
License. See www.dungeon-world.com for details.
The Topps Company, Inc. has sole ownership of the names, logo, artwork, marks, photographs, sounds, audio, video and/or any proprietary
material used in connection with the game Shadowrun. This is a fan-created adaptation, and no challenge is intended toward Topp’s ownership of the Shadowrun intellectual property.





\end{titlepage}
%%% Local Variables: 
%%% mode: latex
%%% TeX-master: "sixth_world"
%%% End: 



\begin{multicols}{2}
\tableofcontents
\end{multicols}
\addtocontents{toc}{\protect\thispagestyle{plain}}

\newcommand{\critterspec}[7]{
\setlength{\parskip}{.1em}
\vspace{.5cm}
\begin{minipage}{\linewidth}
{\large\bfseries #1}

{\itshape #2}

#3

#4\vspace{.25em}
\hrule
\vspace{.25em}
#5 \textit{Instinct:} #6
\begin{adjustwidth*}{.5cm}{.5cm}
#7
\end{adjustwidth*}
\end{minipage}
}


\newpage
\pagestyle{fancy}
\setcounter{page}{1}
\renewcommand{\headrulewidth}{0pt} % remove lines as well
\renewcommand{\footrulewidth}{0pt}
\fancyhf{}
\fancyhead[RO]{
  \orbitronfont\fontsize{.5cm}{0em}\selectfont\bfseries
  \MakeUppercase{ \parttitle{} | SIXTH WORLD }}
\fancyfoot[RO]{\orbitronfont\fontsize{.5cm}{0em}\selectfont\bfseries\thepage}

\invisiblepart{Introduction}
\section{WELCOME TO SIXTH WORLD}
\label{introduction}

\begin{multicols}{2}

\SW/ is a ``hack'' of the game \DW/ which
attempts to capture the flavor of the world of the well-known
RPG \SR/®.

The ``Sixth World'' is the dangerous and grim future of our
own world, where magic has resurfaced, megacorporations
rule the world, and humanity has perfected incredible new
technological capabilities including advanced cybernetics and
the worldwide virtual reality network called the
Matrix.

This game assumes familiarity with \SR/, as well as
with \DW/.


\subsection{WHAT DO I DO?}

You take on the role of a shadowrunner, an individual who conducts,
let’s say, \textit{quasilegal} activities at the behest of the
corporations, governments, and organized crime. You’ll build a dossier
for a character who will serve as your proxy as you experience the
excitement and danger of the shadows of the Sixth World.


\subsection{FICTION FIRST}
\label{fictionfirst}

Everything that happens in a session of \SW/ starts with the fiction,
proceeds to rules (if necessary), and ends with the fiction. Most of
the rules of the game are encapsulated in items called
\textbf{moves}. That’s simply game terminology for a small package of
instructions telling you how to attempt to perform certain actions and
resolve them using the rules. So for instance, the move called
\textit{Rock \& Roll} contains instructions on how to fight with
someone.


However, it is important to remember that because the game starts with
and ends with the game fiction, you should never say ``I use Rock \&
Roll on that guy!''


In fact, this is a cardinal rule for both players and the GM:
\textbf{you never say the name of your move}. You simply determine,
from what you are doing in the game world (running, shooting, jumping,
dying, etc.), what move would apply. When the rolling is done, you
conclude with some more fiction (or perhaps the GM does, depending on
the outcome). Thus the flow of play is:

\vspace{.1cm}
\begin{adjustwidth*}{1em}{0em}
{\orbitronfont \textbf{FICTIONAL ACTION > RULES > FICTIONAL OUTCOME}}
\end{adjustwidth*}
\vspace{.1cm}

For the most part, it’s what you’ve always done when gaming: try
something, roll some dice, and see how it comes out.


Also remember this: if you do something in the game world that would
trigger a move, then \textit{you must make that move}.  You can’t say
``I’m diving into the closet to avoid being spotted'' and then
\textit{not} make the \textit{Stay Frosty} move. Likewise, you can’t
make a move unless the situation actually demands it.  If you’re not
fighting someone who’s fighting back, then you \textit{don’t} get to
make the \textit{Rock \& Roll} move.

\begin{gmtip}
  When a player does something to trigger a move that seems
  questionable given the circumstances, it’s nice to remind them of
  their situation, and give them a chance to revise what’s
  happening. As the GM, it’s not your job to nail them with gotcha
  moments. Instead, point out the potential issue you see and let
  them decide.
  
  A good example of this is the \textit{Centering}
  move from the Mage skillset. It simply says ``when you take a
  moment to concentrate and restore yourself, hold 2 for future
  spell casting.'' So all the fiction \textit{requires} is that the
  mage stop what they’re doing, take a moment, and gather their
  strength. Nothing confusing there.  However, if the mage is in the
  middle of a firefight, and needs to center themselves, they might
  just say ``okay, I need to get things together here...I calm myself
  and draw on the power of the astral realm.''
    
  When they do that, remind them that they’re in a firefight, and
  based on what happened just \textit{before} they needed to center
  themselves, they could be exposed to real danger. Suggest, for
  instance, that they dive for cover or get behind something sturdy
  before they hit the astral gas pump. This isn’t hand-holding, this
  is just making sure the fiction is working. If they say ``no, no
  time, I’ll do it now,'' you can decide what kind of opportunity
  that gives you, and what you’ll do about it.%
\end{gmtip}

On a related note, since the fiction anchors the game, remember that
if you want to speak to or ask something of a character played by
another person, don’t say ``Hey Keith, do you have a spare frag
grenade?'' Instead, speak to the Keith's character: ``Hey, Valentin, do
you have a spare frag?''

Even though character names should be used, you don’t have to act in
first person. What is important is to remain focused on the
characters. So if the GM says, ``Valentin, there’s an ork with a bat
coming your way. What do you do?'' Keith is perfectly free to say,
``Valentin pulls his trenchcoat aside to show the gleam of his custom
Ares Predator.''

Just remember: flow from the fiction to the rules and back to the
fiction, and stay focused on the characters, and everything will be
all right!


\subsection{STATS}
\label{stats}

Most of the rules of \SW/ rely on the value of a player character’s
Stats. You’ll hear more about these later on (especially when you get
to the Dossiers on page \pageref{dossiers}), but every player
character in \SW/ is described by 5 stats:

\vspace{.1cm}

\begin{dent}
{\shadowrunbfont Oomph:} your raw physical power and toughness. A
person's Oomph refers to both their strength and their ability absorb
punishment, physically and mentally. A character with a lot of Oomph
might be a overbuilt street bruiser with fists like concrete blocks,
or they could be a quiet spiritual person with near unbreakable willpower.

{\shadowrunbfont Twitch:} your alertness, reflexes, and ability to react to
dynamic situations. Everyone who runs in the shadows develops some
amount of Twitch, otherwise you quickly find yourself bleeding out in
an alley. Anyone heavily into combat can make use of high Twitch, but
good reflexes are never a bad thing.

{\shadowrunbfont Mastery:} a general rating of your special, usually
technical, talents. Mastery represents your finely honed technical,
academic, or social knowledge. Mastery is generally more important for
Hackers and Riggers, but every archetype has some use for Mastery.

{\shadowrunbfont Flair:} all the aspects that set you apart from the
standard drudgery of metahumanity. Be it a winning smile, imposing
stride, or intricate body art, your Flair allows you to make a lasting
impression (typically without bullets) on others. Flair is classically
important for Faces, but cold-stone killers often use their own
special brand of Flair as they curb stomp a ganger. 

{\shadowrunbfont Essence:} your life force and metahumanity, this
fuels the supernatural powers of magical archetypes (Adept, Mage, and
Shaman) and technomancer archetypes (Botmaster, Architect,
Infovore). Essence also controls how easily one recovers from damage
and limits personal augmentation.

\vspace{.1cm}
\end{dent}

\subsection{ROLLING THE DICE}
\label{rollingdice}
In this game, the dice rolling revolves around the concept of
the Move. When you are instructed to roll dice for a move,
your responsibility is simple: roll 2d6, and add the value of
a stat (or sometimes some other value) to the result. When
a roll is needed, it is usually phrased as ''roll+Something,''
where ``something'' is the value to add to the roll.

\vspace{.1cm}
\begin{dent}
\textbf{Example:} \textit{if you are told to roll+\oomph/, you would roll
2d6, sum the total, and add the value of your \oomph/ stat to the result.}
\end{dent}
\vspace{.1cm}

The total of the roll indicates the outcome of the action taken
by the character:

\begin{dent}
\textbf{On a 10+, you achieve a strong success:} you’ve achieved
your aim without complication, and to the fullest extent
possible.

\textbf{On 7-9, you have achieved a weak success:} your achieve
your aim, but with a cost. You will usually be presented with a list of complications to choose from, although
sometimes instead the GM will tell you what complication
occurs.

\textbf{On a total of 6 or less, you have failed:} you don’t get
what you want. In fact, things are probably going to get
worse.
\end{dent}

\begin{gmtip}
A player's failure on a roll is a opportunity for you as the
GM. It should not be seen as a dead zone in the flow of the game where
nothing happens. Nor is it a place to unnecessarily punish the player - primarily
because this is unfair!  They didn't decide to roll badly. Instead, as
the GM, you should interpret a failure roll as a point in the game to
reclaim narrative control from the players. 

This is covered in a later section on GM Moves on page
\pageref{gmmoves}. 
\end{gmtip}

Note that if a move just says ``roll,'' then you don’t add anything. You just roll 2d6.

In addition to the common 2d6 roll, \SW/  uses the 
other common polyhedral dice: \textbf{d4}, \textbf{d6}, \textbf{d8}, \textbf{d10}, and \textbf{d12}. 
Twenty-sided dice are not used for mechanics, but can be 
used for some of the random generators at the end of this 
document. 

\subsubsection{ROLL MODIFIERS }
\label{rollmods}
While the basic move roll is 2d6+(\textit{something}), there are a few 
modifiers and tricks that may apply to a roll. The rules will 
always indicate when to use one of these
modifiers. 

\begin{dent}
\textbf{hold n:} when you are told to Hold \textit{n}, or that you gain \textit{n} 
Hold, this means you have a small pool of points that can 
be spent at some future moment of your choosing. You 
will be told on what, specifically, you may spend the Hold. 
Note that if you can spend Hold on a dice roll, you can do 
so \textit{after} you see the results of the roll! 

\textbf{take +n forward/-n forward:} this means take a bonus
(the +) or a penalty (the -) equal to \textit{n} to your
next Move.

\textbf{take +n ongoing/-n ongoing:} this means to take a bonus or penalty equal to \textit{n} to all of your future rolls, until
whatever circumstances caused the ongoing modifier have
changed.

\textbf{boosted:} whenever you are boosted, your result is never
lower than 7 (even if you roll 6 or less). So, when boosted,
you cannot fail, though success may still come at a cost
(not least of which is the fact that while boosted, you can’t
mark XP).

\textbf{glitched:} glitched rolls are the opposite of boosted rolls.
Whenever you are glitched, your result is never higher
than 9, even if you rolled a 10+. You can succeed while
glitched, but it will always come with a cost.

\textbf{b:} this means ``take the best of'' - you roll multiple dice,
but keep only one of them to determine the final total.
For instance, if you are instructed to roll 2d6b, you would
roll 2d6, and keep the highest die. When written by itself
(without a dice expression) it will be written as
[b].

\textbf{w:} this means ``take the worst of'' - if you are instructed to
roll 2d6w, then you would roll 2d6 and keep the lowest
die. When written by itself (without a dice expression), it d
will be written as [w].

\end{dent}

\subsection{STATS IN DETAIL}
\label{detailstats}

\subsubsection{Oomph}
\label{stat_oomph}

\oomph/ represents a character's raw power and resilience. It is an abstract rating
that rolls in a variety of possible sources. For instance, a heavy
muscle-bound jock type might have a lot of oomph, but so too could a
lithe monk with powerful self-control and willpower. Oomph
fundamentally represents raw, unfocused power. This can be seen in
contrast to the power represented by \mastery/ or \flair/. These forms
of power are calculated or channeled for best effect. In terms of the
street, \oomph/ is a high explosive grenade, while \mastery/ is a sniper
rifle.

\oomph/ is most important to characters who live and die by physical
confrontation. Gangers, bouncers, and street samurai all generally have
large amounts of \oomph/. Generally, feats of athletic prowess are
all governed by a character's \oomph/. Resisting mental influences also
draws upon a character's \oomph/, as they form mental barriers and
struggle to maintain a sense of self. Lacking \oomph/ can reflect a
character who is physically or mentally weak, lacking in resolve, or
simply willing to go with the flow.

\oomph/ also defines a large part of a character's ability to take
damage and recover from fatigue.  

\subsubsection{Twitch}
\label{stat_twitch}

In the \SW/, the ability for character to move and react quickly is
defined by their \twitch/ stat. Similar to \oomph/, \twitch/ is an
abstract stat that covers many things: reaction time, dexterity,
initiative, or even just a sense of danger. Characters with a large
amount of \twitch/ might be especially nervous, ready for anything, or
simply unusually keen and aware of their surroundings. Lacking
\twitch/, might mean a character is slower to react, clumsy, or perhaps
simply implacable and chooses not to react to external events.

\twitch/ has a direct role in combat, acting as a limited form of
armor, representing a character's innate ability to dodge out of the
way of danger. Wearing heavy armor acts as a penalty, however, so
characters generally need to choose between being well armored and
being nimble. Characters with a negative values of \twitch/ don't
necessarily take more damage, but a GM might penalize them in combat
more than usual. See Section \ref{sec_combat} on Combat for more information.

\subsubsection{Mastery}
\label{stat_mastery}

\mastery/ represents a character's ability to use trained skills. The
skills themselves are represented by the particular moves the
character has access to, while the \mastery/ stat represents how much
expertise they have with the skill.

\mastery/ is important to each archetype differently, as every
archetype has a different set of skills that uniquely define
it. However, most firearm usage depends on \mastery/, as successfully
using a firearm is not as simple as swinging a section of rebar at a
ganger's head.  

\subsubsection{Flair}
\label{stat_flair}

In the \SW/ everyone who is anyone has some amount of \flair/. Runners
build and depend on their reputations when dealing with one
another. But those with something extra, something \textit{special}
have more. Maybe its a signature combat move, a smile that shines
like the light of a thousand suns, or simply the ability to adapt to
social situations like a fish in water - all of this is covered by
\flair/.

\flair/ is most important to the Face archetype, though other
archetypes have occasional uses for this stat. But given that your
reputation is often the only thing that stops you from getting shot or
lands you a job, \flair/ important to everyone.

\subsubsection{Essence}
\label{stat_essence}

Every character in \SW/  has a stat called \essence/,
representing their metahumanity, life force, and mystical connection
with the world. Unlike other stats, Essence values range between 2 and
-6. Positive values represent a character who is full of life and
vitality, or, if they are Awakened or a Technomancer, possessed of a
deep connection to supernatural abilities. Negative values of Essence
generally means that a person is damaged in some way. This damage is
not simple, like being wounded, but reflects some deep inner damage to
their humanity. Over the course of the game, events can alter a
person's essence. These changes are typically permanent, so they
reflect major events in a character's life. As an example, a drug user
who overdoses might not suffer essence loss, but a drug user that
spirals down into broken wreck of their former self probably would.

Another common source of essence loss is the installation of cyberware
or other major body augmentations. The cause is unknown, but wholesale
replacement of one's body seems to result in a fracturing of a
person's fundamental humanity. Magical theorists have made comments
about people's auras and spirits losing connection to their physical
body, but most medical doctors dismiss such claims as
superstition. Regardless, once a person has replaced enough of their
original body that their essence would drop below -6, they die. There
are rumours on the street of people surviving with extremely low
essence, but they are often horror stories of shambling mechanical
monsters, or emotionless killing machines.

Less important for the mundane members of metahumanity, is how essence
interacts with supernatural abilities. Those members of society who
are Awakened or possess the strange Technomancer abilities, depend on
high values of essence to practice their crafts. A loss of essence for
these people can be a death sentence for their abilities, and they
lose them completely if their essence ever drops below -2.

But even for mundane metahumanity, essence is important. High essence
values improve healing and increase general resistance to
fatigue. Every point of essence provides an additional box of fatigue,
but every negative point of essence removes a box. 

\subsection{EDGE}
\label{stat_edge}

Each Archetype in this game has a variable pool of points
called \textbf{Edge}. Edge is an in-game currency representing a
number of real-world (or at least, game-world) concepts,
from combat experience to how many jobs they’ve pulled off
 to their ability to turn a bad situation into a survivable one to
 their general, flat-out awesomeness. A character can never have more
 edge than their Edge pool (See \textbf{EARNING EDGE} on page \pageref{earningedge}).

 \subsubsection{SPENDING EDGE}
 \label{spendingedge}
The main way to spend Edge is to gain bonuses to damage
and to rolls. When a player wishes it, they can spend edge
as follows:

\begin{dent}
\textbf{To improve damage:} for every point of Edge spent, they
can add a point of damage to their most recent
attack.

\textbf{To improve a roll:} for every two points of Edge spent,
a character can add one point to the result of their most
recent move.
\end{dent}

Edge is also used to:

\begin{dent}
\tcirc{} Attuning to magical items (see page 40)

\tcirc{} Investing magical fetishes (see page 40)

\tcirc{} Activating certain cyberware (see page 38)

\tcirc{} Surviving when things are at their darkest (see the Last
\textbf{Chance} move, page \pageref{move_lastchance})
\end{dent}

Feel free to think of other ways that Edge can be spent; just
make sure it’s fun.

\subsubsection{EARNING EDGE}
\label{earningedge}
When Edge is spent, it remains spent until the character has a chance
to spend at least a few hours resting in a place of relative safety,
at which point the pool of Edge refreshes.  Starting characters
generally have a relatively small pool of Edge. However, they will
increase the size of their Edge pool over the course of their
adventures. Players increase their pool size in 2 ways:
\begin{dent}

  \tcirc{} Choosing to gain a point of edge when they make the
  \textbf{Advance} move (page 5)
  
\tcirc{} Attempting something insanely awesome. Actions of this
magnitude can be nominated by any other player and the group votes on
it. Typically this is a small reward for players who take on extremely
dangerous and risky tasks and succeed (or fail!)  brilliantly, but it
could also result from a particularly excellent job of roleplaying. If
you’re the GM, don’t be too harsh here: players rewarding each other
for having a good time and getting into the spirit of things is a
\textit{good thing}. Indulge it!

\end{dent}



\subsection{XP}
\label{stat_xp}
Characters advance by earning \textbf{XP} (typically called ``Marking
XP'') as they navigate their shadowruns. Characters can mark
XP in the following circumstances:

\begin{dent}
\tcirc{} when they fail a move (this is the most common reason
XP is marked)

\tcirc{} when they finish a run, or a significant portion of a major
run

\tcirc{} when they resolve one of the debts or
favors they have with another character

\tcirc{} when they are manipulated (see page 4) by another
character
\end{dent}

Once a character marks 10 XP, they may use the Advance
move (page 5) to ``spend'' that XP to improve their character.
Possible improvements include gaining new moves, gaining
more Edge (as mentioned above), or improving a core stat.

\subsection{DEBTS \& FAVORS}
\label{debtsfavors}
Even in the high-tech world world of the 2070’s, nobody
goes it alone in the shadows for long. Sooner or later, you
need to get help from somebody. Sometimes, you can buy
that help with money. Other times, legal tender won’t cover
it and that’s when debts and favors come into
play.

Together, Debts \& Favors form the \textbf{bond} between runners in
a team. If, at the end of a session, you have resolved one of
these bonds, you erase the debt or favor, and you and the
other runner mark XP.

\subsubsection{DEBT}
A debt is something you owe a fellow runner. Maybe they
yanked your ass out of a bad situation down in Aztlan, or
helped spring you from jail, or just lent you some of their own
hard-won experience that saved your bacon.

\subsubsection{FAVOR}
A favor, conversely, is something owed to you by a fellow
runner. Maybe you were the one doing the hot-LZ extraction
in Aztlan, or you took the rap for them on a particular smash
‘n grab job.

Debts and favors are not necessarily reciprocal! A character might
perceive a debt to another that is entirely self-imposed.  Conversely,
a character might feel like one of their teammates owes them
something, while that teammate might be completely unaware of that
feeling. So, when establishing debts and favors, don’t assume that a
debt on one sheet has to correspond to a favor on another!

\end{multicols}

\newpage

\invisiblepart{MOVES}
\section{MOVES}
\label{moves}
\begin{multicols}{2}

In \SW/, the place where rules and fiction intersect are the
character’s \textbf{Moves}. Moves are the mechanical structure used
when the fictional actions of a character require some resolution,
and where the outcome of such actions is sufficiently interesting -
or in doubt - as to be worth taking a risk to achieve.

\paragraph{Active Moves}
It is tempting to think of moves as a character’s ``powers'' or
``abilities'', but remember: you should not be looking for a move to
make. Instead, you should describe fictional actions that fit the
circumstances, and when those actions coincide with a move, that is
the point at which you engage the game mechanics to determine the
outcome.

For example, in a situation where Valentin, a street samurai, is
raiding a military compound, his player should not be looking to see
when he can bust out his Rock \& Roll move. Instead, Valentin’s player
should describe what Valentin is doing, and if what Valentin is doing
would fit the criteria for the Rock \& Roll move, then the player uses
those mechanics. Basically, it is the difference between this:

\begin{dent}

\textbf{GM:} \textit{A security guard moves into
  view. What do you do?}

\textbf{Keith (Valentin’s player):} \textit{I should use Rock \& Roll. I’ll lean
around the corner and shoot.}
\end{dent}

and this:
\begin{dent}

\textbf{GM:} \textit{A security guard moves into view, gun out, looking
for you. What do you do?}

\textbf{Keith:} \textit{I lean around the corner enough to bring my sights
to bear on him, and unload three rounds from my
HK227.}

\textbf{GM:} \textit{That sounds like the Rock \& Roll move, for sure. Roll
2d6 and add your Mastery stat.}
\end{dent}

\paragraph{Passive Moves}
That said, some moves do describe passive traits of a character,
either as innate qualities or as the result of specialized
training. These types of moves are generally distinguished by their
lack of rolls and simple passive changes to existing rules.

\subsection{Types of Moves}
There are four general categories of moves in
\SW/: \textbf{Core}, \textbf{Secondary}, \textbf{Metatype},
\textbf{Background}, and \textbf{Archetype}.

\begin{dent}
\textbf{Core moves} are the most commonly used moves, and
provide mechanics for frequent activities like
fighting, hiding, looking around, and interacting. Core moves are
described on page \pageref{coremoves}.

\textbf{Secondary moves} are less frequently used, and are usually
situational. Secondary moves are described on page \pageref{secondarymoves}.

\textbf{Metatype moves} are moves that reflect the differing traits
of the five human metatypes in the game. Metatypes and their  moves are described
on page \pageref{metatypes}.

\textbf{Background moves} are moves that reflect your life before the
shadows. Everyone starts somewhere and it leaves a mark. Backgrounds
and their moves are described on page \pageref{backgrounds}.

\textbf{Archetype moves} are moves unique to one
of the character archetypes, and reflect their particular skill
sets. Refer to each skillset (starting on page \pageref{skillsets})
for more information.

\end{dent}

Core, secondary, metatype, and background moves are detailed on the
following pages. Archetype moves can be found in the dossier for each archetype.

\subsection{CORE MOVES}
\label{coremoves}

\label{move_checkthesituation}
\textbf{CHECK THE SITUATION:} when you \textbf{assess a situation} or
\textbf{determine facts about your environment}, roll+\twitch/.
On 10+, you may ask the GM 3 of the following questions.
On 7-9, ask 1 question. Either way, take +1 if you act on the
answers.
\begin{dent}

\tcirc{} What is my best escape/access/evasion route?

\tcirc{} Which enemy is most vulnerable?

\tcirc{} Which enemy is the biggest threat?

\tcirc{} What is my enemy’s true position?

\tcirc{} What should I be on the lookout for here?

\tcirc{} Who’s really in control here?
\end{dent}

Note: you may ask any question you wish; however, the GM
is only obligated to give answers the questions from the list
above.

\label{move_dropscience}
\textbf{DROP SCIENCE:} when you \textbf{call on your knowledge of a
particular subject}, roll+\mastery/. On 10+, the GM tells you
something useful and interesting about the topic. On 7-9, the
GM simply tells you something interesting.

\label{move_fuckitup}
\textbf{FUCK IT UP / MAKE IT RAIN:} when you \textbf{aid or interfere
with someone you have Bond with}, roll+your Bond with
them. On 10+, they are boosted or glitched, your choice. On
7-9, they’re still boosted or glitched, but you are exposed to
danger or retribution.

\label{move_gutcheck}
\textbf{GUT CHECK:} when you \textbf{check off your last fatigue box},
roll+\oomph/. On 10+, you stay on your feet, and if the damage you
just received would take you beyond your last box, ignore
any excess. On 7-9, as above, but (choose 2):
\begin{dent}

\tcirc{} you are glitched

\tcirc{} you’ll pass out in a few moments

\tcirc{} you’re making it worse; First Aid moves to help you
take -1
\end{dent}

On a failure, you collapse unconscious. If you were taken down by stun
damage, you are merely unconscious. Otherwise, follow the rules for
GETTING HURT on page \pageref{gettinghurt}. You may require first aid (page
\pageref{move_firstaid}) to stabilize you.

\label{move_manipulate}
\textbf{MANIPULATE:} when you \textbf{have leverage over someone}
(something they need, want, or wish to hide) \textbf{and wish to
get something from them}, roll+\flair/. If the
person is an:
\begin{dent}

\textbf{NPC:} On 10+, they’ll ask you for something in return,
but will give you what you need now. On 7-9, they’ll
need to see some concrete assurance you’ll do what
they ask before they help you.

\textbf{PC:} on a 10+, both of the following apply. On 7-9, only
1 applies (you choose):
\begin{dent}
\tcirc{} If they comply, they get to mark XP.

\tcirc{} If they refuse, they have to Stay Frosty.
\end{dent}
\end{dent}

\label{move_makethemsweat}
\textbf{MAKE THEM SWEAT:} when you \textbf{impose your will on someone
  by threat of force}, roll+\oomph/. On 10+, they comply without
argument. On 7-9, they comply, but (choose 1):
\begin{dent}

\tcirc{} They look for payback

\tcirc{} They do only the bare minimum

\tcirc{} They tell someone else about it
\end{dent}

\label{move_rockandroll}
\textbf{ROCK \& ROLL:} when you \textbf{attack an enemy who can defend
  themselves}, roll+\oomph/ (or \mastery/ if ranged). Determine the
result based on the type of attack, as follows:
\begin{dent}

\textbf{Melee Attacks:} on 10+, you hit and deal damage. On 7-9,
you deal damage, but your target attacks you as
well.

\textbf{Ranged Attacks:} on 10+, you hit and deal damage. On
7-9, you hit, but (choose 1):
\begin{dent}

\tcirc{} You need to expose yourself to danger

\tcirc{} You burn up ammunition. Mark off 1 ammo.

\tcirc{} You only graze the target (-2 damage)
\end{dent}
\end{dent}

\label{move_stayfrosty}
\textbf{STAY FROSTY:} when you \textbf{act despite imminent danger,
fear, or risk}, you must roll. The stat you add depends on
how you’re addressing the risk. If you’re:
\begin{dent}

\tcirc{} staying alert and reacting quickly,
roll+\twitch/

\tcirc{} hoping you’re tough enough mentally or physically to
weather the storm, roll+\oomph/

\tcirc{} banking on your skill or knowledge,
roll+\mastery/

\tcirc{} flashing a smile or banking on charm,
roll+\flair/
\end{dent}

On 10+, you succeed. On 7-9, you succeed, but the GM will
present you with a choice: a worse outcome, hard bargain,
or ugly choice.

\label{move_takeabullet}
\textbf{TAKE A BULLET:} when you stand in defense of another,
roll+\oomph/. On 10+, the attack hits you instead. On 7-9, you
take half the damage.

\subsection{SECONDARY MOVES}
\label{secondarymoves}

\label{move_advance}
\textbf{ADVANCE:} when \textbf{you have downtime, and have marked
10 XP}, you can spend time reflecting on your experiences
and honing your skills. When you Advance, choose one of
the following:
\begin{dent}

\tcirc{} Advance a stat (each stat may be advanced one time,
fill the small triangle on the dossier when you’ve
advanced a stat)

\tcirc{} Gain a new move from any skill sets on your dossier

\tcirc{} If you have at least two moves from each of your current
skill sets, you may elect to add an additional Archetype as a new
skill set and choose one move from it.

\tcirc{} Gain 1 Edge

\end{dent}

You may only choose one benefit each time you advance.
However, you can choose a benefit multiple times, subject
to the limits specified above. Once you have advanced, clear
your XP track.

\label{move_lastchance}
\textbf{LAST CHANCE:} when \textbf{you are facing death and out of
options}, permanently sacrifice at least 1 Edge and roll+the
amount sacrificed. On 10+, you miraculously make it through,
and it’s not as bad as it looked. On 7-9, you make it through,
but you must agree to a painful bargain. On 6 or less...it’s all
over. Edge sacrificed for this move is gone until you earn it
back; it does not refresh with rest as usual.

\label{move_citationneeded}
\textbf{CITATION NEEDED:} when you \textbf{research a topic, person,
business, or location}, roll+\mastery/. On 10+, you spend 1 day
searching, and locate a useful detail about the topic of the
research. On 7-9, you locate a useful detail, but
(choose 1):
\begin{dent}

\tcirc{} you end up in a rabbit warren of information; spend 1
additional day digging through it

\tcirc{} your search raises a flag in someone else’s systems (the
GM determines whose)

\tcirc{} the information is in hardcopy, and you need to go to it;
spend 1 additional day on the search
\end{dent}

\label{move_firstaid}
\textbf{FIRST AID:} when you \textbf{try to keep a teammate from dying}
from their wounds, roll+\mastery/. On 10+, you stabilize your
teammate. On 7-9, you stabilize them, but (choose
1):
\begin{dent}

\tcirc{} you can’t move them to cover

\tcirc{} you expose yourself to danger (take 2
damage)

\tcirc{} their wounds force you to Stay Frosty
\end{dent}

On a failure, you can't help them and your teammate cannot be saved.

\label{move_goshopping}
\textbf{GO SHOPPING:} when you \textbf{hit the market to buy legal or
  illegal items}, roll+\flair/. On 10+, you find what you need: if
it’s a legal item, you’ll have it in 1 day; illegal items take 2
days. On 7-9, you can get it, but you must wait 1d4 additional days.

\label{move_hitthebooks}
\textbf{HIT THE BOOKS:} when you \textbf{spend time training,
  practicing, or studying your abilities}, you gain Prep. You gain 1
Prep for every 2 days spent in training or practice. When that
training and preparation pays off, you can spend 1 Prep to get +1 to
any roll. You can only spend 1 Prep per roll.

\label{move_overwatch}
\textbf{OVERWATCH:} when you’re \textbf{providing cover for an ally
  and a threat appears}, roll+\twitch/. On 10+, your ally gets the
drop on the threat. On 7-9, they’re alerted, and take +1 on their next
move. On a miss, the threat gets the drop on your ally.

\label{move_poppills}
\textbf{POP PILLS:} when you \textbf{indulge in a drug},
roll+\essence/. On a 10+, you experience the effects as normal. On
7-9, you experience the effects but you got a weak batch, so the
effects last half as long.  If you roll snake eyes when you pop pills,
you become addicted to the drug. If you go 3 sessions without a hit,
roll 2d6w. If you roll a 4 or higher, you are no longer addicted;
otherwise, you’re still hooked. If you are an addict and roll snake
eyes while popping pills, you overdose and either:

\begin{dent}
  
  \tcirc{} Take full fatigue

  \tcirc{} Loose 1 point of essence
  
\end{dent}

\label{move_pullstrings}
\textbf{PULL STRINGS:} when you \textbf{hit up a contact for info or
  assistance}, roll+\flair/. On 10+, the contact provides useful
information (related to their own knowledge) or assistance.  On 7-9,
the contact provides information or assistance, but (choose 1):
\begin{dent}

\tcirc{} Has to get back to you; wait 1 day

\tcirc{} Isn’t happy about it; take -1 forward to the next time
you Pull Strings with this contact

\tcirc{} Requires a favor in return
\end{dent}

If you fail, your contact doesn’t want to see you for a while, and
will not return calls or meet with you for 1d6+1 days.  Repeated
failures of this move can permanently sever your relationship.

\label{move_suppressionfire}
\textbf{SUPPRESSION FIRE:} when you \textbf{suppress an area to pin
  the enemy down down}, roll+\mastery/ and mark off 2 Ammo. On 10+,
the targets are suppressed and cannot move or return fire. On 7-9, the
targets are suppressed, but deal 2 damage first.


\subsection{METATYPE MOVES}
\label{metatypes}

There are five primary metahuman types (or ``metatypes'') in the \SW/:
\textbf{Human}, \textbf{Dwarf}, \textbf{Elf}, \textbf{Ork}, and
\textbf{Troll}, each with their own unique moves. When you choose your
metatype, you also choose one move from the list as your metatype
move.

While there are regional differences in the appearance and
nature of metatypes, such as the trollish Oni in Japan and the
elvish Dryad in England, all metahumans have access to the
same moves.

Additionally, if there are other metatypes or species you wish
to add to the game, don’t hesitate: just name the metatype,
and come up with a move or two for it (or just lift one from
the list here).


\subsubsection{HUMAN}
\label{metatype_human}

Humans are still the majority of all metatypes.

\begin{dent}

\textbf{ORDINARY:} when trying to remain unnoticed, you blend into
crowds easier.

\textbf{PROFESSIONAL:} choose an area of knowledge or training.
When you Drop Science about that area of expertise, you
are boosted.

\textbf{PRIVILEGE:} when interacting with humans, take +1 to \flair/ moves.
\end{dent}

\subsubsection{DWARF}
\label{metatype_dwarf}
All dwarves have natural thermographic vision.

\begin{dent}

\textbf{TONIGHT WE DRINK:} if you’re drinking with someone, you
may manipulate someone using \oomph/ instead of
\flair/.

\textbf{NEVER SICK:} you are immune to disease and
poisons.

\textbf{SAVVY:} when you repair or improve machines, you are
boosted.
\end{dent}

\subsubsection{ELF}
\label{metatype_elf}
All elves have natural low-light vision.

\begin{dent}

\textbf{FANCY:} when attempting to blend into high society, take +1 to
\flair/ moves.
  
\textbf{UNCANNY GRACE:} Each day, hold 2. Spend this hold to reduce
damage from an attack by half before considering dodge and armor
bonuses. This resets each day.

\textbf{ETHEREAL:} when manipulating someone via charm or seduction, you are boosted.
\end{dent}

\subsubsection{ORK}
\label{metatype_ork}
All orks have natural low-light vision.

\begin{dent}

\textbf{‘ARD BASTARD:} take +1 to gut checks

\textbf{STREETFIGHTER:} the first time you attack an enemy with a
nonlethal weapon (fists, feet, batons, etc), you
are boosted.

\textbf{FEARLESS:} take +1 to stay frosty in the face of fear.
\end{dent}

\subsubsection{TROLL}
\label{metatype_troll}
All trolls have natural thermographic vision.

\begin{dent}

\textbf{DERMAL BONE PLATING:} you have +1 armor.

\textbf{YOU’LL JUST MAKE IT ANGRY:} you gain 1 additional fatigue
box.

\textbf{JUGGERNAUT:} your fists should be licensed weapons. You
deal lethal damage in unarmed combat.
\end{dent}

\subsection{BACKGROUND MOVES}
\label{backgrounds}
Your background, in combination with your skills and metatype, is key
to defining who your character is. People come to the shadows for all
sorts of reasons and from all walks of life. Life in the shadows
attracts the desperate and adventurous, provides opportunity to the
ambitious and greedy, acts as a shelter to the pursued and cast-out,
and grants freedom to the anarchists and those tired of a controlled
modern existence. Your character's background is flexible and open. It
should primarily add color and flesh out a concept into a real,
breathing person with a complex history. This section provides some
examples of backgrounds, along with potential moves these life
experiences may provide a character. However, don't feel limited to
these backgrounds or moves - work with your GM to develop the perfect
background for your character.

\subsubsection{The Researcher}
\fluff{Escaped, or ex-filtrated, you used to work for a megacorp as a well
known and respected researcher. Now you run the shadows, using your
web of academic contacts and cool logic to get you through the night.}

\textbf{PEER REVIEW:} When you use CITATION NEEDED, you may
roll+\flair/. 


\subsubsection{The Ex-Cop}
\fluff{You were the law. Emphasis on \textit{were}. Perhaps you did corporate
security, enforced the will of governments, or were simply a lowly
beat cop. Regardless, now you inhabit the same world that you once
opposed. Adversaries are now potential allies, and your former
employers now valuable targets.}

\textbf{HEY BUDDY:} When you approach a former associate for a
favor, roll+\flair/. On a 10+, you get what you were after. On a 7-9,
also choose one:
\begin{dent}
  \tcirc{} Your contact is pissed that you asked for help, given your
  current \textit{situation.} Gain a \textbf{debt} towards this
  person.

  \tcirc{} The situation in your old life is not quite what you
  remembered and your request provokes unintended consequences.
\end{dent}

\subsubsection{The Wage Slave}
\fluff{Wake up in friendly corp apartments, consume friendly corp food, work
at your friendly corp desk, sleep in your friendly corp bed. These are
the things you remember from your former life. On those bad days you
sometimes look back on them fondly. On the good days, well, those were
the good days, right?}

\textbf{COG IN THE MACHINE:} You are an expert at looking normal. When
you try to blend into a corporate environment, you are boosted.


\subsubsection{The Hippy}
\fluff{Peace. Love. Hard, hard drugs. You're not really sure how you
  got involved in the shadows. It could have been last year or five
  years ago. Your memories are a little fuzzy. What you do know is
  that people really need to get along a little more. Lacking that,
  perhaps some self-medication is called for.}

\textbf{PEACE PIPE:} When you POP PILLS, gain 2 hold. You may spend 1
hold to boost any \flair/ move while you are still high.

\subsubsection{The Trustfund Baby}
\fluff{The trids always made runners look so wiz. You grew up watching
everything you could find on their legendary exploits. When you got
old enough, you cashed out your inheritance and bought some gear. Now
you're a runner.}

\textbf{BEGINNERS LUCK:} When you try something new for the first
time, you are boosted.

\subsubsection{The Street Rat}
\fluff{The streets were always your home. From the first moment you
  could remember, you were running. Of course, back then it was
  generally away from trouble. Now? Trouble runs from you.}

\textbf{CONCRETE PLAYGROUND:} When you navigate through an urban
environment, roll+\mastery/. On a 10+, you are able to find a shortcut
invisible to everyone else. On 7-9, your shortcut exists, but there is
an unexpected complication.

\subsubsection{The Disillusioned}
\fluff{The world doesn't care about you. Maybe it once did. Maybe
  society, long ago, looked out for its members. Today, we are all
  just meat. Meat to be ground up and consumed by governments and
  megacorps, used to fuel their power plays. At one point you
  cared. Not anymore.}

\textbf{UNNERVING:} When you MAKE THEM SWEAT, you can use your
emotionless detachment to use \flair/ instead of \oomph/.


\subsubsection{The Artist}
\fluff{It was hard for a while. Your muse gone, your creativity burnt
  away like grey ashes. Turning to the shadows was partially escapism,
  partially desperation. But since you've been running, you find your
  muse has returned. Of course, now your modes of expression are a
  little more... unconventional.}

\textbf{A BEAUTY TO BEHOLD:} When you perform, and describe, your next action in a
aesthetically pleasing fashion, it is boosted.


\subsubsection{The Activist}
\fluff{It was important to you like it was to no one else. The
  cause. Your cause. You did all you could legally, but it wasn't
  enough. So now you run the shadows. Here you continue the fight, but
  without the rules that stopped you before.}

\textbf{THE FIGHT ISNT OVER:} When you LAST CHANCE, take +1 if you were actively
pursing your cause before you had to make the move.

\subsubsection{The Native}
\fluff{The power of the land is all around you. You can see it in
  earth, in your family, your tribe, and in yourself. When the Great
  Ghost Dance helped your people return to the land they came from,
  everything seemed like it was all going to work out after all. But
  that was naive thinking. You can see that now. Keeping your land,
  people, and traditions requires you to fight for them. The corps and
  governments are too big to take on directly, but you can always
  strike at them from the shadows.}

\textbf{ONE WITH THE LAND:} When you need to survive off the land,
roll+\mastery/. On a 10+, you find food, water, and shelter for
yourself and your companions. On 7-9, choose two out of the three.

\subsection{CROSS-ARCHETYPE MOVES}
\label{crossarchetypemoves}
Archetypes are, in effect, the character classes in \SW/.
However shadowrunners need to be adaptable to survive and often pick
up skills from other archetypes over the course of their careers. To
represent this, the \SW/ uses the concept of Micro-Archetypes, where
each archetype provides a relatively limited, but focused set of
moves. Players are allowed, even encouraged, to ``multiclass'' their
characters by choosing moves from more than one archetype, with some restrictions.

When you make the Advance move, you have the option of
selecting a move from another archetype. You can choose
moves freely from other archetypes, subject to the following
two restrictions:
\begin{dent}

1. You must have at least two moves from all current archetypes before
adding a move from a new one. This requirement exists for two reasons:
First, it encourages players to design more focused, thematic
characters by carefully considering their makeup, and second it
discourages artificial min-maxing by requiring some depth from each
archetype chosen.

2. Your character is restricted to their supernatural
``class''. Mundane characters cannot take moves from Awakened or
Technomantic archetypes. Likewise, Awakened and Technomantic
archetypes cannot take moves from each other. However, they can take
moves from Mundane archetypes. 

\end{dent}

Of course, both restrictions are entirely subject to GM and
group discretion.

Restriction \#2, for example, can be modified easily if the group
wishes all characters in their game to have some magical potential. Or
if players would like to have their characters develop magical or
technomantic abilities mid-campaign, GM's might wish to use the
optional Awakening/Emergence rules later in this book.

\end{multicols}

\newpage
\invisiblepart{CHARACTER CREATION}
\section{CHARACTER CREATION}
\label{charactercreation}
\begin{multicols}{2}

Creating a character is a multi-step process (don’t worry,
though, it’s pretty easy) of filling out a personal dossier. The overall process is described
here; while specific details can be found in each Archetype’s skill set.
You’ll record the details you create on the dossier page or the
supplemental ``extra info'' page located on page 28.

\paragraph{1.  Choose your Primary Archetype}

There are many Archetypes to choose from. Each one represents a small,
focused character concept. Archetypes can be divided into three major
categories: Mundane, Awakened, and Technomantic. Awakened archetypes
represent characters with magical abilities, while Technomantic
archetypes represent different forms of technomancy. Your first
archetype is considered your Primary Archetype and helps define any
initial gear or extra features. List this archetype under the
\textbf{Role} section of your personal dossier and add it as your
first skill set.

\subparagraph{Mundane}

\begin{dent}
\begin{description}
\item[Face] A skilled people person
\item[Professional] Trained and dedicated
\item[Hacker] Matrix cowboy
\item[Infiltrator] Getting in, Getting out
\item[Driver] Has wheels, will travel
\item[Mercenary] Everything has its price
\item[Ganger] Territory. Brotherhood. Blood.
\item[Street Doc] What can't be fixed, can be replaced
\item[Street Samurai] Honor and steel
\item[Radical] Anything for the cause
\item[Investigator] The truth is never hides for long
\end{description}
\end{dent}

\subparagraph{Awakened}

\begin{dent}
\begin{description}
\item[Mage] Spellslinger extrodinare
\item[Shaman] Communes with spirits
\item[Adept] Cyberware? Pfff. Who needs it.
\end{description}
\end{dent}

\subparagraph{Technomantic}

\begin{dent}
\begin{description}
\item[Botmaster] Speaks with machine minds
\item[Architect] Shapes the Matrix at will
\item[Infovore] Collects and Connects
\end{description}
\end{dent}

For more detail, consult the Archetypes section, page \pageref{archetypes}.

\paragraph{2.  Choose your Metatype}

There are 5 metatypes: \textbf{Human}, \textbf{Dwarf}, \textbf{Elf}, \textbf{Ork}, and \textbf{Troll}.
Each metatype offers a choice of Metatype Moves. Choose
one move from the \textbf{Metatype Moves} section, page 6. List your
metatype and chosen move in the \textit{Metatype} section of the dossier.

\paragraph{3.  Choose your Look}

Each character primary archetype will present options for look; you
are free to make up your own as well.

\paragraph{4.  Choose your Name and Street Name}

Pick a real name and street name. You may use the lists provided in the \textbf{GM Resources} section on page 72, or create
your own. Your street name can be flashy or subtle, its really up to
how you want your character to be seen in the \SW/.

\paragraph{5.  Assign your Stats}

All characters have the following stats: \oomph/, \twitch/,
\mastery/, \flair/, and \essence/.

All core stats start with a modifier of +0.

\paragraph{6.  Spend your Build Points}

You have \textbf{4 build points} to distribute among your stats. To
increase a stat by 1 point costs 1 Build Point (e.g., it is a
straight 1-for-1 cost).

You may increase a stat to a maximum of +2 as a starting
character. Additionally, if you wish, you may lower one stat
to -1 in order to gain an additional Build Point to spend elsewhere.

You may, if you choose, reserve one build point to later add
an additional starting move. This allows your character to have a
larger variety of skills, at the cost of less likely hood of success.

Each archetype has its own affinity with certain stats. Please consult
your moves to determine the most advantageous configuration, or choose
whatever feels right for your character idea.

\paragraph{7.  Set your Edge}

Depending on your primary archetype, you start with a varying
amount of Edge. Note this amount on your character sheet.

\paragraph{8.  Choose Equipment}

Each primary archetype will present various weapon, spell, cyberware,
and equipment options. Choose from the suggested items, or if you want
to create your own equipment, use the equipment creation rules
starting on page 60 to customize your kit.

Note: Choosing cyberware at character creation still costs points of
essence. Reduce your starting essence appropriately for the cyberware chosen.

If you choose cyberware, and one of the options provides a
capability you already have (such as thermographic vision),
you may exchange it for any equivalent ability or other item;
just check with the GM.

\paragraph{9.  Choose Contacts}

Everybody knows somebody. You will be presented with a list of
potential contacts your character might know as a result of their
experiences both before and after they became shadowrunners.

\paragraph{10.  Establish Debts and Favors}

In your life before and after becoming a shadowrunner, you’ve worked
with a lot of people, and ended up owing, or being owed, by
them. These relationships include at least one of your fellow
shadowrunners, and are called \textbf{debts} and \textbf{favors}.
When you are instructed to create your debts and favors with fellow
runners, you’ll see a list of sample statements to help you create
them. You don’t have to use these; they’re simply suggestions.

To create a debt or favor, place the name of one of the other
characters in the blank space in one of the statements presented. You
can place the same name more than once (that is, in more than one
sentence), but you must establish at least one debt or favor to start
with.

Collectively, debts and favors are known as \textbf{bonds}. Later,
during play, you may end up resolving a bond with someone.  If you do,
both of you mark XP.

\paragraph{11.  Starting Skill Sets}

Your character knows all the Core and Secondary Moves. Additionally,
your primary archetype provides your character a special move that
only characters with the same primary archetype can use. You also know
one other move from your primary archetype.

At this point, you may optionally add one additional Archetype as a
skill set (remember to follow the rules described in CROSS ARCHETYPE
MOVES on page \pageref{crossarchetypemoves}). Add this archetype to your skill sets, and choose one move
from this archetype. If you reserved build points from Step 6, you can
spend them now on additional moves from any skill set you have.


\paragraph{12.  Advancement}

Each time you fail a roll - that is, you roll a 6 or less - you mark
XP. When you mark 10 XP, and you have downtime, you can
make the \textbf{Advance} move (page 5).


\end{multicols}

\newpage
\switchToLayoutPageA{}
\invisiblepart{COMBAT}
\section{COMBAT}
\label{combat}
\begin{multicols}{2}

Shadowrunners tend to get themselves into lots of trouble, the kind
that ends with some high-intensity interpersonal conflict
resolution. In other words, combat. As you’ll find when you read
through the rest of this document, most of combat (in fact, pretty
much everything the player characters do, ever) is handled through
the application of various moves as they intersect with the
fiction. This section explains a few specific quirks of combat in
\SW/.

Remember: although you’re reading a section titled ``Combat,'' there’
no point at which the game switches to ``combat rounds,'' and nobody
rolls initiative. In other words, there’s no true division between
combat and everything else that happens in \SW/. Since everything
flows from the game fiction and returns to the game fiction, combat is
just another part of the regular flow of the game.

\subsubsection{ARMOR \& DODGE}
Because a shadowrunner leads a dangerous life, a big premium is put on
not getting hit or at least not taking all the damage. The obvious way
to do so is to wear armor. In \SW/, armor reduces incoming damage on a
1 for 1 basis.  Of course, adding bulky body armor isn't the only way
to avoid damage in the middle of combat. Being light and nimble is
often just as effective as being a walking tank.

This is represented by a character's \twitch/ stat. Every point of
\twitch/ a character has allows them to avoid a point of damage just
like armor by dodging. Unfortunately, heavy armor and dodging don't
mix very well. Every point of armor a character is wearing adds a -1
penalty for any \twitch/ based move.

Note: A character with negative \twitch/ (modified or not) doesn't
take extra damage from attacks against them, though the DM is free to
take your lack of physical dexterity into account when they make their
moves. From a game fiction perspective, the other tradeoff, of course,
is that you can’t spend all day walking around in combat armor—it’s
hot, itchy, intimidating, and cops tend to notice.

Some metatypes and archetypes offer moves that let you reduce damage,
or otherwise avoid some of the less pleasant outcomes of damage. For
example, the \textit{‘Ard Bastard} move (an ork metatype move) lets
the character take +1 to gut checks, and the troll move \textit{You’ll
  Just Make It Angry} grants an additional fatigue box..


\subsubsection{SURPRISE}
The \textit{Rock \& Roll} move and most other damage-dealing moves
assume that your target can fight back. If that’s not a possibility
(that is, if your target is surprised, helpless, etc.), the fiction
can’t trigger the \textit{Rock \& Roll} move. You just put a round in
their head and move on.

When you get the drop on someone in combat, you don’t need to use a
move to deal damage to them—you can simply deal your damage (or kill
them outright, depending on the situation). Likewise, if someone gets
the drop on you in combat, expect to eat some lead.


\subsubsection{FIRE MODES}
Weapons in the game can fire in semi-automatic, burst, or
full-auto modes, depending on their specific capabilities.
Semi-auto is the ``default'' assumption; in that mode you only
use up ammunition when you roll 7-9 on the \textit{Rock \& Roll}
move, and choose to burn extra ammo.

Firing in \textbf{burst} or \textbf{auto} modes when using
\textit{Rock \& Roll} allows you to add +1 damage to your attack;
however, it always uses 1 ammo (even if you roll 10+).

Finally, full-auto mode is very useful for suppression fire, and
lets you take +1 when you use the Suppression Fire move.


\subsubsection{RELOADING}
Most of the weapons indicate some ammo capacity using the ammo tag -
this indicates how much ammunition a weapon can carry in its magazine
or clip before it must be reloaded.  If a weapon has 3 ammo, for
instance, you have ammunition in the gun until you have marked off all
three ammo.  Ammo is an abstraction - 1 ammo does not represent a
single round, but simply ``some ammunition.'' The game assumes (for the
most part) that a character fires multiple shots in a single move.

During combat, assume that combatants are reloading their weapons when
appropriate, keeping them topped up. Mechanically, this is handled by
the fact that \textit{Rock \& Roll} doesn’t cost ammo unless you roll
a 7-9, and choose to burn up extra ammo (or if you use burst or
full-auto weapons).

When you mark off all your ammo, you’ll need to reload.
There is no specific move to reload a weapon. If taking the
time to reload would not expose you to danger, then you
can reload simply by saying so. On the other hand, if you’re
reloading despite an imminent risk, that’s a job for the Stay
Frosty move.

\subsubsection{LIGHT AND SOUND}
You’ll note in the Metahuman Moves section that some metahumans have
the ability to see either in low-light, or see into the infrared (and
you’ll also note in the Cyberware section that cyberware can grant
similar abilities). At the GM’s discretion, he or she may establish
that the area the characters are in has low visibility due to one of
the following factors, and impose modifiers on players’ rolls. There
are four visibility options:
\begin{dent}

  \textbf{Darkness:} both low-light and thermographic vision allow
  normal vision in dark environments. Characters with normal vision
  must use a light or take -1 ongoing as long as it remains
  dark. \textbf{Note:} low-light vision is ineffective in truly
  complete darkness, and no vision type is effective in supernatural
  darkness.

  \textbf{Smoke/Fog:} characters with normal or low-light vision take
  -1 ongoing while the smoke or fog persists; characters with
  thermographic vision suffer no vision difficulties.

  \textbf{Glare/Flash:} in circumstances of very bright light, all
  characters without some sort of compensation (sunglasses, or flare
  compensators for things like flash-bang grenades) take -1 ongoing
  until they recover or compensate from the bright light.
\end{dent}

As with vision, it’s important to be able to hear in combat.
In a very noisy environment (a factory, an active airstrip, etc.)
or in the event of intensely sharp or loud noises (flash-bangs,
explosions, even sustained gunfire), the GM may impose -1
forward or -2 forward penalties. Certain cyberware (such as
frequency filters or dampers) or protective equipment like
earplugs can eliminate these penalties.


\end{multicols}

\invisiblepart{DAMAGE AND HEALING}
\section{DAMAGE AND HEALING}
\begin{multicols}{2}

  Inevitably, when you play with guns, magic, and sensitive secrets,
  somebody is going to get shot. Or burned, or hit with a brick, or
  drenched in elemental acid summoned from beyond the realm of mortal
  ken, or thrown out a window, or...well, you get the point.

  In any case, damage will be given and taken, and quite possibly end
  with someone being little more than yesterday’s garbage.

\subsection{DEALING DAMAGE}

When you make a move that has the potential to deal damage, the move
will usually say, as a possible result, ``deal your damage'' or ``you
deal damage.'' Damage in the game is usually variable, based on the
damage dice for the weapon being used (see the \textbf{Equipment}
section for information on weapons). This is the amount of damage that
is applied to your target.

\begin{dent}

  \textbf{Example:} \textit{Johnny Chopz hits a ghoul with his trusty
    katana. The katana deals 2d6b damage (meaning roll 2d6, and take
    the best result). Johnny’s player rolls 2d6, getting 3, 5. Thus,
    the attack deals 5 damage to the ghoul. Bad news, creep.}
\end{dent}

If a move indicates that you deal half damage, roll the damage as
normal, and then divide the result in half (rounding up) to get your
final damage amount.  The most common situation in which you’ll deal
half damage is if you’re shooting at a vehicle with small
arms. Vehicles take only half damage (before armor) from small arms,
and no damage from melee weaponry.

\begin{dent}

  \textbf{Example:} \textit{Johnny is being chased down by a
    go-ganger, and turns to shoot at the onrushing psycho with his
    Ares Predator. When he rocks \& rolls with the ganger, he’s able
    to deal his damage (1d8+1) and wants to hit the vehicle, not the
    ganger. He rolls 5 damage. Halving that yields 3 damage (5 ÷ 2,
    rounded up) means that a bullet just gets through the armor, but
    it ain’t gonna help. If he’d pulled out his katana and stood his
    ground...well, what would happen is that he’d end up with a
    motorcycle wheel up his nose.}
\end{dent}

\subsection{GETTING HURT}
\label{gettinghurt}

\subsubsection{GUT CHECKS}
When a character takes damage in the game, it is recorded by marking
\textbf{fatigue boxes} the character’s playbook. Fatigue is an
abstract measure of a character's exhaustion and minor damage during
combat. All characters have at least four boxes of fatigue. Additional
boxes are provided by their \oomph/ stat at a one to one rate, though
negative values of \oomph/ don't decrease fatigue boxes. Additionally,
some racial moves, archetype moves, magic, or cyberware can increase
available fatigue boxes. As long as a character has available fatigue
boxes to fill in, taking damage is generally no big deal. However, a
character might need to take a GUT CHECK when:

\begin{dent}
\textbf{Last Wound:} when you check off that last box of your
Fatigue track, you must make the Gut Check move.

\textbf{Major Trauma:} if you take 6 or more damage (after applying
armor) in a single hit, you have just taken Major Trauma. You will
need to make the Gut Check move.
\end{dent}

When you run out of fatigue boxes, you are completely exhausted and combat
starts to get a lot more dangerous. The maximum fatigue boxes a
player's character can have is eight.

\subsubsection{PHYSICAL DAMAGE}

Most weapons in the game deal physical damage; when taking damage from
this kind of weapon, mark off a number of boxes on the Fatigue Track
equal to the damage taken. Getting dealt 3 damage, for instance, would
mean that (all else things being equal) the player would mark 3 Wounds
on their playbook. This value is reduced by a character's armor and
ability to dodge, as covered earlier in this section. Once your
fatigue is exhausted, physical damage starts to really hurt.

\subsubsection{STUN DAMAGE}

If a weapon specifies that it deals stun damage, you still check off
boxes on the Fatigue Track. However, if a weapon dealing stun damage is
the one that takes the last fatigue box (or if you don't have any
boxes remaining), you are knocked unconscious.



\subsubsection{BLEEDING OUT}
Once a character takes physical damage beyond their maximum number of
fatigue boxes, they are \textbf{Bleeding Out.} This basically means
they're on the verge of taking some serious damage.

All characters have six bodily areas that can be damaged severely as a
result of heavy physical damage: the head, torso, two arms, and two
legs. Each area has a durability rating which is, for unmodified
metahumans, zero, except for the body which naturally starts at
one. These ratings can be increased with magic or
cyberware. Durability acts as a final level of armor protecting each
region. Unlike fatigue boxes, each box of durability can absorb the
entirety of a single attack (or the results of an attack which
overflows your fatigue).

However, if you take damage to an area with no durability remaining,
you take a severe, debilitating injury to that area. The precise
definition of this term is variable, as are the effects, and depend on
the fiction of the game. The GM is encouraged to make such injuries
difficult to deal with, potentially immediately life-threatening
necessitating stabilization from FIRST AID or a Trauma Patch and/or
requiring long term healing. In short, severe injuries are best
avoided at all times.

Note: There are two bodily areas which are considered critical: the
head and the torso. These areas, if a severe injury is taken there,
automatically require stabilization.

\subsubsection{CHRONIC INJURY}
If a character reaches the Bleeding Out stage, takes a severe injury,
and survives their precarious situation, they will be left with a
\textbf{Chronic Injury.} This is a long-term (and possibly permanent)
reminder of their brush with death.

Chronic injuries are generally ruled by fiction more than anything
else. They can be anything from lasting psychological damage, missing
limbs to the loss of senses (blindness, deafness, etc). The GM is
encouraged to make severe injuries and the resulting chronic injuries
particularly annoying. 

\subsection{GETTING BETTER}
\subsubsection{HEALING FATIGUE}
Simple fatigue damage is fairly simple to heal. At the end of an encounter,
scene, or situation (in other words, once the character has a chance
to take a breather), their damage is healed.

\subsubsection{HEALING SEVERE \& CHRONIC INJURIES}
Chronic Injuries are not necessarily permanent injuries, unless the
player wishes them to be. However, they can only be healed or
ameliorated by major or long-term treatment. A chronic physical injury
may be fixed via cybernetic replacement, for instance, which is a
major surgical intervention.  Chronic psychological injury may require
therapy over a long term as well.

It is up to the GM and players to negotiate the specific plan
for removal of a Chronic Injury. It may be that recovery may
evolve into a shadowrun of its own, but that is not required:
spending funds to pay for therapy, new cyberware, surgery,
or the like is sufficient if you want to keep the story of the
recovery as background events.

\subsection{GETTING BURIED}
With the rules covering stabilization, chronic injury, armor, and so
forth it’s actually fairly hard to all-the-way die in \SW/. However,
it can happen in a few different ways.
\begin{dent}

\textbf{Failed to Stabilize:} if the person attempting to provide
First Aid to Bleeding Out character fails their move, the
wounded character cannot be stabilized, and dies at the
end of the encounter.

\textbf{Continued Damage:} if a character takes more than one severe
injury in an encounter, they’re too badly mangled to be
saved. Players, understand that this can happen; GM’s, be really
careful with this one.

\textbf{Overwhelming Kaboom:} if a character is hit with an attack of
such overwhelming power that surviving it strains all credulity,
they’re killed immediately. For example, if a character is, say, hit
by an antiship missile, or falls into a crucible of molten iron...just
forget it, they’re gone.

\end{dent}

\end{multicols}

\newpage
\invisiblepart{ARCHETYPES \& SKILLSETS}
\section{ARCHETYPES \& SKILLSETS}
\label{archetypes}
\begin{multicols}{2}

\subsection{MUNDANE}

\subsubsection{Face}
The \textbf{Face} represents a combination of the frontman and the
grifter. When a shadowrunner team needs to do a deal, talk themselves
into a complex, slip into complex social situations -  they turn to
the Face to get it done right.

\paragraph{Critical Stats:}
\flair/, \mastery/

\begin{tcolorbox}[title=Special Move]
\movename{DON THE MASK:} When the Face \textbf{attempts to slip into a new social role},
roll+\flair/. On a 10+, they are able to present a convincing new
version of themselves and hold 3. While maintaining the role, the Face
can spend the hold to gain a +1 on any move required to keep the act
believable. On a 7-9, the act is good, but not perfect. Gain 1
hold. On a failure, you can't be sure if you're convincing or not.
\end{tcolorbox}


\movename{Razor Insight:} when you \textbf{have a casual conversation
  with someone to extract information}, roll+\mastery/. On 10+, you
learn three of the following things. On 7-9, you learn 2.
\begin{dent}
  \tcirc{} Something they love
  
  \tcirc{} Something they lost
  
  \tcirc{} Something they fear
  
  \tcirc{} Something they took
  
  \tcirc{} Something they need
\end{dent}
If you use this information when \movename{fast talking},
\movename{manipulating}, or \movename{making them sweat}, you are
boosted.

\movename{Fast Talk:} when you \textbf{try to convince somebody of
  something quickly in a pressure situation}, roll+\flair/. On 10+,
your quick thinking gets you through. On 7-9, they’re convinced, but
their suspicion is raised. Take -1 ongoing with any further dealings
with them.

\movename{Work the Angles:} when you \textbf{manipulate someone}, take +1.

\movename{Come Hither:} when you \textbf{attempt to seduce someone},
roll+\flair/. On 10+, they’re into you, and you can get a favor from
them or get access to some of their personal stuff.  On 7-9, they’re
into you and will provide minor help, but it will take some more time
and \textit{personal attention} to get a favor from them.

\movename{Crazy Smooth:} when you \movename{FAST TALK}, you are boosted.

\movename{Mind Trick:} when you \textbf{use your sheer force of
  personality to convince someone to take an action}, roll+\flair/. On
a 10+, you can convince the person to follow your instruction and they
think nothing of it. On a 7-9, they follow your instruction but
realize their manipulation moments later. 

\movename{Honeyed Words:} when you \movename{MAKE THEM SWEAT}, you may
roll+\flair/ instead of \oomph/.

\movename{Chameleon:} when you \textbf{attempt to blend in to a social
  environment}, roll+\flair/. On 10+, nobody questions your
presence. On 7-9, you catch the eye of someone who becomes curious
about what you’re doing there.

\movename{Irresistible:} even if you \textbf{anger, insult, or
  otherwise tick off a contact}, they just can’t stay mad at you. They
only avoid you for half as long as normal.



\subsubsection{Professional}

The \textbf{Professional} is a skilled domain expert. Professionals
come from all backgrounds and have turned their unique skill sets to
an advantage in the shadows. Professional characters often are seen as
a the smart guy of the group, along with the Hackers and Street Docs.

\paragraph{Special:}
Before taking any moves from this skillset, you must choose a
\textit{domain specialty}. This is the area of knowledge where you are
considered an expert and applies to all moves from this skillset. It
should be narrow enough to avoid being generic knowledge, but broad
enough to avoid being useless. A good rule is you should be able to
describe your domain in two to three words.  Examples: \textit{Seattle
  Motercycle Gangs, New York Political Factions, Psychotropic
  Pharmaceuticals}

\paragraph{Critical skills:}
\mastery/, \twitch/

\begin{tcolorbox}[title=Special Move]
  \movename{Its all connected!}: When the \textbf{Professional}
  \textbf{takes an action directly related to their domain specialty},
  they may roll+\mastery/ instead of the normal roll.
\end{tcolorbox}

\movename{Assured Confidence:}

\movename{Rivally:}

\movename{}

\subsubsection{Hacker}

The \textbf{Hacker} is a master at operating and disabling electronic
systems. Part engineer, part scientist, the Hacker generally
understands things well enough to break them and usually well enough to
fix them. In the technology driven \SW/, having a hacker on a team is
almost essential.

\paragraph{Critical skills:}
\mastery/, \flair/

\begin{tcolorbox}[title=Special Move]
  \movename{This is my world:} While the \textbf{Hacker} \textbf{takes
    actions in the Matrix}, they take +1 ongoing.
\end{tcolorbox}

\subsubsection{Infiltrator}

The \textbf{Infiltrator} is a ghost. Excelling at silent ingress and
egress, removing targets, and generally being super annoying in dark
rooms, the Infiltrator can fill many roles in a team of runners. Often
Infiltrators act as scouts, or hidden backup. More social infiltrators
can sometimes be found running high visibility wet work jobs.

\paragraph{Critical skills:}
\twitch/, \mastery/

\begin{tcolorbox}[title=Special Move]
  \movename{Where did he go:} When the \textbf{Infiltrator}
  \textbf{avoids detection from active pursuers}, roll+\twitch/. On a
  10+, the Infiltrator escapes as if by magic and gains +1
  hold for the next combat action against his pursuers. On a 7-9, the
  Infiltrator escapes, but further investigation reveals a evidence of
  his presence.
\end{tcolorbox}


\subsubsection{Driver}

The \textbf{Driver} is the quintessential road warrior. Flying,
rolling, or pushed by a propeller: if it moves the Driver can drive
it. Drivers often are self employed as smugglers or coyotes, but its
not uncommon to see them joining a runner team, especially the more
professional outfits.

\paragraph{Critical skills:}
\mastery/,\twitch/

\begin{tcolorbox}[title=Special Move]
  \movename{Just whack it:} When the \textbf{Driver} \textbf{attempts
    to fix their vehicle while they are currently driving it},
  roll+\mastery/. On a 10+, they do some fast jury rigging and
  everything keeps working. On a 7-9, they manage to keep it running,
  but the machine will need some downtime afterwards.
\end{tcolorbox}

\subsubsection{Mercenary}

The \textbf{Mercenary} is a trained troubleshooter, a provider of
solutions solved with violence, for a price. Hailing from war torn
regions, abandoned slums, or ex-military companies, the Mercenary is
better defined by their code rather than their background. Of course,
the code is typically more money, more money...

\paragraph{Critical skills:}
\oomph/, \mastery/

\begin{tcolorbox}[title=Special Move]
  \movename{I've seen worse:} The first time the \textbf{Mercenary}
  \textbf{receives a severe injury in a fight, they ignore it.} 
\end{tcolorbox}

\subsubsection{Ganger}

The \textbf{Ganger} is the classic tough guy. Life on the streets is
all the training a Ganger requires. Living hard, fighting hard,
surviving anyway possible is the creed of the Ganger. Lacking the
sophistication of other street warriors, the Ganger makes up for it
with sheer brutality and guts.

\paragraph{Critical skills:}
\oomph/, \flair/

\begin{tcolorbox}[title=Special Move]
  \movename{My teacup:} When the \textbf{Ganger}
  \textbf{searches the immediate area for a weapon}, roll+\flair/. On
  a 10+, you find something particularly special and deadly in your
  hands. Take +1 forward for the \movename{Rock \& Roll} move with
  this weapon. On a 7-9, you find something you can use as a weapon,
  but it's nothing really special. In either case, it is only useful
  for one attack.  
\end{tcolorbox}


\subsubsection{Street Doc}

The \textbf{Street Doc} is a medic, sometimes the only one people on
the street can get. Street Docs are used to making the most of what
they can get. Bootlegged medicines, pirated cyber-enhancements, shoddy
equipment are all part and parcel with a Street Doc's
practice. Despite all of this, seeing a Street Doc is often the
difference between life and death for a runner.

\paragraph{Critical skills:}
\mastery/,\twitch/

\begin{tcolorbox}[title=Special Move]
  \movename{Stay with me:} When the \textbf{Street Doc}
  \textbf{applies \movename{First aid} to another, they can try again
    if they fail on their first attempt.}
\end{tcolorbox}

\subsubsection{Street Samurai}

The \textbf{Street Samurai} is a sleek killing machine restrained only
by their personal code. The Street Samurai has dedicated their life
and body to the art of combat, sacrificing even their biological flesh
for a mechanical edge. Some say that the Street Samurai have lost more
than just their bodies. Some say they have lost their souls...

\paragraph{Critical skills:}
\twitch/, \oomph/

\paragraph{Special:}
The Street Samurai possesses a code of honor. Their code describes
both why they fight and what their limits are. Choose one below, or
make up your own:
\begin{enumerate}
  \item Protect the Weak: The weak and helpless need protection from
    those who are strong and would harm them. To disregard a plea for
    help from someone powerless and truly in need would be a disgrace. 
\end{enumerate}


\begin{tcolorbox}[title=Special Move]
  \movename{Steel and Blood:} When the \textbf{Street Samurai} would
  install additional cyberware that would kill them due to essence
  loss, they can survive up to a essence rating of -7. They also start
  play with one \textbf{Major} cyberware item already installed.
\end{tcolorbox}

\subsubsection{Radical}

The \textbf{Radical} is a unrelenting champion of a cause. A Radical
can come from many backgrounds, but somewhere along the way they were
converted from an ordinary person to someone who would do anything for
their goals. Radicals are often referred to as terrorists, but these
are labels assigned by those in power to turn the public away from the
cause of the righteous.

\paragraph{Critical skills:}

\flair/, \mastery/

\begin{tcolorbox}[title=Special Move]
  \movename{My Life for the Cause:} When the \textbf{Radical} performs
  an action critical for their cause and rolls, they can add their
  \essence/ in addition to whatever stat they normally would add. The
  results of the action proceed as normal, except that if they fail,
  they lose a point of \essence/ in addition to any other results. 
\end{tcolorbox}

\subsubsection{Investigator}

The \textbf{Investigator} is a consumate seeker of the truth. Whether
they are a matrix blogger, a private eye, or an ex-cop who just can't
let it go, the Investigator is always poking into things better left
alone. Part people person, part tech-wizard, the Investigator often
employs a wide range of skills and is generally considered a
jack-of-all-trades in the dark streets of the \SW/.

\paragraph{Critical skills:}
\mastery/, \flair/

\begin{tcolorbox}[title=Special Move]
  \movename{Trust My Gut:} When the \textbf{Investigator}
  \movename{CHECKS THE SITUATION}, they can ask the following
  additional questions:
  \begin{dent}
    \tcirc{} Who here is secretly nervous or uncomfortable?

    \tcirc{} What details are out of place?
  \end{dent}
\end{tcolorbox}

\subsection{AWAKENED}

\subsubsection{Mage}

\subsubsection{Shaman}

\subsubsection{Adept}

\subsection{TECHNOMANTIC}

\subsubsection{Botmaster}

\subsubsection{Architect}

\subsubsection{Infovore}


\end{multicols}

\newpage
\invisiblepart{MAGIC}
\section{MAGIC}
\label{magic}
\begin{multicols}{2}
In the \SW/, the magic has returned to the world,
and dormant powers have reawakened. In the \SW/, magic is a natural
force, deeply connected to a person's life force. This is captured by
a character's \essence/, which serves to quantify how \textit{natural}
a person still is. 

\subsection{ESSENCE \& MAGIC}
Three archetypes in the game - the \textbf{Adept}, the \textbf{Mage},
and the \textbf{Shaman} - are magically gifted, or \textbf{Awakened},
which means that they depend on their \essence/ to use their magical
abilities. When a metahuman is Awakened, they have access to a pool of
magical power, called \textbf{Power Points}, equal to their
\essence/+2. If this pool ever drops to zero or lower, they
immediately loose their connection to the magical forces around them,
leaving them powerless. While every magical tradition has their own
interpretation of Power Points, mechanically they are identical across
all magical traditions.

\begin{dent}
  \textbf{The Adept:} adepts turn their magical ability inward to
  improve themselves, sometimes to superhuman levels. An adept
  allocates their Power Points into their abilities, allowing them to
  augment their natural physical and mental abilities. By locking
  these points to an ability, the adapt gains its power as long as
  they remain conscious. 

  \textbf{The Mage:} when a mage casts a spell, the mage's available
  Power Points serve as a limit to how powerful the spell can be. The more
  powerful the spell, the more of a chance something will go wrong -
  potentially injuring the mage in the process. Mages can also
  allocate Power Points to maintain spells over time, though doing so
  limits their ability to cast strong spells simultaneously.

  \textbf{The Shaman:} when a shaman summons a spirit or elemental,
  the shaman's Power Points serve as a limit to the how powerful the
  spirit can be. More powerful spirits can perform more tasks for
  their summoner. However, dealing with powerful spirits is risky -
  the summoning could injure the shaman or even turn an angry spirit
  loss upon them.
\end{dent}

\subsubsection{MAGIC, CYBERWARE, AND INJURY}

A character's essence is a combination of their life force and
humanity, together providing a mystical connection to the astral
planes and allowing the character, whether they are an adept, mage, or
shaman, to perform magic. Anything that interferes with this
connection can reduce a character's essence and consequently reduce
their magical abilities. Receiving a serious injury or implanting
cyberware into one's body are the most common ways a character can
lose essence. As a person loses parts of their body or has them
replaced by machines, their life force is diminished.

While mundanes typically pay little attention to a loss of essence,
this can have serious consequences for a magic user. Adepts can lose
access to magically enhanced abilities, while mages and shamans will
have a much harder time casting spells or summoning spirits. So while
that implanted sword arm might look enticing to a hardcore melee
adept, it is critically important to balance the benefits of the
chrome against the loss of one's magical abilities.

\subsection{MANA BARRIERS}

\textbf{Mana barriers} are protective wards shaped by a mage's
will. Mana barriers block both astral and physical objects, including
spells. They also act as solid barriers towards astral perception and
projection. However, barriers are typically limited in scope, either
placed across narrow choke points, like hallways, or as full domes
several meters in diameter. Larger mana barriers are possible, but
only with special training and when working within a group of
mages. These large barriers are typically only employed to protect
high value corporate or government sites, in conjunction with more
traditional security.

From a game perspective, mana barriers possess a single stat:
\textit{Force}. The barrier's force determines the following qualities:

\begin{dent}

  \textbf{Wound Boxes \& Armor} A mana barrier has its force X 2 as
  wound boxes. It has its force $\div$ 2 as armor.

  \textbf{Duration} A mana barrier will last for a number of days
  equal to its force. However, the barrier's creator can refresh the
  barrier before it expires.

\end{dent}

If you can't wait for a barrier to naturally dissipate, it is possible
to destroy one using brute force. Barriers don't fight back, so a Rock
\& Roll move doesn't apply, but the combination of high armor and
numerous wound boxes makes destroying a mana barrier potentially time
consuming. Alternatively, an awakened character can attempt to pass
through a barrier without destroying it. A word of warning however:
dealing damage to a barrier will immediately alert the mage who
created it. Passing through a barrier is more subtle, but still
carries risks of detection.

\textbf{Jump the Fence:} when you \textbf{pass through a mana barrier
  as an awakened character}, roll+\oomph/. On 10+, you pass through
unnoticed.  On 7-9, you pass through, but the barrier's creator is
alerted to the attempt. On a failure, you fail to pass through and the
barrier's creator is alerted.

\subsection{ASTRAL SPACE}
Much like the Matrix, Astral Space is a sort of alternate universe
adjacent to our own. It is where spells, spirits, magical creatures,
wards and more reside.

When an individual \textbf{perceives} the Astral, they can see the
entities existing in Astral Space. All three arcane archetypes can
astrally perceive. In addition, they can perceive emotional auras of
living beings, as well as background magical nature of the area. When
an individual \textbf{projects} themselves into astral space, they
transfer their consciousness from their physical body to the astral
plane, and can fully interact with other Astral entities and traverse
great distances.

The following effects occur while perceiving or projecting:
\begin{dent}
  \textbf{Perceiving:} while astrally perceiving, take -2 ongoing to
  any moves in the physical world.

  \textbf{Projecting:} you cannot take action in the physical world
  (your body is unconscious and helpless). When you make moves in
  astral space, always roll+\mastery/, instead of the usual stat.
\end{dent}

\subsubsection{ASTRAL QUESTS}
The Astral also serves as a huge deposit of magical information,
though most of the deepest knowledge is hidden in the
metaplanes. Metaplanes are the planes beyond the Astral, the real
sources of all magic. Every metaplane has a \textbf{citadel}, a core
of pure magical energy that can alter the magical world.  Accessing it
can let you destroy a spirit permanently, learn some information such
as the true name of a spirit, or learn an individual’s true
aura. Note, however, an astral quest may only have a single goal.

Astral Quests are also dangerous in that you are stuck in a
metaplane until you either complete your Quest or fail. You
can’t give up, and you can never go back, only forward.

\subsubsection{DOMAINS}
To go on an Astral Quest, you must visit various metalocations known
as domains, similar to Nodes in the Matrix (in fact, mapping these
\textbf{domains} is a useful tool to keep play on track and
engaging). The number and nature of these domains depends on the quest
you are undertaking, but each one presents a challenge the character
must complete in order to move on to the next domain. This could be
fierce combat, a riddle, a puzzle or any variety of things.

Minor quests usually have 3 or 4 domains, while major quests can have
up to 10 or more, all of which lead, ultimately, to the Citadel, where
the quester will find the object or information they seek. Moving from
domain to domain is as simple as willing yourself there once the task
in the current domain is completed.

\subsubsection{THE DWELLER}
The first domain you encounter is always the Domain of the
\textbf{Dweller}, a mystical being who blocks the entrance to the
metaplanes. The Dweller knows everything about the quester, and will
always question the nature your quest before granting passage. The
Dweller is an enigmatic trickster, but if you go on quests often,
you’ll get to know this being quite well.

\end{multicols}


\invisiblepart{THE MATRIX}
\section{THE MATRIX}
\begin{multicols}{2}
  The \textbf{Matrix}, a world-spanning high-fidelity virtual reality
  network, is the domain of the Hacker. A hacker’s job is unique, and
  the conflicts they face usually take place in the gleaming virtual
  world of the matrix. However, this conflict is no less important—or
  deadly—than the one their street sam buddy is going through. With
  security hackers, rogue software, and deadly black IC out there, a
  piece of Matrix code can be every bit as lethal as a 7.62mm bullet.

  \subsection{BUILDING SYSTEMS}
  Including matrix and hacking challenges for the Hacker is one of the
  things the GM should keep in mind as gameplay evolves; a hacker with
  nothing to hack is a sad panda indeed.  One way to do so is
  outlining a \textbf{system}. This is different from hacking devices
  individually or wireless hacking (see ``Wired or Wireless?'').

  \subsection{SCULPTING}
  In the classic movies made at the turn of the century, hacking was
  often depicted as an activity performed in front of a terminal, with
  text flowing down the screen telling the hacker of its arcane
  secrets. Now, of course, all interactions with the Matrix involve
  some level of virtual sculpting. These fabricated environments range
  from augmented reality to ultra-rez UV environments and everything
  in between. The amount of processing power dedicated to a System's
  sculpt is variable, subject to costs and aesthetics preferred by the
  system's users. Interacting with sculpts can be disorienting,
  especially if the administrator has chosen a particularly unusual
  style, but most replicate the environment that corporate admins are
  familiar with: hallways and rooms. Skilled hackers can attempt to
  override the default sculpting to suit their needs, though this
  mostly affects other users - IC isn't generally aware of
  sculpting. Alternatively, hackers can use sculpting to hide subtle messages
  or create digital graffiti.
  
  \subsubsection{NODES}
  A matrix system is made up of a series of Nodes. Each node
  represents a particular secured (or, if the hacker is lucky,
  non-secured) region of the network that can be penetrated and
  controlled. GM’s are encouraged to draw simple maps of connected
  nodes, or create a list of different nodes and brief notes about
  them for to use when the Hacker starts slinging code.

  Different nodes have different purposes, challenges, and payoffs:
  \begin{dent}
    \textbf{Security Node:} this node houses and dispatches intrusion
    countermeasures.

    \textbf{Datastore:} this node contains data, and may have
    encryption or even a data bomb failsafe to render data useless if
    intrusions are detected

    \textbf{Credentials Node:} contains user credentials or grants
    permissions which can help the hacker avoid detection or access
    secured areas

    \textbf{Process Node:} runs a process on the network, slowing down
    the activity of other system software

    \textbf{Control Node:} this is a node to which multiple device
    nodes are connected; it serves as a master controller for the
    attached devices.

    \textbf{Device Nodes:} a single device connected to the network.
    Devices range from cameras to security drones to maglocks; almost
    everything is connected. Devices are frequent targets for
    intrusion attempts. Most simple devices have minimal privilege on
    the network, but that is often enough.
  \end{dent}

  \subsubsection{ARMORED NODES}
  Many matrix nodes have only one layer of security: once you hack in,
  the node is yours. However, more secure systems have additional
  defenses. These nodes, called \textbf{armored nodes}, are both
  hardened against intrusion and contain intrusion countermeasures.

  Mechanically, Armored Nodes have both Wounds (how many is up to the
  GM), and embedded Intrusion Countermeasures (see \textbf{Threats},
  page 46) which fight back against intruding hackers.

  It’s possible to have nodes that have only Wounds, but no defensive
  IC. In this case, the node is effectively defenseless, and the
  Hacker simply deals damage to the node.

  \subsubsection{ALERT LEVELS}
  A System has four \textbf{Alert Levels}, representing both how aware
  the system is that it has been compromised, and how actively it will
  attempt to locate, identify, and stop the intrusion.
  \begin{dent}
    \textbf{Green:} the system is unaware that it has been
    compromised.

    \textbf{Yellow:} the system has detected a possible
    intrusion. Routine notifications are dispatched, but no direct
    countermeasures are taken.

    \textbf{Orange:} the system is aware of an intrusion and is
    actively trying to locate, disable, and trace the
    hacker. Nonlethal countermeasures are approved.

    \textbf{Red:} the system is aware of a serious intrusion. Lethal
    countermeasures are approved.
  \end{dent}
  \subsection{HACKING}
  When a Hacker encounters a node or device, he or she must first hack
  into the node using the \textbf{Sling Code} move. Once inside, the
  Hacker can transit through the node, or take advantage of any
  actions or bonuses the node provides (unless it is an Armored Node
  or is protected by IC, in which case it will not be nearly so
  trivial to use the node’s functions).

  \subsubsection{WIRED OR WIRELESS?}
  Although node maps evoke a particular style of Matrix runs, namely
  using the ``wired connection'' paradigm of older editions of
  Shadowrun, you can easily use wireless hacking, or a mix of the
  two. For wireless hacking, all devices are a node. They may contain
  multiple nodes inside, as well, or be standalone., but they’re also
  usually accessible via a wireless connection (or if not, accessible
  via connection to another device that is).

  Devices such as firearms, cyberware, and other items carried by
  individuals are also fair game for hacking. In such case, assume
  them to be armored nodes. You’ll need to indicate how many wounds
  the device has, and how much damage it can do to a hacker, if any.

  A sample device might be:
  \begin{dent}
    \textbf{Commlink} [6 wounds, 1d4 stun dmg]
  \end{dent}
  An armored node or device can only deliver its damage in matrix
  combat; the commlink above didn’t suddenly become a taser.

\end{multicols}


\invisiblepart{LEGWORK \& DOWNTIME}
\section{LEGWORK \& DOWNTIME}
\begin{multicols}{2}
  While most of the interesting parts of \SW/ happen in the middle of
  a shadowrun, most shadowrunning teams, if they have the opportunity,
  will take time to do some research on their run and the people
  associated with it, and gather necessary equipment, before they
  stick their head in the alligator’s mouth.

  Likewise, after a run, shadowrunners might take some time to go to
  ground, heal up their wounds, spend some of their ill-gotten nuyen,
  and generally maintain a low profile while the aftermath of their
  latest job blows over. The cycle of activity in \SW/, then, can
  usually be described as

  \begin{adjustwidth*}{1cm}{2cm} {
      \orbitronfont \textbf{LEGWORK > THE RUN > DOWNTIME}}
  \end{adjustwidth*}

  (Please note this is descriptive, not prescriptive: your games don’t
  have to resemble this in the least, if you don’t want them to!)

  In \SW/, the research portion of the run is called legwork, and the
  time after a run—and before the work starts on the next run—is
  generally referred to as downtime. While legwork has some optional
  rules to structure it, downtime is much less rules-oriented, and is
  handled much like downtime in other games: narratively, as a chance
  for players to talk about what’s going on without rolling dice, and
  to set the stage leading up to the next run.

  \subsection{LEGWORK}
  Shadowrunners do not (always) charge headlong into danger, guns and
  spells blazing. In fact, those who do generally only do it once.

  Instead, a savvy runner does legwork before a run, getting as much
  information as possible within the time they have. This section
  outlines how to play through the legwork process, letting the
  players create details that give them advantages, while giving you a
  few wrenches to throw in the works in return. The methodology below
  was originally described in the ``Dirty Dungeons'' segment of John
  Wick’s Play Dirty gaming advice videos, and is an option for lending
  more mechanical weight behind the legwork that goes into a
  shadowrun. There are 3 basic steps:

  \textbf{1.  PROVIDE THE ANCHOR}

  Give the players a premise they have to deal with. This can be
  anything from ``extract scientist X from the corporate facility at
  Y'' to ``a Humanis Policlub group is preparing a terrorist attack
  and we want it stopped.''

  \textbf{2.  START THE LEGWORK}

  During the actual legwork, characters search for information, speak
  to contacts and other NPCs, purchase or otherwise acquire equipment,
  get assets into position, and discover details that will help flesh
  out the mission. Details discovered in this fashion are awarded
  through moves taken during the legwork phase.

  When a detail is uncovered, the player establishes the nature of the
  detail: what it is and why it’s valuable.  Details found this way
  can be anything from floor plans to passkeys to security procedures,
  whatever a player might think is useful. Problematic details (too
  much of an advantage, one-shot-mission-solvers, mission-evaders, and
  the like), however, should be discussed immediately, and replaced
  with something else that’s more reasonable and believable.

  When a character discovers or establishes a detail, add a point to
  the Mission Pool (it’s probably best to use poker chips or pennies
  or something to track Mission Points). Continue gathering details
  and building the Mission Pool until the players are satisfied or any
  game-imposed time limits run out.

  \textbf{3.  GATHER COMPLICATION POINTS}

  While the players are prepping their info, they are also building up
  a number of Complication Points you’ll have available. Every Legwork
  move specifies how much time is spent, and for every day of ``game
  world time'' spent on Legwork, you add one point to your
  Complication Pool—the longer they spend getting ready, the more
  likely it is that the details might change a bit.

  \subsubsection{MISSION POINTS}
  At any point during the run, a player may draw one point from the
  Mission Pool and spend it to boost their next move.  Players must
  use the Mission Point on their next move (they can’t hold onto it
  until later - once drawn from the pool, it’s use it or lose
  it). Additionally, once a Mission Point is used, it is removed from
  the mission pool. Mission Pools do not refresh (the only way to get
  another mission pool is, of course, to get another mission).

  \subsubsection{COMPLICATION POINTS}
  When the characters gather information for a run, it is important
  for the GM to remember that all of the information they gather is
  true. Detail gathering is an opportunity for players to declare what
  they know to be true about a mission, and not an opportunity for the
  GM to feed them erroneous information. On the other hand, if
  everything always went exactly to plan, it wouldn’t be a shadowrun!

  To introduce these little wrinkles, the GM may spend complication
  points to throw a small wrench into the works, by declaring a change
  or inaccuracy in one of the details discovered during mission prep.

  \begin{dent}
    \textbf{Example:}\textit{ during mission prep, the characters
      discovered that security patrols on the 6th floor of their
      target building happen in two shifts, but there is a 5 minute
      gap in coverage they could exploit. As they approach the entry
      point from an adjacent building, the GM elects to spend a
      complication point to introduce a twist - a new guard is being
      trained, and he and his supervisor happen to be right near the
      window where the team was going to make their entry.}
  \end{dent}

  Complication Points are an opportunity to use a GM Move to alter a
  detail the characters discovered legwork (in the example above, the
  GM has revealed an unwelcome truth about the security patrols), with
  the added concession that you have spent a limited resource in order
  to do so.

  In that vein, a caution to the GM: use care when introducing
  complications. Remember that much of the detail provided by the
  players will be plenty exciting - and get plenty complicated -
  simply by playing to see what happens, Because success with a cost
  is a constant companion in \SW/, the characters’ own actions are
  going to complicate things, so you should let the details they have
  help them out.

  Finally, remember that Complication Points can only be spent to
  alter a mission detail, and they must be spent if you wish to do
  so. Spend carefully, and only when it will make things more
  interesting – never just to screw the characters.  Like Mission
  Points, Complication Points, once spent, are gone.

  \subsubsection{LEGWORK MOVES}
  This section’s title is a bit of a misnomer. \SW/ doesn’t specify a
  fixed set of approved ``legwork moves,'' nor any ``legwork only''
  moves. Nevertheless, several moves (both secondary moves as well as
  some archetype moves) involve preparation, information gathering,
  training, and similar activities, Moves that feature prominently in
  preparation and legwork include:
  \begin{dent}
    \tcirc{} Citation Needed

    \tcirc{} Pull Strings

    \tcirc{} Hit the Books

    \tcirc{} Go Shopping

    \tcirc{} Build a Legend (Face)

    \tcirc{} I Know A Guy (Face)

    \tcirc{} Contracts Available (Mercenary)

    \tcirc{} Field Trial (Mercenary)

    \tcirc{} Gun Cage (Ex-cop)

    \tcirc{} Pharmacy is Open (Street Doc)
  \end{dent}

  \subsubsection{OTHER ACTIVITIES}
  Other activities that can be done during legwork (or during
  downtime) include writing programs (page 66), spellcrafting (page
  67), working on gear (page 60), or bonding with new spirits (page
  69). The rules for each of those activities specify the time the
  character must spend to successfully complete the activity.

  \subsection{DOWNTIME}
  Downtime is, in effect, ``free time'' for the characters. This is
  the time spent dealing with their lives outside of shadowrunning:
  recovering from injury, paying their rent, working out, getting
  drunk, or spending time with family (believe it or not, not every
  shadowrunner is a hyper-paranoid loner drifter with nothing to lose).

  Time spent in downtime is handled in a narrative fashion.  If
  something done during downtime specifies an amount of time required,
  that time is spent, but that serves mainly to indicate the overall
  passage of time in the world, rather than racing toward an oncoming
  deadline.

  On the other hand, the world does live and breathe. If an event is
  coming, it will happen when it happens, and will not necessarily
  wait for the characters’ schedules to line up. (On the upside,
  unless the event is ``bombs fall, everybody dies,'' then world
  events that happen during downtime should only serve to make the
  runners’ lives more interesting).

  \subsubsection{DOWNTIME MOVES}
  Although downtime is largely a move-free time, moves can occur
  then. One move that must occur during downtime is the Advance move
  (page 5), where characters can to reflect on their experience and
  improve themselves.

\end{multicols}

\newpage


\invisiblepart{EQUIPMENT}
\section{EQUIPMENT}
\begin{multicols}{2}
  In this section you’ll find example equipment (weapons, cyberdecks,
  vehicles, etc.) available in the \SW/. This isn’t an exhaustive list
  of what’s available; rather, they’re just samples of some classic
  items to help you get playing quickly.  Also, although it’s not
  exactly the correct word, in this document the term equipment refers
  to pretty much any resource the character has (so spells and spirits
  are also considered ``equipment'' for the sake of simplicity).

  \SW/ also offers rules to create customized and personalized
  versions of the following:
  \begin{dent}

    \tcirc{} weapons

    \tcirc{} cyberdecks

    \tcirc{} vehicles and drones

    \tcirc{} spells

    \tcirc{} programs

    \tcirc{} spirits
  \end{dent}

  If you want to create and customize your own stuff, check out the
  \textbf{Creating Gear} section starting on page 60. That section
  explains \SW/’s ``template-based'' customization system.

  Of course, you should also feel free to simply make up new equipment
  or add in things you think are missing—just because there isn’t a
  set of creation rules for something doesn’t mean it doesn’t exist!

  \subsection{EQUIPMENT TAGS}
  Equipment—like many items in \SW/—is described in terms of
  \textbf{tags}, which are short keywords that indicate various
  capabilities or qualities. Certain tags apply to multiple kinds of
  equipment (such as obvious, supply, or armor). Tags that only apply
  to specific kinds of equipment are described in the listing of that
  kind of item. The following tags apply to multiple types of
  equipment.
  \begin{dent}

    \textit{2-hand:} this item must be used with both hands

    \textit{armor +n:} grants a +n bonus to existing armor

    \textit{armor n:} grants n Armor (for vehicles or drones,
    indicates armor rating, and is abbreviated arm)

    \textit{arcane:} can only be used by magical archetypes

    \textit{area:} affects multiple targets

    \textit{+bonus:} grants a bonus to a particular move; e.g. +1 to
    Stay Frosty

    \textit{conceal:} this weapon or item is easily hidden and will
    not be spotted by enemies

    \textit{damage n:} the amount of damage a weapon or other item
    deals. Abbreviated dmg

    \textit{heal n:} restores n wounds

    \textit{ignores armor:} bypasses the target’s armor

    \textit{loud:} noisy and audible to anyone with functioning
    hearing; for weapons, it means the weapon cannot be suppressed

    \textit{messy:} deals damage in a particularly gruesome way

    \textit{obvious:} cannot be concealed, or is immediately visible
    to any observer

    \textit{range:} the range(s) at which the weapon or other attack
    is effective. Ranges are \textbf{touch (t)}, \textbf{close (c)},
    \textbf{short (s)}, \textbf{medium (m)}, and \textbf{long (l)}.

    \textit{shock:} the weapon deals electrical shock

    \textit{special (description):} if the effect of the item requires
    explanation, use this tag.

    \textit{stun:} this weapon or attack deals Stun damage only

    \textit{subtle:} not easily noticed (as opposed to conceal, which
    means it is unnoticeable)

    \textit{supply n:} the amount of supplies or uses you can get out
    of an item. Each use of the item consumes 1 supply (unless
    otherwise stated).
  \end{dent}


  \subsection{WEAPONS}

  \textbf{WEAPON TAGS}
  \begin{dent}

    \textit{2-hand:} this item must be used with both hands

    \textit{AP n:} this weapon ignores n points of armor; note that
    each point of AP requires the payment of the 25\% customization
    premium

    \textit{auto:} this weapon can fire in full auto mode. Abbreviated
    fa.

    \textit{burst:} this weapon fires in burst mode. Mark off 1
    additional Ammo to deal +1 damage. Abbreviated bf.

    \textit{chem:} this weapon delivers a chemical agent of some kind
    to the target; depending on the delivery mechanism, armor may be
    ignored.

    \textit{forceful:} when this weapon deals damage, it also deals 1
    stun

    \textit{fuzed:} this weapon cannot be used at less than the
    shortest range increment listed

    \textit{reload:} after using this weapon, it takes more than a
    moment to reload it.

    \textit{semiauto:} this weapon fires one shot every time the
    trigger is pulled. Abbreviated sa.

    \textit{stabilized:} this weapon cannot be fired except from a
    bipod, tripod, or supported position.

    \textit{suppressed:} this weapon makes little to no noise when
    fired

    \textit{thrown:} this item can be throw. If thrown, the range is
    short.

    \textit{vented:} the weapon has recoil venting, granting +1 to
    Suppression Fire
  \end{dent}

  \subsubsection{MELEE WEAPONS}
  \textbf{Staff} [\textit{range c, stun, 1d6+2 damage, 100¥}]

  \textbf{Combat Axe} [\textit{range c, messy, 1d6+2 dmg, 1,250¥}]

  \textbf{Combat Knife} [\textit{range c, 2d4b dmg, 1 AP, 300¥}]

  \textbf{Compound Bow} [\textit{range s/m/l, 2-hand, dmg 1d6+1, ammo
    1, 500¥}]

  \textbf{Crossbow} [\textit{range c/s/m, 2-hand, dmg 1d6, suppressed,
    reload, 400¥}]

  \textbf{Fists/Feet} [\textit{range c, 1d6 dmg, stun}]

  \textbf{Katana} [\textit{range c, 2d6b damage, 1,000¥}]

  \textbf{Spiked Glove} [\textit{range c, 1d4 wound + 1 stun, 50¥}]

  \textbf{Stun Baton} [\textit{range c, 1d4 dmg, stun, shock, ignores
    armor, 750¥}]

  \textbf{Tomahawk} [\textit{range c, messy, thrown, 1d6 damage,
    200¥}]


  \subsubsection{HOLD-OUT PISTOLS}
  \textbf{Streetline Special} [\textit{range s, sa, dmg 2d4b, ammo 3,
    con- ceal, 250¥}]

  \textbf{ Fichetti Needler} [\textit{range s, dmg 2d4b, AP 1,
    conceal, ammo 3, 400¥}]

  \textbf{Walther PP} [\textit{range s, sa/bf, dmg 1d4+1, ammo 1,
    conceal, 325¥}]


  \subsubsection{LIGHT PISTOLS}
  \textbf{Colt L36} [\textit{range s/m, sa, dmg 1d6, conceal, ammo 3,
    500¥}]

  \textbf{Beretta 101T} [\textit{range s/m, sa/bf, dmg 1d6, subtle,
    ammo 2, 450¥}]

  \textbf{Ares Lightfire 70} [\textit{range s, sa, dmg 1d6, conceal,
    ammo 3, 350¥}]


  \subsubsection{HEAVY PISTOLS}
  \textbf{Ares Predator} [\textit{range s/m, dmg 1d8+1, sa, AP 2, 3
    ammo, 675¥}]

  \textbf{Colt Manhunter} [\textit{range s/m, dmg 1d8, sa/bf, AP 1, 3
    ammo, 560¥}]

  \textbf{Ruger Super Warhawk} [\textit{range s/m, dmg 1d10, sa, AP 1,
    2 ammo, loud, 560¥}]

  \textbf{Browning Max Power} [\textit{range s/m, dmg 2d8b, sa, 3
    ammo, 675¥}]


  \subsubsection{SUBMACHINE GUNS}
  \textbf{HK227} [\textit{range s/m, sa/bf, dmg 1d8, suppressed, ammo
    4, 900¥}]

  \textbf{AK-97K} [\textit{range s/m, sa/fa, dmg 1d8, AP 1, ammo 3,
    1,000¥}]

  \textbf{Ingram Smartgun} [\textit{range s/m, bf/fa, dmg 1d6+1, AP 1,
    ammo 3, 950¥}]


  \subsubsection{ASSAULT RIFLES}
  \textbf{AK-97} [\textit{range s/m/l, 2-hand, sa/fa, dmg 1d10, AP 1,
    obvious, ammo 3, 800¥}]

  \textbf{Ares Alpha} [\textit{range s/m/l, 2-hand, sa/bf/fa, dmg
    2d8b, AP 1, obvious, ammo 4, 1,150¥}]

  \textbf{Colt M22A2} [\textit{range s/m/l, 2-hand, sa/bf, dmg 1d10,
    AP 1, obvious, ammo 3, 850¥}]

  \textbf{FN-HAR} [\textit{range s/m/l, sa/bf, dmg 2d8b, AP 2,
    obvious, loud, 2-hand, 1,050¥}]


  \subsubsection{SHOTGUNS}
  \textbf{Remington 990} [\textit{range s/m, sa, dmg 1d10+1, obvious,
    loud, forceful, ammo 2, 750¥}]

  \textbf{Enfield AS7} [\textit{range s/m, 2-hand, sa/bf, dmg 1d10,
    obvious, loud, forceful, ammo 3, 900¥}]


  \subsubsection{SNIPER RIFLES}
  \textbf{Ranger Arms} [\textit{range l, sa, 2-hand, dmg 1d10+1, AP 3,
    ammo 3, 1,150¥}]

  \textbf{Walther WA2100} [\textit{range 1, sa, 2-hand, dmg 1d12, AP
    2, ammo 4, 1,100¥}]


  \subsubsection{HEAVY WEAPONS}
  \textbf{Ingram Valiant LMG} [\textit{range m/l, 2-hand, loud, fa,
    stabilize, obvious, loud, messy, dmg 1d12, ammo 4, AP 1, 2,000¥}]

  \textbf{Stoner M202 HMG} [\textit{range m/l, 2-hand, loud, bf/fa,
    stabilize, obvious, loud, messy, dmg 2d10b, ammo 3, AP 2, 2,500¥}]

  \subsubsection{SPECIAL WEAPONS}
  \textbf{Narcoject Rifle} [\textit{range s/m, 1d8+1 stun, suppressed,
    chem, slow, 700¥}]

  \textbf{Taser} [\textit{range s, 1d8 stun, shock, slow, 500¥}]


  \subsubsection{GRENADES}
  \textbf{EMP} [\textit{thrown, area, shock, disables electronis,
    95¥}]

  \textbf{Flash} [\textit{thrown, area, stun, dmg 2d4, +1 to Rock \&
    Roll/Stay Frosty, 125¥}]

  \textbf{Frag} [\textit{thrown, area, forceful, dmg 2d6b, 100¥}]

  \textbf{Incendiary} [\textit{thrown, area, 2d6b dmg, burn, 75¥}]

  \textbf{Smoke} [\textit{thrown, area, +1 to Stay Frosty, 40¥}]

  \textbf{Stun} [\textit{thrown, area, dmg 2d6b, stun, 100¥}]

  \subsection{ARMOR}
  Armor provides protection against incoming attack, reducing the
  damage dealt by the armor value. Armor of the same type (e.g
  inherent) does not stack. Armor of differing types can stack. Armor
  has the following unique tags: inherent: this armor is either
  implanted, or occurs naturally.  Cyberware armor is inherent armor.
  worn: this armor is worn on the body mystic: this armor is magical
  in nature

  \subsubsection{SAMPLE ARMOR}
  \textbf{Lined Coat} [\textit{armor 2, obvious, worn, 600¥}]

  \textbf{Ballistic Vest} [\textit{armor 2, obvious, worn, 750¥}]

  \textbf{Armorweave Professional Wear} [\textit{armor 1, subtle,
    worn, 1,500¥}]

  \textbf{Chameleon Suit} [\textit{armor 1, conceal, worn, +1 to Stay
    Frosty, 6,000¥}]

  \textbf{Leather Armor} [\textit{armor 1, subtle, worn, 250¥}]

  \textbf{Armor Charm} [\textit{armor +1, mystic, conceal, 400¥}]

  \textbf{Light Armor Jacket} [\textit{armor 2, subtle, 850¥}]

  \textbf{Combat Armor} [\textit{3 armor, obvious, 2,500¥}]

  \textbf{Form-fitting Armor} [\textit{armor 1, conceal, 550¥}]

  \textbf{Riot Shield} [\textit{armor 2, occupies one hand, 700¥}]


  \subsection{CYBERDECKS}
  Cyberdecks are the essential tool of the hacker. They are the
  Hacker’s connection to the Matrix. Cyberdecks have the following
  special tags:

  \begin{dent}
    \textit{CPU:} the raw processing power of the deck

    \textit{Mask:} the stealthiness of a cyberdeck

    \textit{Hardening:} the deck’s resistance to damage; this acts as
    armor protecting the hacker

    \textit{Storage:} the deck’s capacity for loaded programs
  \end{dent}

  \subsubsection{EXAMPLE DECKS}
  \textbf{Allegiance Alpha} [\textit{CPU 1, mask 1, hardening 1,
    storage 8, 25,000}]

  \textbf{Fuchi Cyber-4} [\textit{CPU 1, mask 2, hardening 1, storage
    8, 50,000¥}]

  \textbf{Fuchi Cyber-7} [\textit{CPU 3, mask 1, hardening 1, storage
    8, 75,000¥}]

  \textbf{Fairlight Excalibur} [\textit{CPU 3, mask 2, hardening 1,
    100,000¥}]


  \subsection{PROGRAMS}
  Programs run on a cyberdeck. Hackers don’t need programs do to their
  job — they can sling code well enough to bend the matrix to their will
  on the fly — but a program can improve their chances or offer special
  tricks to help the hacker.

  Programs have the following special tags:
  \begin{dent}

    \textit{routines:} the different routines that make up the
    program.  See \textbf{Writing Programs}, page 66, for details
    about routines.

    \textit{size n:} the amount of space a program takes up in the
    cyberdeck’s storage.
  \end{dent}

  Armor or damage tags on programs only work when in the Matrix.

  \textbf{RUNNING PROGRAMS } When a program is loaded into the storage
  on a cyberdeck, it is assumed to be running. If the hacker has to
  change programs, they may do so at any time; however, if it would
  be despite risk of some sort (for instance, while in combat with
  IC), then they must \textit{Stay Frosty}.

  \subsubsection{AGENTS}
  Hackers can compile separate programs into pseudo-sentient matrix
  entities called \textbf{agents}. See the Programs section (page 66)
  for more information.

  \subsubsection{SAMPLE PROGRAMS}
  \textbf{Armor} [\textit{armor +2 (matrix only), routines (armor x
    2), size 4, 500¥}]

  \textbf{Black Hammer} [\textit{dmg 1d6, relocate hostile programs,
    routines (armor, bounce), size 4, 500¥}]

  \textbf{Stealth} [\textit{mask +2, routines(stealth x 2,
    interference), size 6, 750¥}]

  \textbf{Lockpick} [\textit{mask +1, +1 to hack Data nodes,
    routines(stealth, decrypt), size 4, 500¥}]

  \textbf{Assassin} [\textit{mask +1, dmg 2d6b, armor +1,
    routines(stealth, armor, attack x 2), size 8, 1,000¥}]

  \textbf{Ghost} [\textit{mask +2, routines (stealth x 2), size 6,
    500¥}]

  \textbf{Tarpit} [\textit{slow alarms and relocate hostile programs,
    routines (bounce x 2, interference), size 6, 500¥}]

  \textbf{Bloodhound} [\textit{+2 Check the Situation in the matrix,
    +1 to hack data nodes, routines (analyze x 2, decrypt), size 6,
    750¥}]

  \textbf{Medic} [\textit{heal 2 matrix damage, routines(repair x 2),
    size 4, 500¥}]

  \textbf{Codebreaker} [\textit{+2 to decrypt data nodes,
    routines(decrypt x 2), size 4, 500¥}]

  \subsection{VEHICLES}
  Vehicles have the following special tags:

  \begin{dent}
    \textit{Power (pwr):} the vehicle’s horsepower, speed, and
    acceleration.

    \textit{Armor (arm):} the vehicle or drone’s armor rating.

    \textit{Frame (frm):} the vehicle’s or drone’s resilience. This is
    the equivalent of a vehicle’s wounds. Remember that small arms
    deal half damage to vehicles (see Dealing Damage, page 9).

    \textit{Sensors (ssr):} the quality of the vehicle’s sensors (used
    when Checking the Situation while driving or piloting the vehicle)

    \textit{Seats n:} the number of people who can normally occupy the
    vehicle, including the driver or pilot

    \textit{Fuel:} fuel or battery capacity
  \end{dent}

  \subsubsection{BIKES}
  \textbf{Dodge Scoot} [\textit{seats 1, pwr 1, arm 0, frm 4, ssr 0,
    fuel 4, 1,800¥}]

  \textbf{Yamaha Rapier} [\textit{seats 1, pwr 2, arm 0, ssr 1, frm 4,
    fuel 4, 9,500¥}]

  \textbf{Harley Scorpion} [\textit{seats 2, pwr 2, arm 1, frm 7, ssr
    1, fuel 2, 17,500¥}]


  \subsubsection{CARS \& TRUCKS}
  \textbf{C-N Jackrabbit} [\textit{seats 3, pwr 1, frm 6, ssr 0, arm
    0, fuel 3, 10,000¥}]

  \textbf{Ford Americar} [\textit{seats 4, pwr 1, frm 8, ssr 1, arm 0,
    fuel 3, 16,000¥}]

  \textbf{Eurocar Westwind} [\textit{seats 6, pwr 3, frm 9, arm 1, ssr
    1, fuel 3, 200,000¥}]

  \textbf{GMC Bulldog} [\textit{seats 8, pwr 2, frm 9, arm 1, ssr 1, 3
    fuel, seats 8, 45,000¥}]

  \textbf{Ares Roadmaster} [\textit{seats 6, 3 pwr, 11 frm, 2 armor, 2
    fuel, 52,000¥}]


  \subsection{DRONES}
  Drones have most of the same qualities as vehicles, although they
  lack the seats tag, and replace it with the following:

  \begin{dent}
    \textit{Tactical:} the quality of the drone’s tactical expert
    system, which comes into play when the drone is in autonomous
    mode. Abbreviated tac.
  \end{dent}

  Armed drones also use the \textit{damage} tag, indicating the damage
  of their built-in weapon systems.

  \subsubsection{GROUND DRONES}
  \textbf{Aztechnology Crawler} [\textit{pwr 1, frm 5, ssr 2, arm 0,
    tac 0, fuel 3, 4,000¥}]

  \textbf{GM-Nissan Doberman} [\textit{pwr 1, frm 7, arm 1, ssr 1, dmg
    1d6, tac 1, fuel 3, 5,000¥}]

  \textbf{Steel Lynx} [\textit{pwr 1, frm 9, arm 2, ssr 1, tac 2, dmg
    2d6b, fuel 2, 9,500¥}]


  \subsubsection{AIRBORNE DRONES}
  \textbf{Lockheed Optic-X} [\textit{pwr 1, ssr 2, arm 0, frm 2, tac
    1, fuel 2, 12,500¥}]

  \textbf{MCT Roto-Drone} [\textit{pwr 2, frm 5, arm 0, ssr 1, dmg
    2d4b, tac 1, fuel 2, 15,750¥}]

  \textbf{CD Dalmatian} [\textit{pwr 1, frm 8, arm 1, ssr 0, tac 2,
    dmg 1d8, fuel 3, 22,000¥}]


  \subsection{CYBERWARE}
  The cyberware items in the Archetype’s starting packages are shown
  here with all their tags. Cyberware has the following special tags:
  \begin{dent}

    \textit{add-ons:} this is installed in an existing piece of
    cyberware, instead of independently. The item takes up capacity
    equal to its essence cost.

    \textit{always on:} the implant remains on all the time. If adding
    this tag to an item that modifies a move, multiply the cost of the
    implant by 2.

    \textit{capacity n:} the cyberware item has capacity for n add-on
    items.

    \textit{device:} this implant is a device of some sort (usually a
    weapon or computing tool) that does not offer sensory
    modification.

    \textit{link (device):} this cyberware must be connected to the
    proper kind of device to be effective

    \textit{loaner:} this implant was given to you by an organization
    lots of money, and they expect you to repay them somehow.

    \textit{resist (hazard):} the augmentation protects against
    particular environmental hazards such as toxins or electrocution

    \textit{sealed:} a sealed implant requires at least an hour and
    the proper tools to reload or refill.

    \textit{sota:} state of the art; sota cyberware has a lower
    essence cost than equivalent standard cyberware

    \textit{toggle:} this item is toggled on and off (that is, once
    activated, it stays on).

    \textit{used:} this implant started its life in someone else’s
    body.  The first time you fail a move related to the implant or
    are in a situation where the added capability of the device comes
    into play, roll 1d6. On a 3 or better, you’re fine. On a 2, the
    implant simply fails gracefully. On a 1, the implant goes haywire:

    \begin{dent}
      \tcirc{} If the implant modifies a move, that move is glitched
      until you get it fixed or shut down

      \tcirc{} If the implant provides a capability, that ability
      suddenly poses a big problem

      \tcirc{} You can shut down a haywire implant by spending a point
      of Edge.
    \end{dent}
  \end{dent}

  \subsubsection{ACTIVATING CYBERWARE}
  To gain the benefits of any of the following items, you must spend a
  point of Edge to activate the implant. Implants that offer no
  mechanical benefit (related to moves or defenses), such as
  cyberlimbs, are always on—you don’t have to spend edge to use them.

  \subsubsection{HEADWARE}
  
  \paragraph{EYES}

  \textbf{Cybereyes} [\textit{always on, capacity 2, essence 1}]

  \textbf{Thermographic Enhancement} [\textit{ability(thermographic
    vision), essence 1}]

  \textbf{Vision Magnification} [\textit{always on, ability(long
    distance vision), essence 1}]

  \textbf{Low-light enhancement} [\textit{ability(low-light vision),
    essence 1}]

  \textbf{Camera} [\textit{ability(record video or images), essence
    1}]

  \paragraph{EARS}
  \textbf{Cyberears} [\textit{always on, capacity 2, essence 1}]

  \textbf{Damper} [\textit{ability(resist:sound), essence 1}]

  \textbf{Noise Filter} [\textit{ability(enhanced hearing), essence
    1}]

  \textbf{Recorder} [\textit{ability(record audio or video), essence
    1}]

  \textbf{Ultrasound System} [\textit{ability(perceive ultrasound),
    essence 1}]

  \paragraph{OTHER}
  \textbf{Cranial Cushion} [\textit{always on, armor +1 vs. stun,
    essence 1}]

  \textbf{Tactical Computer} [\textit{modifies(Check the Situation:
    use Combat instead of Awareness), essence 1}]

  \textbf{Synaptic Hardening} [\textit{armor +1(matrix only), essence
    1}]

  \textbf{Voice Modulator} [\textit{ability(alter voice), essence 1}]


  \subsubsection{BODYWARE}
  \textbf{Active Camouflage} [\textit{special(if you remain
    motionless, enemies cannot see you), essence 2}]

  \textbf{AutoDoc} [\textit{special(gain 1 extra wound box), toggle,
    essence 3}]

  \textbf{Bone Lacing} [\textit{always on, special(deal lethal damage
    when unarmed, gain 1 additional wound box), essence 2}]

  \textbf{Boosted Reflexes} [\textit{modifies(Stay Frosty: hold 1),
    special(incompatible with wired reflexes, cannot be upgraded),
    essence 2}]

  \textbf{Cyberarm/Cyberleg} [\textit{always on, device, obvious,
    capacity 2, essence 3}]

  \textbf{Dermal Plating 1} [\textit{armor +1, inherent, always on,
    essence 2}]

  \textbf{Dermal Plating 2} [\textit{armor +2, inherent, always on,
    essence 3}]

  \textbf{FeatherTouch} [\textit{ability(enhanced sense of touch),
    essence 1}]

  \textbf{Gyrostabilizer} [\textit{modifies(Suppression Fire: hold 1),
    essence 2}]

  \textbf{Hand Razors} [\textit{range c, dmg 1d4 dmg, essence 1,
    toggle}]

  \textbf{Light Cybergun} [\textit{range c/s, 1d6 dmg, toggle, sealed,
    essence 2}]

  \textbf{ReadiMed System} [\textit{modifies(First Aid: hold 1),
    supply 2, sealed, special(can also modify relevant Street Doc
    moves), essence 2}]

  \textbf{Skillsoft} [\textit{link(skillwires), special(required for
    skillwires to function; specify area of knowledge when
    purchasing)}]

  \textbf{Skillwires 1} [\textit{modifies(Drop Science: hold 1),
    link(skillsoft), essence 2}]

  \textbf{Skillwires 2} [\textit{modifies(Drop Science: hold 2),
    link(skillsoft), essence 3}]

  \textbf{Shocktrodes} [\textit{range c, dmg 1d4 stun, essence 1}]

  \textbf{Smartlink} [\textit{move(Rock \& Roll: add +1 damage on 10+,
    on 7-9, don’t mark off ammo), ranged, essence 1}]

  \textbf{Spurs} [\textit{range c, dmg 1d6, essence 2, toggle}]

  \textbf{Wired Reflexes 1} [\textit{modifies(Stay Frosty: hold 1),
    essence 2}]

  \textbf{Wired Reflexes 2} [\textit{modifies(Stay Frosty: hold 2),
    essence 3}]

  \subsection{OTHER EQUIPMENT}
  \subsubsection{DRUGS}
  Costs listed below are per dose (one dose equals 1 Supply)
  \textbf{Bliss} [\textit{take +1 to Gut Check, lasts 2 hours, 15¥}]

  \textbf{Cram} [\textit{take +1 to Stay Frosty, lasts 3 hours, 10¥}]

  \textbf{Deepweed:} [\textit{user can perceive Astrally, lasts 1
    hour, 400¥}]

  \textbf{Jazz} [\textit{take +2 to Stay Frosty, lasts 30 minutes,
    75¥}]

  \textbf{Kamikaze} [\textit{take +1 to Rock \& Roll and Gut Check,
    lasts 1 hour, 100¥}]

  \textbf{Long Haul} [\textit{you can go without sleep for four days
    with no consequence, 50¥}]

  \textbf{Nitro} [\textit{take +2 to Rock \& Roll and +1 to Gut Check,
    lasts 30 minutes, 75¥}]

  \textbf{Novacoke} [\textit{take +1 to Push Someone and Check the
    Situation, lasts 2 hours, 10¥}]

  \textbf{Psyche} [\textit{take +1 to Drop Science, lasts 3 hours,
    200¥}]

  \textbf{Zen} [\textit{take +1 to Stay Frosty, lasts 30 minutes, 5¥}]

  \textbf{BTLs} [\textit{allow you to experience almost anything
    virtually, lasts 30 minutes to 3 hours, 20-100¥}]


  \subsubsection{MISCELLANEOUS}
  \textbf{Medic Patch} [\textit{supply 1, heal 2, 500¥}]

  \textbf{Stimulant Patch} [\textit{supply 1, take +2 to next move,
    take 1 stun afterwards, 175¥}]

  \textbf{Antidote Patch} [\textit{halts poison damage, 200¥}]

  \textbf{Trauma Patch} [\textit{supply 1, +1 to First Aid Move,
    300¥}]

  \textbf{Quik-Hax Kit} [\textit{supply 4, bypasses low-grade security
    locks/ electronic devices, 350¥}]

  \textbf{Spy Kit} [\textit{supply 4, +1 to Citation Needed or Check
    the Situation (assuming bugs haven’t been found), 4000¥}]

  \textbf{Countersurveillance Kit} [\textit{supply 4, +1 to Check the
    Situation to search for bugs, 3000¥}]

  \textbf{Infiltrator’s Kit} [\textit{supply 4, +1 to Stay Frosty to
    infiltrate or avoid detection, 1,000¥}]

  \subsection{MAGICAL SUPPLIES}
  \subsubsection{FOCI}
  A focus is a mundane item that has been imbued with an astral
  construct. When used by someone to which it is attuned, a focus
  helps them channel astral power greatly enhances their abilities.

  \textbf{ATTUNING}

  Before a focus can be used, the user must \textbf{attune} themselves
  to it. To do so, they must invest at least one point of essence into
  the focus. Essence committed in this fashion remains spent until the
  user de-attunes themselves from the focus, or the focus is
  destroyed, at which point the essence is recovered.

  A mage, adept, or shaman can only be attuned to a number of foci
  equal to their Craft rating.

  \textbf{TYPES OF FOCI}

  \begin{dent}
    \textbf{Spell Focus:} a spell focus enhances the casting of a
    specific spell. When attuned, the mage using the spell focus has
    hold equal to the Essence spent attuning the focus. Spend this hold
    toward casting that specific spell.

    \textbf{Spirit Focus:} a spirit focus enhances the summoning of a
    specific type of spirit. When attuned, the shaman has hold equal
    to the essence invested in the focus toward summoning that
    specific spirit type.

    \textbf{Weapon Focus:} weapon foci are primarily used by adepts.
    When attuned to a weapon focus, the adept using it has hold equal
    to the invested Essence to spend on the Rock \& Roll move or on
    dealing damage.
  \end{dent}

  \subsubsection{FETISHES}
  Fetishes are essentially one-shot magical supplies—small mundane
  objects imbued with structure and energy of a spell or summon a
  spirit, needing only to be triggered by the mage or shaman.

  \textbf{INVESTING}

  To create a fetish, the mage or shaman decides what spell or spirit
  to place into the fetish, and then \textbf{invests} the fetish with
  power, spending the Essence required for the spell, or the essence
  they wish to provide to the spirit. Essence invested in a fetish in
  this manner remains spent until the fetish is used, at which point
  it immediately returns.

  \textbf{ACTIVATING A FETISH}

  Normally, to cast a spell or summon a spirit, the mage or shaman
  must make the \textit{Cast a Spell} or \textit{Conjure} moves. With
  a fetish, this is no longer the case: instead, they can simply
  declare that they’re using it (making any other moves that the
  fiction would dictate of course, for instance, \textit{Stay
    Frosty}).  Once triggered, the stored spell or spirit is
  immediately cast or conjured. The fetish is good for a single use,
  after which it crumbles to dust.

\end{multicols}

\section{SPELLS}
\begin{multicols}{2}
Like other equipment, spells (although they’re not exactly 
  ``equipment'') are described in terms of tags. Spells have the 
      following special tags: 

\begin{dent}

      \textit{Force:} the minimum Force required to 
      cast the spell. When determining the effects of the spell, use the \textbf{Effective Force}, or \textbf{EF}, value which is the \textbf{(Force Cast - Minimum Force) + 1}.

      \textit{Effect:} describes the actual result of a successful casting of 
      the spell. 
\end{dent}

\textbf{RANGE TAGS }


\begin{dent}

\textit{Touch:} the spellcaster must touch the target to cast the
spell.

\textit{LOS:} the spellcaster must be within line of sight of the target. Technological vision enhancements (aside from old fashioned optics) do not count for line of sight.

\textit{Linked:} the spellcaster must possess an object of high significance to the target, or a fresh (under 24 hours old) bodily sample. With an appropriate link, the spell has a range of \textbf{EF} kilometers.

\end{dent}

\textbf{TARGET TAGS}

\begin{dent}

\textit{Self:} the spell only affects the caster

\textit{Metahuman:} the spell only affects metahumans

\textit{Creature:} the spell affects any living creature

\textit{Spirit:} the spell affects only spirit beings

\textit{Object:} the spell affects inanimate objects

\textit{Device:} the spell affects technological devices
\end{dent}

\textbf{DURATION TAGS}

\begin{dent}

\textit{Instant:} the spell occurs very quickly.

\textit{Short:} the spell lasts long enough for the target to take one
move, more or less.

\textit{Triggered:} this spell is triggered by an outside event (for
instance, taking damage)

\textit{Sustained:} the spell remains in effect for a period determined by the caster. Each sustained spell in effect inflicts a stacking -1 to future spellcasting moves to account for the split concentration of the caster.
\end{dent}
\end{multicols}

\subsection{COMBAT SPELLS}

\rowcolors{2}{white}{lightgray}
\begin{tabular}{>{\bfseries}m{.1\linewidth}m{.34\linewidth}>{\bfseries\centering}m{.11\linewidth}m{.35\linewidth}}
Spell& \textbf{Description}&Minimum Force&\textbf{Tags}\\\midrule
Mana Bolt & deals 1d4 damage (bypassing armor) to creatures or spirits at LOS & 2 & LOS, target(creatures, spirits), duration:instant,
dmg 1d4, ignores armor, force 2\\
Fire bolt& deals 1d6 damage and fire effects to creatures at LOS range.& 2 & LOS, target(creatures), instant, dmg 1d6, fire, force 2\\
Taser Hands& deals 1d6+EF damage and shock effects to
creatures at touch range& 2 & touch, target(creatures), object, instant, dmg 1d6+EF, shock,
force 2\\
Acid Stream& deals 1d6 damage and acid effects to targets
and objects at LOS range & 2 &
Tags: LOS, acid, target(creature), object, instant, dmg
1d6, force 2\\
Fireball& deals 1d6+EF damage and fire effects to
all creatures and objects in an area within short
range.& 3 & LOS, fire, area, target(creature), instant, dmg
1d6+EF, obvious, force 3\\
Manaball& deals 1d6 damage (bypassing armor) to creatures
and spirits within the target area&4 & LOS, area, target(creatures, spirits), instant, dmg
1d6, ignores armor, force 4 \\
Knockout& deals 1d6 stun (bypassing armor) to creatures
in touch range& 1 & touch, target(creatures), instant, dmg 1d6 stun, ignores armor,
essence 1\\
\bottomrule
\end{tabular}

\subsection{DETECTION SPELLS}
\rowcolors{2}{white}{lightgray}
\begin{tabular}{>{\bfseries}m{.1\linewidth}m{.34\linewidth}>{\bfseries\centering}m{.11\linewidth}m{.35\linewidth}}
Spell& \textbf{Description}&Essence Cost&\textbf{Tags}\\\midrule
Analyze Device& take +1 to your next move involving the
device being analyzed, or learn what the device does&1& touch, analysis, target(device), duration:short, effect(take +1 to a move involving the device),
force 1\\
Clairvoyance& when you Check the Situation, you can ask
questions about a location you cannot see within the range
of the spell& 2 & LOS, perception, target(self), duration:short, area, effect(Check the Situation in
a remote area), force 2\\
Combat Sense& while you sustain this spell, you cannot be
surprised, and take +1 to your first Rock \& Roll or Stay Frosty
move when combat starts& 2(S) & touch,
perception, target(self), duration:sustained, effect(you cannot be surprised and take +1 to your first Rock \& Roll or Stay
Frosty), subtle, force 2\\
Mind Probe& when you touch the target, you can hold 1 toward Manipulate or Make ‘Em Sweat& 2(S) & touch, telepathy, target(metahumans), duration:sustained, effect(hold 1 toward Negotiate or
Push Someone), force 2\\
Detect Life& when you look for living creatures in an area,
take +2& 2& LOS, perception,
target(self), duration:short, effect(take +2 to look for living
creatures with Check the Situation), force 2\\
\bottomrule
\end{tabular}

\subsection{HEALTH SPELLS}
\rowcolors{2}{white}{lightgray}
\begin{tabular}{>{\bfseries}m{.1\linewidth}m{.34\linewidth}>{\bfseries\centering}m{.11\linewidth}m{.35\linewidth}}
Spell& \textbf{Description}&Essence Cost&\textbf{Tags}\\\midrule
Antidote& when you touch the target, you halt poison or
other toxin effects in the target& 1 & touch, cure, self, target(metahumans), effect(halts poisons and
other toxins), force 1\\
Heal& when you touch the target, heal a number of wounds
equal to the EF&1& touch, heal, self, target(metahumans), effect(heal EF wound boxes), exhausting, force 1\\
Increase Attribute& when you touch the target, choose 1
stat. Moves using that stat take +1 while the spell is sustained. &2& touch, enhance, target(self), metahuman, duration:sustained, effect(choose 1 stat; moves using that stat
take +1 while the spell is sustained), exhausting, force 2\\
\bottomrule
\end{tabular}

\subsection{ILLUSION SPELLS}
\rowcolors{2}{white}{lightgray}
\begin{tabular}{>{\bfseries}m{.1\linewidth}m{.34\linewidth}>{\bfseries\centering}m{.11\linewidth}m{.35\linewidth}}
Spell& \textbf{Description}&Essence Cost&\textbf{Tags}\\\midrule
Chaotic World& when you cast this spell, you can hold 1 to
spend on your or your teammate’s moves&2& LOS,
distraction, target(creatures),area, duration:sustained, effect(hold
1 toward your or your teammate’s moves in combat), force 2\\
Group Invisibility& while you sustain this spell, you conceal
a number of creatures equal to the EF from being
seen by creatures or metahumans& 3(S) & LOS, area, concealment, target(metahumans),
duration:sustained, effect(you cannot be seen by creatures or
metahumans), force 3(S)\\
Silence& while you sustain this spell, all sound is silenced
in the area you specify&3(S)& LOS, area, concealment, target(creatures),
duration:sustained, effect(all sound is silenced in the area),
force 3(S)\\
Stink& while you sustain this spell, all creatures in the affected
area have to either leave the area or use air filters or take
1 stun&3(S)&LOS, area, distraction, target(creatures),
duration:short, effect(enemies must flee, use respirators or
filters, or take 1 damage), force 3\\
\bottomrule
\end{tabular}

\subsection{MANIPULATION SPELLS}
\rowcolors{2}{white}{lightgray}
\begin{tabular}{>{\bfseries}m{.1\linewidth}m{.34\linewidth}>{\bfseries\centering}m{.11\linewidth}m{.35\linewidth}}
Spell& \textbf{Description}&Essence Cost&\textbf{Tags}\\\midrule
Mana \mbox{Barrier}& while you sustain this spell, you create a barrier that blocks living creatures and spirits& 3(S) & LOS, protection, target(creatures,spirits), duration:sustained, effect(create a barrier that blocks living creatures
and spirits), force 3\\
Light& while you sustain this spell, an area you specify is illuminated by bright light& 3(S)& LOS, area,
energy, duration:sustained, effect(generates bright illumination in an area; large areas cost more essence),
force 3\\
Shadow& while you sustain this spell, an area you specify
is cloaked in arcane darkness&3(S) &LOS, area, energy, duration:sustained effect(generates arcane
darkness in an area), force 3\\
Fling& when you cast this spell on a target you are touching,
you hurl the target out of melee range& 1& touch, telekinesis, target(creatures), duration:instant, effect(hurl target out of melee range; target takes 1
stun), force 1\\
\bottomrule
\end{tabular}


\section{SPIRITS}
\begin{multicols}{2}
Spirits are the companions and tools of the Shaman, who
summons them from the astral plane to perform services for
him. Spirits have the following special tags:

\begin{dent}
\textit{aspect:} the spirit takes on the appearance of their domain,
and is invisible in their domain unless it chooses to be
seen. Elementals automatically gain this tag, otherwise it
requires 1 spirit point.

\textit{desert:} a spirit of the forbidding landscape of
the deserts

\textit{earth:} a spirit who dwells in the earth, caves, or landscape;
earth spirits are widespread

\textit{elemental:} these spirits represent the basic four elements,
air, earth, fire, and water, and can be summoned
anywhere.

\textit{engulf:} the spirit may enclose a target in the ubstance of its
domain, typically (but not always) dealing damage.

\textit{enthrall:} use this stat for the Enthrall move

\textit{forest:} a spirit of the forests, woods, or similar
areas

\textit{generous:} the spirit will perform one extra move; adding
this tag costs 1 spirit point.

\textit{guard:} use this stat for the Guard move

\textit{harm:} use this stat for the Harm move

\textit{insubstantial:} damage dealt and taken is halved

\textit{mentor:} use this stat for the Mentor move

\textit{mountain:} a spirit that dwell in foothills, crags, ridges, and
other mountainous terrain

\textit{natural:} natural spirits are spirits associated
with particular domains (such as ``city spirits'' or
``mountain spirits'').

\textit{plains:} a spirit of the open plains, grasslands, fields, and
farms

\textit{robust:} the spirit is particularly resistant to damage; all
damage rolls against it are [w]. Adding this tag costs 1
spirit point.

\textit{search:} use this stat for the Search move

\textit{sky:} a spirit of the open sky

\textit{storm:} a spirit of storms and harsh weather

\textit{swamps:} a spirits of the depths of the swamp, bayou, or
wetlands

\textit{urban:} a spirit dwelling in urban or developed
lands, especially cities

\textit{water:} a spirit of lake, river, or ocean 

\textit{weakness (specify):} the spirit has a weakness to a particular
material or element which ignores insubstantiality, armor,
and robustness. Adding this tag allows the free addition of
another tag.

\textit{wild:} this spirit has an extra spirit point, but the shaman
must take -1 when he or she conjures it
\end{dent}

\subsection{SPIRIT MOVES}
Spirits are independent entities, and have thier own moves.
Their moves correspond to the harm, search, guard, enthrall,
and mentor tags.

\textbf{HARM:} when a spirit attacks someone or something,
roll+Harm. On 10+, the spirit deals its damage. On 7-9, the
spirit deals damage, but also takes damage.

\textbf{SEARCH:} when the spirit attempts to locate individuals or
items within its domain, roll+Search. On 10+, the spirit locates the item and can tell the Shaman where it is. On 7-9,
the spirit can tell the shaman whether the item or person is
within its domain, but not it’s specific location. Note: the GM
and player should determine the search range for
elementals.

\textbf{GUARD:} when a spirit stands in defense of its domain or
inhabitants thereof, roll+Guard. On 10+, the spirit prevents
damage or hostile effects from occurring. On 7-9, the spirit
halves damage or the potency of a hostile effect.

\textbf{ENTHRALL:} when a spirit attempts to control someone’s
actions or thoughts, roll+Enthrall. If the target
is a:
\begin{dent}
\textbf{An NPC:} On a 10+, the spirit issues two instructions
that the NPC must follow, or take 3 damage. On 7-9,
the spirit may issue one instruction.

\textbf{A PC:} On a 10+, both of the following apply. On 7-9,
only 1 applies:
\begin{dent}
\tcirc{} If the character complies, they mark XP

\tcirc{} If the character refuses, they must Stay Frosty
\end{dent}
\end{dent}

\textbf{MENTOR:} when a spirit imparts knowledge or truth, roll+Mentor. On 10+, the GM provides, in secrete, a useful or
interesting piece of information to the target. On 7-9, the GM
provides an interesting piece of information.

\subsection{EXAMPLE SPIRITS}
There are 5 general spirit natures: Watchers simply observe
and report. Teachers seek to instruct and guide others, but
are reluctant to do harm. Protectors seek to defend their domain and its inhabitants, while Destroyers seek battle, blood,
and vengeance. Finally, Seducer spirits desire control and
devotion.

\subsubsection{ELEMENTALS}
\textbf{Fire Elemental} [\textit{destroyer, aspect, harm 2, search -1, guard
1, enthrall 1, mentor 0, dmg 1d10, armor 2, wounds
9}]

\textbf{Water Elemental} [\textit{seducer, aspect, harm -1, search 2, guard
0, enthrall 3, mentor 1, dmg 1d4, armor 1, wounds
8}]

\textbf{Air Elemental} [\textit{teacher, aspect, harm -2, search 2, guard 0,
enthrall 1, mentor 2, dmg 1d4, armor 2, wounds 7}]

\textbf{Earth Elemental} [\textit{protector, aspect, harm 1, search 2, guard
2, enthrall -1, mentor 0, dmg 1d8, armor 1, wounds
10}]

\subsubsection{NATURAL SPIRITS}
\textbf{Forest Protector} [\textit{natural, forest, harm 1, search 1, guard 2,
enthrall -1, mentor 0, dmg 1d8, aspect, armor 1,
wounds 8}]

\textbf{Forest Watcher} [\textit{natural, forest, search 3, guard 0, enthrall
1, mentor 1, aspect, armor 1, wounds 6, special:may not
Harm}]

\textbf{Sky Watcher} [\textit{natural, aspect, search 3, guard 0, enthrall 0,
mentor 2, armor 1, wounds 6, special:may not
Harm}]

\textbf{Urban Destroyer} [\textit{natural, harm 2, search 0, guard 1, enthrall
1, mentor -1, dmg 1d10, armor 2, wounds 9}]

\textbf{Urban Seducer} [\textit{natural, seducer, harm 0, search 2, guard 0,
enthrall 2, mentor 1, dmg 1d4, armor 1, wounds 7}]

\textbf{Mountain Teacher} [\textit{natural, aspect, harm 0, search 0, guard
2, enthrall 0, mentor 2, dmg 1d4, armor 1, wounds
8}]

\textbf{Swamp Destroyer} [\textit{natural, aspect, harm 2, search 2, guard
0, enthrall 0, mentor -1, dmg 1d10, armor 2,
wounds 9}]


\end{multicols}

\newpage

\invisiblepart{GM GUIDELINES}
\section{GAMEMASTER GUIDELINES}
\begin{multicols}{2}
As mentioned in the introduction to this game, I’m assuming some familiarity with Dungeon World on the part of the
reader. Dungeon World provides a list of important rules for
the GM to follow. Here they are (modified for proper cyberpunk-ness, of course):


\subsection{ALWAYS SAY}
\textbf{What the rules demand:} when a move is triggered, yours
or the players, say what the rules tell you to say. Embellish
and expand, but start from the rules.

\textbf{What the adventure demands:} you know things the players don’t, and you know them ahead of time. If the players
haven’t done anything to change them, stick with
‘em.

\textbf{What honesty demands:} always be honest. If the rules tell
you to give out information, do it. No lies, no half-truths. Be
generous, even. And once it’s set in stone, no going back
on it. Also, if the players achieve something, give it to them
fully.

\textbf{What the principles demand:} use your principles and
agenda as a filter or an inspiration. If you get caught short,
review them to make sure you are abiding by them.
\subsection{YOUR AGENDA}
\textbf{Make the world fantastic:} barf forth cyberpunk! Scenes,
smells, sounds - the glittering height of an arcology, the
stench of a slum hellhole, the scream of turbofans as a GEV
heads toward you, the rrrrrrrrip of a minigun tearing through
your cover - it’s your job!

\textbf{Fill the characters’ lives with adventure:} make the world
they live in exciting, dangerous, full, and epic.

\textbf{Play to find out what happens:} NO. PLOTS. Ideas, yes.
Fronts, sure. But do not come to the table with a story already written in your head, because for sure, the players will
not go where you expect.
\subsection{YOUR PRINCIPLES}
\textbf{Draw Maps, Leave Blanks:} make use of maps, but don’t fill
it all in. Leave holes for imagination.

\textbf{Address the characters, not the players:} never talk to the
players in the fiction. They don’t live in the
\SW/.

\textbf{Embrace the exotic and fantastic:} the world is a crazy
mesh of man, magic, and machine. Make it breathe.

\textbf{Make a move that follows:} when you make a move, you
are participating in the fiction. The move should follow from
the fiction logically.

\textbf{Never speak the name of your move:} moves aren’t things
in \SW/. Moves are shorthand for you. Never say the
name of your move.

\textbf{Give every creature life:} monsters and creatures exist and
are real. Give them smells, sounds, personality.
Name every person: everyone has a name. Make sure you
give it to them!

\textbf{Ask questions, and use the answers:} the easiest question
is ``What do you do?'' Whenever you make a move, end
with ``What do you do?'' And don’t forget to take opportunities to keep the focus moving from character to
character.

\textbf{Be a fan of the characters:} you are not here to beat them;
this is not a contest. You should cheer their successes,
lament their failures, and mourn their passing.

\textbf{Think with the Front Sight:} nothing in the world you create
for the characters is sacred. Every time you put something
or someone onscreen, think about how destroying them
might affect the story.

\textbf{Begin and end with the fiction:} to do it, do it. Everything
stems from, and leads back to, the conversation you’re having. Transition from fiction to rules and back to
fiction.

\textbf{Think offscreen, too:} make your move elsewhere, and
show the effects to the characters later.

\subsubsection{GM MOVES}
The GM has moves of his or her own to use. Although they’re
given formal names, they’re really just the same things GMs
have always done. For example, ``revealing an unwelcome
fact'' isn’t an esoteric trick to learn—it could be as simple as
saying ``that datastore you just cracked? Yeah, it was really a
honeypot, and security hackers are closing in.''

These moves, just like the players’ moves, stem from, and
return to, the fiction of the game. Let them flow!

\textbf{BASIC MOVES}
\begin{dent}
\setlength{\parskip}{.1em}\itshape
Use an NPC, creature, danger, or location move

Reveal an unwelcome fact

Show signs of danger

Deal damage

Use up their resources

Turn their move back on them

Separate them

Give an opportunity to showcase an archetype

Show a downside to their archetype, race, or
equipment

Offer an opportunity - with or without cost

Put someone in a spot

Tell them the requirements and consequences, and
ask
\end{dent}

\textbf{LOCATION MOVES}
\begin{dent}
\setlength{\parskip}{.1em}
\itshape
Change the environment

Point to a looming threat

Introduce a new faction

Use a threat from an existing faction

Make them backtrack

Present riches at a price

Present a challenge to one character
\end{dent}

\end{multicols}


\section{THREATS}
\begin{multicols}{2}
\textbf{Threats} is the general term for the opposition - creatures, 
other runners, security guards, and so forth — that a team of 
  runners might encounter in their adventures. Threats come in 
  many shapes and sizes, and only a few examples are given 
  here, but you can use these examples to expand on the list of 
  threats, and invent your own (you can even use the Monster 
  Creator at http://codex.dungeon-world.com/). 

\subsection{THREATS AND DICE}
If you’re the GM, you should be aware that unlike many
games, \textbf{you never roll dice to make moves} (though you will
roll dice for Threat damage from time to time).
Threats have moves, both the GM moves listed earlier, and
sometimes their own special moves, but you won’t see any
``roll+Stat'' instructions here. Threat Moves happen in response to, and flow from the fiction. If something is done by
a player character that would lead to a Threat move, then it
happens. If the player didn’t fail their move, then it’s likely
that what you’ll do is a \textbf{soft move}: show them some danger
coming, make something happen that will trigger a move on
their part, and so forth.
On the other hand, if the player gives you a golden opportunity, usually by completely failing a move, then you can
make a \textbf{hard move}. An easy example of this is in the case of
doing damage. If a PC Rocks \& Rolls with a threat, and fails
(rolls a 6 or less), then in return, that Threat deals its damage
to the player right away. That’s the default outcome for failing
a Rock \& Roll move.
Keep in mind, however, that you only have to make \textbf{as hard
a move \textit{as you like}}. It doesn’t always have to be the ultimate
sanction — sometimes, you might make a soft move to increase the tension of a situation. You don’t have to deal that
damage, if making a different move would be more fun!

\subsubsection{THREAT DAMAGE}
Threats, in general, deal the damage indicated in their entry
whenever they deal their damage. However, sometimes multiple threats mob a single player character and inflict damage
on the PC. In such cases, they do not all deal their damage.
Instead, deal damage for the most dangerous threat, and add
+1 damage for each additional threat involved in
the attack.

\textbf{Example:} \textit{Valentin is facing down a ghoul and four goblins, who all assaulted him more or less simultaneously. He
attempted to dodge away, but failed. Instead of dealing
2d6b for the ghoul, and then rolling 2d4b four more times
(once for each goblin), you would roll 2d6b for the ghoul,
and add an additional 4 damage (+1 for each
goblin).}

\subsubsection{OPTIONAL: INFLICTING CHRONIC INJURY}

If it suits the group, you can allow a threat to inflict chronic injuries (see page 10) if that threat’s damage pushes a character into the bleeding out stage. If so, choose an appropriate
chronic injury from the list. For example, if a ghoul manages
to take a character to the bleeding out stage with a bite, you
can inform the character that unless they stabilize, they will
take the Faded chronic injury, and reduce their Essence by 1.

\subsubsection{THREAT WOUNDS}
Threats make no distinction between stun and wounds for
threats. If you deal stun to a threat, unless it is listed as immune to stun, simply mark the damage on the wound track.

\subsection{THREAT TRAITS}
The traits that follow are primarily intended to help the GM
describe creatures, figure out what a creature might do, set
scenes, and enhance the story. For example, when using a
threat with the Camouflage trait, the GM might leverage that
trait to describe how the threat materializes out of nowhere,
having been hidden against a wall or some other innocuous
place until the PC’s were in just the right spot.

\textit{Amphibious:} threat is at home in water
and on land

\textit{Arcane:} threat is Awakened

\textit{Aspect:} threat shows traits of its domain
or environment

\textit{Bloodthirsty:} the threat will continue to
attack incapacitated opponents

\textit{Camouflage:} threat is difficult to detect and can blend in
with its environment

\textit{Cyber:} this threat is enhanced with cyberware, which increases its performance in some fashion

\textit{Deathwish:} the threat lacks any sense of self-preservation;
this can manifest in relentless attacks, or simple stupidity,
depending on the threat

\textit{Dual Natured:} threat is visible and active both in Astral
Space and in the physical world. Abbreviated dn.

\textit{Fast:} the threat is exceptionally quick

\textit{Fear:} the threat inspires fear or causes
a fear effect

\textit{Fearless:} the threat will often continue
fighting to the death

\textit{Group:} usually seen in groups of 3-6
individuals

\textit{Hoarder:} the threat collects...something. Sometimes good
things, sometimes horrifying things.

\textit{Horde:} threat is typically found in large
groups

\textit{Huge:} colossal, several times larger than
a human

\textit{Immune (type):} threat is immune to a particular type of
damage, for example immune (stun)

\textit{Infected:} threat carries a disease that can be contracted by
the characters

\textit{Insubstantial:} threat takes half damage

\textit{Intelligent:} threat is smart enough to think and plan; most
metahuman threats are intelligent

\textit{Large:} much larger than a human

\textit{Machine:} threat is mechanical in origin

\textit{Medium:} roughly human size

\textit{Movement:} threat has a special movement
mode

\textit{Night Vision:} threat can see in dark environments without
trouble

\textit{Organized:} threat has an organizational structure, and may
have additional allies upon which to call

\textit{Paranormal:} threat is of paranormal origins

\textit{Poison:} threat poison its targets; victims take 1 damage
each time they make a move, until they receive treatment
of some sort)

\textit{Program:} threat is a Matrix program (such as IC)

\textit{Range:} these are the same as the ranges in the equipment
section

\textit{Small:} smaller than a human

\textit{Spirit:} attacking this threat uses the Battle the Arcane move

\textit{Solitary:} usually seen alone

\textit{Stealthy:} threat is naturally difficult to detect

\textit{Summoned:} this is a spirit being, and can be banished

\textit{Tiny:} much smaller than a human

\subsubsection{TAG NOTES}
All paracritters are assumed to have the paranormal tag.

All Intrusion Countermeasures are assumed to have the
fearless and program tags.

Creatures may or may not fight to the death. Many
metahumans will not, since most of them still have
some sense of self preservation. The fearless tag indicates a much greater likelihood of fighting to the death
even without a reason.

\end{multicols}

\section{PARACRITTERS}
\begin{multicols}{2}
All paracritters have the paranormal tag.


\critterspec
{AFANC}
{amphibious, camouflage, group, large}
{Bite (2d6b dmg, c), tail whip (1d6+1, reach)}
{10 Wounds / 2 Armor}
{The Afanc is an awakened crocodile, typically found in Wales
and Eastern Europe. They exist in family groups of 3-6 individuals, and are highly territorial. They have an exceptional
ability to detect nearby prey.}
{to eat}
{\tcirc{} Detect nearby prey

\tcirc{} Death roll
}

\critterspec
{BARGHEST}
{fast, medium, fear, group}
{Bite (1d6+2 dmg, c), howl (2d8b stun, area, c/s/m)}
{6 Wounds / 1 Armor}
{The barghest is an awakened canine found in North America,
Europe, and Asia. A massive mastiff-like creature, the barghest is best known for its unearthly, paralyzing howl which it
uses to freeze its prey in its tracks.}
{to hunt}
{
\tcirc{} stalk the prey
}

\critterspec
{COCKATRICE}
{dual-natured, hoarder, small, solitary}
{Paralytic tail (2d6b+2 stun, c)}
{4 Wounds / 0 Armor}
{The cockatrice resembles an overgrown, semi-reptilian chicken. It is known best for the paralysis a touch of its long tail can induce in a metahuman. It’s also known for its tendency to collect small items -- jewelry, etc. }
{protect its territory.}
{\tcirc{} turn flesh to stone

\tcirc{} collect the shinies}

\critterspec
{BLACK ANNIS}
{fast, fearless, medium, night vision}
{Slam (1d6 dmg, forceful, c), bite (1d8 damage, c)}
{6 Wounds / 1 Armor}
{The Black Annis is an awakened baboon, highly territorial and vicious. Studies also indicate that the Black Annis is capable of creating an overwhelming sense of depression in metahumans, though this has not been confirmed. }
{to dominate.}
{\tcirc{} tear intruders apart

\tcirc{} show a threat display}


\critterspec
{DEATHRATTLE}
{camouflage, medium, poison, solitary}
{Bite (2d6b, poison, c), spit venom (1d8, s)}
{5 wounds / 0 armor}
{The deathrattle is a large awakened rattlesnake, found across North America. The deathrattle has a potent toxin which operates on both a physical and astral basis. It is very difficult to cure, requiring the attentions of both medical professionals and magical expertise.}
{to eat.}
{\tcirc{} strike from hiding

\tcirc{} shake the rattle}

\critterspec
{DEVIL RAT	}
{disease, horde, small}
{Gnaw (1d6 damage, messy, 1AP, c)}
{4 wounds / 0 armor}
{The devil rat is a giant, hairless, loathsome creature found in sewers and sprawls around the world. Devil rats are somewhat dangerous alone, but when they swarm, they can cause catastrophic damage. Stories about mass disappearances in some of the worst slums are sometimes attributed to devil rat swarms.}
{to devour.}
{\tcirc{} swarm of teeth

\tcirc{} avoid the light}

\critterspec
{DRAGON	}
{arcane, dual-nature, huge, hoarder, intelligent}
{Bite (2d10b dmg, 4AP, c), fire breath (2d6 dmg, s/m)}
{30 wounds, 6 to 8 armor}
{Never cut a deal with a dragon. Extremely intelligent and powerful, these creatures have become heads of megacorps, and one was even the President of the UCAS before he was assassinated. They come in many varieties, including western, eastern, feathered and leviathian. Their ultimate purpose is unknown, but whatever it is, they seem to be doing it well.}
{to be the ultimate.}
{\tcirc{} Get rid of opposition

\tcirc{} Scheme from the shadows

\tcirc{} Unleash its wrath}

\critterspec
{GREATER WOLVERINE	}
{bloodthirsty, fearless, large, solitary}
{Bite (1d8 dmg, messy, c), claw (1d6+1 dmg, messy, c)}
{10 wounds / 2 armor}
{The greater wolverine is a massive engine of destruction, with a mean streak a mile wide. }
{to kill.}
{\tcirc{} Abuse the dead

\tcirc{} Eat to excess}

\end{multicols}

\newpage

\section{METAHUMANS}
\begin{multicols}{2}

\critterspec
{CORPORATE SECURITY	 }
{group, intelligent, medium }
{Sidearm (1d8 dmg, 1AP, s/m), stun baton (1d6 stun, c) }
{8 Wounds / 0 Armor }
{This is the run of the mill corporate security guard. Dangerous in groups, and corporations generally have a near-infinite supply.}
{to guard their station. }
{\tcirc{} Call for backup 

\tcirc{} Trigger the alarm }

\critterspec
{ELITE SECURITY	}
{group, cyber, intelligent, medium }
{SMG (2d6b dmg s/m), Hand-to-Hand (1d6+1 dmg, c) }
{8 Wounds / 2 Armor }
{Although not every facility has an elite security contingent protecting it, when you start running the bigger corporations, you may run into these guys. With better training and better gear than your typical security guard, Elite Security is called in when the regular security grunts run into more than they can handle. }
{secure the facility. }
{\tcirc{} Neutralize targets 

\tcirc{} Strike from ambush }

\critterspec
{BEAT COP }
{medium, intelligent, solitary }
{Sidearm (1d8 dmg, 1AP, s/m), baton (1d6 dmg c) }
{8 Wounds / 1 Armor }
{Even in the seemingly lawless 2050s, there are still people out there who serve in the thin blue line, walking a beat and enforcing the law. Whether a member of Knight Errant, Pinkerton, or Lone Star, the beat cop is the most commonly seen law enforcement officer on the streets. }
{to protect and serve. }
{\tcirc{} make an arrest 

\tcirc{} call backup}

\critterspec
{LONE STAR HTR		}
{cyber, medium, intelligent, group}
{Assault Rifle (2d8b dmg, 2AP, s/m/l)}
{8 Wounds / 3 Armor}
{Hostage situations, major crimes, killing sprees, you name it — when a serious crime goes down, the High Threat Response teams are called in. Highly trained, well-equipped, and thoroughly professional, tangling with HTR is no joke.}
{terminate the threat.}
{\tcirc{} Breach, bang and clear

\tcirc{} Take the shot}

\critterspec
{BLOOD MAGE	}
{arcane, medium, solitary}
{Blood bolt (1d8 dmg s/m), death touch (2d4b, ignores armor, c)}
{8 Wounds / 1 Armor}
{Blood magic — the use of blood (usually not your own) to fuel magical spells and rituals — is illegal almost everywhere in the \SW/. However, that doesn’t stop people from using it.}
{to gather power.}
{  \tcirc{} Inflict bleeding wounds}

\critterspec
{CYBERZOMBIE	}
{dual-natured, medium, intelligent, cyber}
{Arm Cannon (2d6b dmg, 2AP, s/m/l), arm blade (1d6 dmg, c)}
{15 wounds / 3 armor}
{The cyberzombie is an unfortunate soul, a cyborg who has pushed himself too far with cybernetics and died. A cybermancer has managed to reconnect his soul to the body, and now the creature lives a tortured life. }
{to pass on.}
{\tcirc{} Destroy for the creator

\tcirc{} Find a way to end the suffering}


\critterspec
{COMBAT MAGE	}
{arcane, cautious, medium, solitary}
{Manabolt (1d6+1 dmg, s/m), flamethrower (1d6+1 dmg, burn, s/m), confusion (targets take -2, s)}
{8 Wounds / 2 Armor}
{The Awakened are statistically rare in the \SW/, but shadowrunners tend to deal with them considerably more frequently than your average wageslave. One of the more feared foes on the battleground is the Combat Mage, a mage who has devoted his abilities to deadly combat magic.} 
{to see who’s best.}
{\tcirc{} Display their power

\tcirc{} Burn everything}

\critterspec
{SECURITY HACKER	}
{cyber, intelligent, medium, solitary}
{Black hammer (2d6b dmg, c), blackout (1d6+1 dmg, stun c), slow (-1 forward, c)}
{8 Wounds / 2 Armor (matrix only)}
{Any corporation worth its salt employs security hackers to protect its precious data. A corporate hacker is often equipped with excellent gear and has the benefit of being able to navigate a corporate grid easily, since they belong there.}
{to track ‘em and smack ‘em.}
{\tcirc{} Initiate a trace

\tcirc{} Deploy IC}

\critterspec
{STREET THUG}
{group, intelligent, medium}
{Spiked bat (1d6+1 dmg, c), cheap but powerful pistol (2d8w dmg, s/m)}
{9 Wounds / 1 Armor}
{Gangs plague the sprawls, and turf is everything. During a shadowrun, it’s often a good idea to know whose turf you’re on, who the leaders are, and what kind of crime they’re into. If you run afoul of a gang, you might run into someone like the Street Thug.}
{to guard their turf.}
{\tcirc{} Issue a beatdown

\tcirc{} Gather the crew
}

\critterspec
{GHOUL}
{blind, group, infected, intelligent, medium}
{Bite (2d6b dmg, disease, c), talons (1d6 dmg, 1AP, c)}
{6 Wounds / 0 Armor}
{Ghouls are humans infected with HMHVV, which has modified their genetics such that they have an insatiable hunger for human flesh. Intelligent, and often found in packs in sewers, back alleys, and the squats and slums of the \SW/. Despite their physical blindness, they can be a dangerous enemy indeed.}
{to feed the hunger.}
{\tcirc{} consume essence}

\critterspec
{GOBLIN	}
{horde, infected, small}
{Claw (1d4+1 dmg, c), knife (1d6 dmg, c)}
{4 Wounds / 1 Armor}
{Goblins are the result of a dwarf being infected with HMHVV, resulting in a small, twisted, nocturnal creature that tends to run in large packs. Stumbling across a goblin colony can really ruin your day.}
{to scavenge and collect.}
{\tcirc{} ambush}

\end{multicols}

\newpage
\section{INTRUSION COUNTERMEASURES}
\begin{multicols}{2}
Intrusion countermeasures all possess the fearless and program tags. Use these threats in conjunction with matrix nodes
and armored nodes (see page 33).

\critterspec
{ACID}
{}
{Burnout (reduces hardening by 1), chip burn (reduce CPU by1)}
{4 Wounds / 0 Armor}
{Acid is a version of IC designed to damage cyberdecks, opening holes for other more dangerous IC to use to make the attack.}
{burn through defenses.}
{}

\critterspec
{BLASTER}
{}
{Jolt (1d6 dmg, stun)}
{4 Wounds / 1 Armor}
{Blaster IC is designed to inflict nonlethal damage on a hacker, hopefully knocking him or her out and forcing them to disconnect from the grid. Blaster is fairly common, since it is nonlethal, and can be found even in generally lower-security systems. }
{to knock ‘em out.}
{}

\critterspec
{BLACK IC}
{Intelligent, organized}
{Lethal biofeedback (2d8b dmg)}
{6 Wounds / 2 Armor}
{Black IC is the most feared of all intrusion countermeasures. Used by high-security installations, Black IC is designed for one purpose: to kill intruding hackers. Capable of delivering a lethal burst of biofeedback, the victim of a black IC attack is usually found dead in their rig, bleeding from eyes, ears, nose, and mouth. Black IC is not to be trifled with.}
{to kill.}
{\tcirc{} Finish them off}


\critterspec
{CRASH}
{}
{Segfault (crash one program in your deck)}
{3 wounds / 1 Armor}
{A simple countermeasure designed to shut down unauthorized programs, crash is designed to do one thing: corrupt a running program until it crashes. }
{to mess things up.}
{}

\critterspec
{BINDER}
{camouflage}
{Overload (reduce CPU by 1)}
{4 Wounds / 0 Armor}
{Binder is another simple countermeasure, designed to place extra processing load on a cyberdeck’s CPU to decrease its efficiency. }
{to slow down the intruder.}
{}

\end{multicols}

\newpage
\section{SPIRITS}
\begin{multicols}{2}
\textbf{Note:} given the wide array of spirits and their specific manifestations, the GM is encouraged to tweak these entries as
needed!

\critterspec
{SPIRIT OF MAN	}
{aspect, medium, spirit}
{confusion (targets take -2 forward, s), slam (2d6b dmg, forceful)}
{1 armor / 5 Wounds}
{Spirits of Man include spirits of street, hearth, and field, domains intimately linked to the activities of humankind. Known more for their desire to guard and protect an area rather than their innate hostility, they are nonetheless dangerous when their ire is provoked.}
{to guard what man has made.}
{\tcirc{} prevent threats from entering

\tcirc{} cause an accident}

\critterspec
{SPIRIT OF EARTH	}
{aspect, spirit variable size}
{hurl rock (1d8 dmg, forceful), punch (2d6b dmg, forceful)}
{4 Armor / 7 Wounds}
{Spirits of Earth dwell in the very soil and mountain and rock on which life takes root. They usually manifest as beings of rock and dirt, their aspects making them tough to injure. Their powers vary, but as all natural spirits they are motivated to guard their domain.}
{to protect the land.}
{\tcirc{} engulf an intruder

\tcirc{} surge up from the ground}

\critterspec
{SPIRIT OF AIR	}
{aspect, spirit, small, medium}
{fling (1d6+1 dmg, forceful, c), noxious cloud (1d6 dmg, area, poison)}
{3 Armor / 6 Wounds}
{Spirits of Air are capricious beings who dwell in the domain of air. They manifest as howling winds, cold gusts, and vaguely humanoid clouds. Their insubstantial nature makes injuring them difficult.}
{to trick.}
{\tcirc{} move at blinding speed

\tcirc{} toy with an enemy}

\critterspec
{SPIRIT OF WATER	}
{aspect, spirit, small, medium}
{slam (2d8b dmg, c)}
{2 Armor / 7 Wounds}
{Spirits of Water are methodical and inexorable, and take pride that the world will eventually return to the water whence it came. They can be summoned anywhere there is a body of water or river, and they are powerful enemies indeed.}
{to flow}
{\tcirc{} drown the threat

\tcirc{} flow through and around}

\critterspec
{INSECT SPIRIT	}
{aspect, spirit, small/medium/large}
{bite (1d8 dmg, poison, c), strike (2d6b dmg, c)}
{3 Armor / 6 Wounds}
{Insect Spirits are summoned by Insect Shamans, who must ``invest'' a living host with the spirit (since it lacks the capability to materialize). This process is generally done to involuntary hosts, and the results are horrific. Insect Shamans and Insect Spirits are never something to willingly ``get to know.''}
{to breed.}
{\tcirc{} summon the swarm

\tcirc{} scuttle just out of sight}

\critterspec
{TOXIC SPIRIT	}
{aspect, spirit, small/medium/large}
{throw toxin (2d6b, poison, c), poison punch (1d6+1 dmg, poison, c)}
{2 Armor / 10 Wounds}
{Toxic spirits are summoned by toxic shamans from domains that have been corrupted by pollution and other manmade evils. These spirits are as twisted as the domains from which they come.}
{to pollute.}
{\tcirc{} corrupt the environment

\tcirc{} leave their mark}


\end{multicols}

\newpage

\invisiblepart{SPRAWLS}
\section{SPRAWLS}
\begin{multicols}{2}
You could look at shadowrunning as a series of discrete missions, episodes in an ongoing story of quasilegal adventuring.
Ideally, however, the story you weave when you play and/
or GM this game will take place in a world that feels like it’s
alive and breathing, full of real people with realistic motivations, and happening in a place with its own character and
appropriately cyberpunk feel.

Obviously, your adventures have to happen somewhere,
and in the Awakened world of the 2050’s, most of the time
``somewhere'' is one of the vast urban regions that grew up
around the cities of the early 2000: the \textbf{Sprawl}.

Whether through urban growth, massive construction projects
by the megacorporations, mergers, or political realignmen,
many cities have grown so large that they a single coherent
``city plan'' is laughable. Because of this, the environments
within a single city are wildly varied: you can go from glittering financial sector to funky entertainment districts to rumbling industrial zones to blasted near-wastelands of poverty
and deprivation from the comfort of mass transit.

Some things don’t change, though. Every sprawl has it’s own
character, it’s own particular vibe. There are always factions
fighting for something, always people looking for an edge.
People like to have influence, and they’ll use the tools at their
disposal to get it. And frequently, you will be one of those
tools.

\subsection{CREATING A SPRAWL}
In \SW/, we use a system quite similar to creating a
Front in Dungeon World to characterize a Sprawl. Of course,
since Shadowrun takes place in a future version of our own
world, you’re welcome to use this system to decide how a
real-world city (for instance, oh, let’s say Seattle). However,
nothing is stopping you from making one, if you want to
place a new city in the world. You’re in control!

The big difference between Dungeon World Fronts and Sixth
World Sprawls is that Sprawls have the added element of
geography and locale. A Sprawl helps the GM keep track of
both individual forces at work in the world (as with a Front),
but also lets the GM and group define the broad conflicts that
exist over a particular location.

The basic process for creating a Sprawl is as follows (each step
will be explained in more detail):
\begin{dent}
1.	Allocate 5 points among the three main Influences:
\textbf{Man}, \textbf{Magic}, and \textbf{Machine}.

2.	For each point assigned to an Influence, pick a Peril (you
can pick the same Peril twice).

3.	For each Peril, choose a Crisis, and describe how it will
manifest.
\end{dent}

\subsection{INFLUENCES}
\textbf{Influences} are the broad forces acting on a city, which exist
in a constantly shifting equilibrium. There are three influences:

\textbf{Man} is the influence of humanity and its organizations. In
this sense, man represents the influence of people and the
organizations they run on the city: corporations, criminals,
politicians (but I repeat myself), syndicates, religions, celebrities, and so forth.

\textbf{Magic} is the influence of the Awakened and the Astral upon
a city. Often this is tied to the astral beings that populate the
land on which the city stands, but it also includes the desires
and activities of the magically active beings who dwell there
(or who might wish to): mages, dragons, spirits, even paranormal creatures may all exercise their influence on the city.

\textbf{Machine} is the influence of technology, the Matrix, and the
reality of human augmentation. In this modern world, machines and technology are a powerful an influence on the way
people think and feel.

\subsubsection{ALLOCATING INFLUENCE}
The first step of the City Creation process is to allocate influence. The GM should allocate 5 points among the three
Influences, representing the balance or relative weight of that
Influence on the Sprawl in general.
\begin{dent}
\textbf{Example:} \textit{Tanner is creating a Sprawl for Buffalo, NY. He
chooses to allocate 3 to Man and 1 each to Magic and Machine. Buffalo, right now, is the prize in a struggle between
organized crime and megacorporations, while magic and
machine have a subtler influence.}
\end{dent}
\subsection{PERILS}
Each Influence on a city is characterized by one or more Perils: the specific entities, organizations, and creatures that embody the influence in question. Perils vary widely, and are
selected by the group as the city is being created. Creating a
Peril is as simple as one group member suggesting it. Several
categories of perils are presented below, as inspiration.
Choose one peril for each point assigned to an influence (so a
city with Magic 2 would need 2 perils associated with Magic).
You can assign multiple points to the same Peril, representing
competing interests from the same category of danger.
Example: Tanner’s Buffalo Sprawl is coming along. The
next step is identifying Perils for each Influence area. For
Man’s influence, he needs to assign 3 points to perils of
Man. He assigns one to Megacorporation once and two to
Syndicate (he’s thinking about a mob war brewing).

\subsubsection{PERILS OF MAN}
\textbf{Megacorporations} \textit{(impulse: to boost the bottom line)}

Be it one of the Big 10 megacorps, or some poor little rank A,
all corporations need as much help as they can get. What that
help is may be sketchy, but you have no problem with that.

\textbf{Leagues} \textit{(impulse: to influence you)}

Leagues are groups of people with political agendas, be they
either good or misplaced. Policlubs, local governments, merc
squads, terrorist cells, religions, shadow groups, presidents
and more are trying to spread their own version of reality.
Sometimes quietly, other times with a bang.

\textbf{Syndicates} \textit{(impulse: to control the streets)}

As long as there has been crime, someone has tried to organize it. From street gangs to the Triads, the Yakuza, and the
Mafia, organized and not-so-organized crime eyes the sprawl
with hungry and calculating eyes.

\subsubsection{PERILS OF MAGIC}

\textbf{Energies} \textit{(impulse: to empower)}

We pretend that magic is a science to be studied in the halls
of academia, but the wild and unpredictable power of the Astral and Metaplanes, power sites, ley lines, mana surges and
mana storms make a mockery of our learning.

\textbf{Orders} \textit{(impulse: to achieve eldritch ends)}

Orders are those groups of people with a strong interest in
magic. They can range from noble universities and research
organizations to fanatical cults of dark magic. Be it Atlantean
artifacts to Blood Magic, they want to push, discover and
convert.

\textbf{Awakened} \textit{(impulse: to survive and thrive)}

Not all people affected by the Awakening are metahumans.
In fact, most aren’t. There’s a whole world out there of paracritters, free spirits, dragons and metasapients such as centaurs. Some are in power, some want to be in power, and
some simply want to survive.

\subsubsection{PERILS OF MACHINE}

\textbf{Matrix} \textit{(impulse: to absorb and accumulate)}

The Matrix is just a network of 0’s and 1’s...right? Not if you
ask a Hacker. The Matrix is a living, breathing, evolving entity
that we’ve come to take for granted. But in its unvisited or
forgotten corners and gleaming graphical citadels, what feeds
on the information we produce?

\textbf{Technology} \textit{(impulse: to connect and isolate)}

From ubiquitous surveillance, tailored marketing, and betterthan-life virtual reality to orbital space stations, underwater
compounds, and teeming arcologies, it’s hard sometimes to
tell whether we’re using technology, or it’s using us.

\textbf{Advancement} \textit{(impulse: to relentlessly improve)}

New cyberware, robotics, AI, cloning and more are all coming down the pipeline. Some people are afraid that metahumanity is starting to evolve past its tipping point. Some think
it’s already happened. Whatever the case, it pays to be wary.

\subsection{CRISIS}
\textbf{Crisis} is what happens when a particular Peril accomplishes
its primary aims (which are, obviously, determined by the
GM). Left unchecked, a Peril will always progress toward its
goal—the world lives and breathes, and things happen even
when the player characters aren’t around to witness them.
The progress a Peril makes toward its goals is tracked on the
\textbf{Doom Bar} (more on that later), and when it reaches the end,
whatever Crisis was selected for the Peril goes into effect.
There are five main Crises; when you come up with a Peril,
you must also decide on a Crisis for it, and specify the exact
form it will take.
\begin{dent}

\textbf{Control:} insidious influence, strings being pulled, and puppets dancing to the puppetmaster

\textbf{Destruction:} disaster and mass death befall the city

\textbf{Havoc:} the breakdown of order, law, and control

\textbf{Conquer:} unopposed power, and the freedom to enact
any agenda

\textbf{Corruption:} a blight of some sort—crime, graft, or something dark and unnatural—spreads through the Sprawl
\end{dent}
\subsection{DOOM BAR}
At the end of this document is a reference sheet to help you
record notes about your Sprawl. You’ll note on the Sprawl
Sheet that the section for each Peril has five boxes next to it.
These bars are known as the \textbf{Doom Bar}.

The Doom Bar represents how close the Peril is to fulfilling its
desire. At 1 box, they are in the initial phases of construction
and planning, while at 5 they are moments away from unleashing their plan.

At the start of a campaign, every Doom Bar starts at 1. A GM
then has 3 points to divide between the Perils to modify the
initial state of their Doom Bars.

As the campaign progresses, the action (and inaction) of the
player characters will influence changes in a Peril’s Doom Bar.
For example, blowing a run, helping an enemy accidently, or
not stopping some plan in time are likely to increase a Peril’s
Doom Bar.

When the runners can’t stop a Peril, or when the DM deems
it appropriate, you mark a Doom Box under the appropriate
Peril. During the next adventure, the DM should state as a
side-bar what the results of the increased Doom are.
\begin{dent}

\textbf{For example:}\textit{ Two weeks ago, the team barely escaped
a botched run on a corporate arcology that is performing
strange and dangerous experiments on its citizens without
their knowledge. The failed run caused the corporation to
raise security and step up their project’s timeline, dooming
the citizens now trapped inside.}
\end{dent}
The GM could even choose to increase the Doom on multiple
Perils if it makes sense.

\subsubsection{THE END OF THE DOOM BAR}
If a Peril has 5 boxes, and the GM goes to mark another one,
it’s too late: the Peril has accomplished what they were trying
to do, and their Crisis goes into effect. This could have major
impacts on both the Sprawl and the world.

\subsubsection{REDUCING DOOM BAR}
Runners can, believe it or not, reduce the Doom Bar for a
Peril. If they do something that hampers the Peril, the GM
should erase one Doom Box. If the runners do something
really significant to strike a blow to the Peril, such as blowing
up a Renraku datacenter, the GM reduces the Doom Bar by
two boxes.

A minor setback won’t reduce the Doom, but it will prevent
it from increasing.

If runners ever reduce a Peril’s Doom Bar to 0, the Peril goes
into \textbf{remission}. Remission means the Peril may be gone, or
perhaps it’s just licking its wounds. Either way, a Peril in remission does not show up for 2 adventures. Once that time
is over, the GM can either bring back the Peril at 1 Doom, or
bring in a totally new Peril. If a Peril is ever redudced to 0, it
is a good idea to give the players a free advance to award
them for their skill.
\begin{dent}

\textbf{Example:} \textit{the team pulled off a run that culminated in blowing up the Renraku datacenter mentioned earlier. Renraku
had been slowly subsidizing Matrix usage, trying to cut the
Sprawl off from the main Matrix grids (and thereby achieve
Control). That Peril stood at 2 Doom before the run, but
the GM decides to remove both Doom boxes—reducing
the Doom to 0— due to the success of the run. Renraku
decides to back off the Matrix control plan.}
\end{dent}
However, two sessions later, the team gets word of Renraku
performing some sketchy genetic experiments on Awakened
rats. Looks like Renraku’s back with a new plan.

\subsection{SPRAWL DISTRICTS}
Sprawls are a way to get an idea of the large influences at
work in a particular area, giving you an idea of whch entities
are the movers and shakers of a given city.

\textbf{Districts}, on the other hand, are areas within a Sprawl where
a runner might find him- or herself. Districts are a shorthand
way to record basic descriptive information about different
neighborhoods, areas, and communities within a Sprawl.

The word ``district'' should be interpreted broadly—a small
neighborhood, a glittering financial sector full of high-rise
buildings, and a sprawling industrial zone can all be Districts.

\subsubsection{CREATING A DISTRICT}
A District is described by tags (like equipment and threats),
which provide some descriptive information to help players
and the GM get a handle on an important area.

Creating a district is very simple:
\begin{dent}

1.	Name the District

2.	Determine the core tags of the district (type, economy,
population, and trust)

3.	Determine any other special tags the district may have.

\textbf{Example:} \textit{the GM wants to create an industrial area for
some of the action of this latest run to happen in. He pictures an oil refinery area, full of containers, pits, fences, low
warehouse buildings, tall processing plants, and pipelines
of all sizes crisscrossing the district. Economically, it’s active, though not exactly a ``glittering rich'' place. It’s isolated due to the industry, and polluted with leavings. It’s
also owned by Ares. The tags for this district are} industrial,
average, stable, cooperative, corporate, polluted, isolated.
\end{dent}
\subsubsection{DISTRICT TAGS}
There are four basic or core tags that describe a district, which
are, in order, Type, Economy, Population, and Trust.

\textbf{Type} identifies the general type of district, what kind of things
happen there, and its role in the Sprawl.
\begin{dent}

\textit{Residential:} this district is a place where people live,
whether in housing projects, suburbs, apartments, rowhouse, etc.

\textit{Commercial:} this district is primarily occupied by retail and
service businesses of varying size

\textit{Financial:} this district is primariily occupied by financial institutions such as brokerages, stock markets, banks, and
investment firms

\textit{Industrial:} this district is primarily occupied by heavy industry such as construction, manufacturing, and shipping
firms.

\textit{Entertainment:} this district is primarily occupied by entertainment businesses such as casinos, theaters, clubs, bars,
and sports venues.
\end{dent}

\textbf{Economy} indicates the general financial strength of the district.
\begin{dent}

\textit{Rich:} this district is extremely wealthy, with a great deal of
financial pull in the Sprawl. Examples include high-stakes
financial districts and upper-crust residential areas.

\textit{Affluent:} this district is well-off, with some financial sway.
Examples include luxury residential areas and gated communities, or ritzy entertainment districts.

\textit{Middle-class:} this district has only a modicum of financial
pull, being primarily a middle-class / median income area;
housing is small and efficient, businesses (if there are any)
small as well.

\textit{Poor:} this district is struggling, with little to no resources.
Residences are tiny and shabby, employment is minimal,
and businesses are struggling.

\textit{Slum:} this district is a wasteland, with abandoned buildings, no jobs to speak of, failing (or failed) businesses, and
no monetary influence whatsoever.
\end{dent}

\textbf{Population} describes the size (and growth or decline) of the
inhabitants of a district (or the people employed there, if it is
a business district).
\begin{dent}

\textit{Booming:} the population is large and getting larger fast;
people are moving there, or businesses are expanding
there at breakneck pace.

\textit{Growing:} the population is large and growing, with a
steady (but not explosive) increase in population.

\textit{Stable:} the population is moderate and steady, with only
minor increases and decreases that tend to even out over
time.

\textit{Dwindling:} people are leaving for some reason, whether
because of abandonment by the city, or failing businesses,
or redevelopment. The current population is small, with
numerous abandoned buildings and businesses.

\textit{Abandoned:} this district has been largely abandoned by
businesses and/or residents. The legitimate population
is tiny, and most buildings are empty and decaying. The
largest population by far is likely to be criminals and the
outcast.
\end{dent}

\textbf{Trust} is the final core tag, indicating the districts view of authority, including politicians, law enforcement, and organizations. Remember that this is relative to the 2050’s, where
trust is a little harder to come by anyway.
\begin{dent}

\textit{Cooperative:} the community tends work closely with authority.

\textit{Neutral:} the community is neutral toward authority.

\textit{Reserved:} the community is not inclined to trust authority
figures, though it will not actively hamper their work

\textit{Wary:} the community instinctively suspects authority figures and will not cooperate unless compelled.

\textit{Hostile:} the community is openly hostile to authority figures; law enforcement may avoid the area and it may be
``written off'' by politicians and organizations
\end{dent}

Other tags can be used to add additional description as necessary or for special features of a particular district:
\begin{dent}

\textit{Big name:} a person of significant renown (the GM determines to whom) lives or works in this zone

\textit{Corporate:} this neighborhood is owned, managed, and
serves one of the megacorporations or a subsidiary

\textit{Dense:} tight streets, densely packed homes/businesses,
and narrow passages.

\textit{Despair:} the district is blighted and collapsing, and the despair of the people is palpable.

\textit{Highrise:} this area is predominantly high-rise office and/
or residential buildings with few open areas, but well-organized streets

\textit{Infestation:} there is an infestation of some creature in this
area (e.g. goblins, devil rats, etc.). It generally remains hidden inside buildings and underground. Note that this may
be a natural infestation, or something worse

\textit{Isolated:} although uncommon in the \SW/, there
are some districts that are still difficult to get to, or cut off
from other areas by construction, road modification, and so
forth. Police and emergency response is slowed.

\textit{Lawless:} police presence in this district is absent, and
crime is rampant and unchecked except by the criminals
themselves

\textit{Open:} this area is remarkably devoid of construction, and
has open (perhaps even green) space and room to move
easily (or to move large vehicles)

\textit{Outbreak:} there is a disease outbreak of some sort in this
District; medical services may be present, depending on
the neighborhood’s economic value. If not, quarantine
may be in place.

\textit{Policed:} the neighborhood is regularly patrolled by law enforcement, and response time is short

\textit{Prejudice:} this is a dislike, dismissal, bigotry, or hatred
against a particular category of individuals (perhaps another District, or the police, or orks, or ethnicity)

\textit{Prize:} there’s something in the neighborhood or the land it
sits on that is desired by multiple factions

\textit{Protected:} the neighborhood is protected by some group
(for example, a gang, or a cult)

\textit{Rot:} something poisons this neighborhood, perhaps physically or mentally or spiritually

\textit{Religious:} a religion, cult, or other spiritual movement
holds sway here

\textit{Turf (gang):} this zone is the turf of the indicated gang
\end{dent}

\end{multicols}

\section{WILDS}
\begin{multicols}{2}
Most of the action in \SW/ games will take place somewhere in the byzantine environment of a Sprawl. However,
there are plenty of adventure-ready wild spaces left in the
world. In fact, with the upheaval of the early 2000’s, there’s
quite a lot of new wilderness out there, and at some point or
another, you’ll likely end up crossing through it.

If you want to create a \textbf{Wild}, the process is identical to the
creation of a Sprawl: allocate points among the influence of
Man, Magic and Machine, and then determine appropriate
Perils and Crises to accompany those influences.

\subsection{WILDERNESS ZONES}
Just like Sprawls, a single Wild can contain multiple smaller areas with specific characteristics. These smaller areas are
called \textbf{Zones} (since the word ``district'' doesn’t quite fit). Creating a zone, however, is done the same way as a District:
think of a Zone you want to create, give it a name, and select
the appropriate tags to describe it.
\begin{dent}

\textbf{Example:} \textit{the GM creates a region near Lily Lake, deep in
one of the former National Parks. The GM imagines this
to be a thickly forested area, with steep slopes and deep
gullies. Remnants of some park services buildings (mainly
huts and SAR bivouacs) can be found. It’s mostly populated by small animals band birds, althoug a mated pair of
Piasma call this area home. The tags selected for the Zone
are} forest, rugged, typical, ruins, predator.
\end{dent}

The tags for the zone are explained below.

\subsubsection{WILD ZONE TAGS}
Because many of the tags for Sprawl Districts wouldn’t necessarily apply, some new tag options are presented below.
Wild Zones have the following tag types: \textbf{type}, \textbf{terrain}, and
\textbf{wildlife}.

(The categorizations that follow—which were greatly trimmed
and simplified for game purposes—may cause painful grimacing in ecologists, forestry experts, geographers, and zoologists; I apologize sincerely).

\textbf{Type} describes the general type of biome and climate of the 
zone. 
\begin{dent}

\textit{Plains:} characterized by low rolling hills, open fields of 
grass or scrub, high visibility and winds. Climate varies per 
season. 

\textit{Desert:} characterized by aridity, heat, rolling or rocky ter- 
rain. Deserts may be arctic, but this tag primariliy deals 
with the ``hot deserts'' of the world. 

    \textit{Aquatic:} a water-based zone, either riverine, limnic, or 
    oceanic. Depending on specifics could be hostile (if sub- 
    aquatic) 

    \textit{Forest:} characterized by a high density of trees of vari- 
    ous types (different categories of forest will have differing 
    dominant tree types); terrain varies 

    \textit{Jungle:} a land area covered with thick, dense vegetation, 
    typically in a tropical area 

\textit{Polar:} cold northern or southern lands in the polar latitutdes, including arctic regions
\end{dent}

\textbf{Terrain} describes the zone’s physical features and topography, and how difficult or easy it may be to traverse.
\begin{dent}

\textit{Flat:} little to no change in elevation, with only small hills
and depressions

\textit{Rolling:} smoothly transitioning hills, with at times sizable
changes in elevation.

\textit{Wetland:} an area saturated with water, such as a bayou,
delta, swamp, fen, or bog

\textit{Rugged:} terrain with sudden changes in elevation, rocky
outcrops, or thick vegetation that is difficult to navigate
directly or maneuver through

\textit{Mountainous:} rough terrain in a mountainous region, with
large changes in elevation; tiring, demanding terrain

\textit{Broken:} the land is shattered and extremely rugged, very
difficult to cross (almost impassable), and full of blind runs,
rocky outcrops, sharp ridges and technically demanding
terrain.

\textit{Exotic:} the terrain is unusual in some way and not generally encountered; deep subaquatic regions, highly unusual
rock formations, strange caves, and so forth would be examples of exotic terrain
\end{dent}

\textbf{Wildlife} describes the flora and fauna of the area, as well as
the relative biodiversity of the zone.
\begin{dent}

\textit{Limited:} the zone’s biodiversity is low, marked by only a
few kinds/categories of plants and animals

\textit{Typical:} the zone’s biodiversity is typical for the Sixth
World, having several types of animal and plant species
represented

\textit{Diverse:} the zone is populated by a fairly varied number of
different species, both flora and fauna; edible species are
reasonably easy to find

\textit{Rich:} the zone is rich in different animal and plant species;
it is a busy place

\textit{Hotspot:} the zone is a biodiversity hotspot, teeming with
highly varied species of plants and animals
\end{dent}

Other tags may come into play to describe a particular wilderness zone. In addition to the tags below, the tags prize,
protected, and infestation are also applicable.
\begin{dent}

\textit{Awakened:} this zone is heavily imbued with magic, whether it be from ley-lines, artifacts, ritual, or other unknown
reason, magic is almost tangibly present.

\textit{Blasted:} some cataclysmic event happened here, and the
scars remain visible.

\textit{Extreme:} the zone is an extreme representative of its
type—a fiercely hot desert, bitterly cold polar region (e.g.
 Antarctica), a dense jungle.

\textit{Megafauna:} the zone contains a relatively high population
of megafauna (animals exceeding 45kg/100lb) such as
deer, large paranimals, and the like.

\textit{Polluted:} this zone is heavily polluted; water is likely undrinkable without treatment and animals and plants dangerous to eat.

\textit{Predator:} there is an apex predator (or mated pair) that
considers this zone its hunting grounds. Be sure to identify
the predator (because your players will ask about it, and
you may have to answer!)

\textit{Remote:} the zone is a long way from civilization. You’re
on your own.

\textit{Ruins:} this zone is composed of, or contains, the abandoned remnants of (meta)human construction.

\textit{Seismic:} this zone is prone to seismic activity, which may
pose a threat

\textit{Storms:} this zone is prone to storms of some sort: electrical, rainstorms, windstorms, snowstorms. These may lead
to related events (fire, flood, etc.)

\textit{Territory:} this zone is the territory of a particular individual
or pack; intruders may be met with extreme aggression.
Make sure to identify the type of creature.

\textit{Wasteland:} this zone is essentially dead—native fauna and
flora has mostly died, water may be scarce or toxic, the
ground poisonous. Inhabitants of this zone (if any) may be
twisted mutants, odd Awakened creatures, strange infestations, or desperate squatters
\end{dent}

\end{multicols}

\invisiblepart{CREATING WEAPONS \& GEAR}
\section{CREATING GEAR}
\begin{multicols}{2}
\SW/ uses a ``template-based'' gear model for most
equipment used in the game: rather than provide extensive
lists of individual items, such as firearms, there are basic templates for broad categories of item, and rules to modify the
templates to suit the player’s needs and desires.

For example, rather than a list of ten heavy pistols, there is one
template for \textit{Heavy Pistol}, with certain basic tags. From there,
the player may add or remove tags based on the guidelines
for doing so. Use these entries to come up with your own, or
modify these as needed.

The rules that follow are optional and experimental (so they’re
not guaranteed to be completely balanced, and you may end
up using the time-tested practice of ``make the item using the
rules, then, when it doesn’t feel right, change stuff'').

\begin{dent}

\textbf{Bonus Limits:} in general, with the exception of tags that
are the equivalent of wounds, no quality of a piece of gear
may have a value higher than +3.
\end{dent}

\subsection{GENERAL TAGS}
As explained in the \textbf{Gear} section, all gear has one or more descriptive tags (not including its price) describing its particular
qualities. Tags may be descriptive (to aid with the fiction), or
have mechanical import. The following tags apply to multiple
types of equipment.
\begin{dent}

\textit{2-hand:} this item must be used with both hands

\textit{Armor +n:} grants a +n bonus to existing armor

\textit{Armor n:} grants n Armor (for vehicles or drones, indicates
armor rating, and is abbreviated arm)

\textit{Arcane:} can only be used by magical archetypes

\textit{Area:} affects multiple targets

\textit{+Bonus:} grants a bonus to a particular move; e.g. +1 to
Stay Frosty

\textit{Conceal:} this weapon or item is easily hidden and will not
be spotted by enemies

\textit{Damage n:} the amount of damage a weapon or other item
deals. Abbreviated dmg

\textit{Heal n:} restores n wounds

\textit{Loud:} noisy and audible to anyone with functionin hearing;
for weapons, it means the weapon cannot be suppressed

\textit{Messy:} deals damage in a particularly gruesome way

\textit{Obvious:} cannot be concealed, or is immediately visible to
any observer

\textit{Range:} the range(s) at which the weapon or other attack is
effective. Ranges are close (c), short (s), medium (m), and
long (l).

\textit{Special (description):} if the effect of the item requires explanation, use this tag.

\textit{Stun:} this weapon or attack deals Stun damage only

\textit{Subtle:} not easily noticed (as opposed to conceal, which
means it is unnoticeable)

\textit{Supply n:} the amount of supplies or uses you can get out
of an item. Each use of the item consumes 1 supply (unless
otherwise stated).
\end{dent}

\subsection{CREATING WEAPONS}
The templates below represent a starting point to begin customizing a weapon. There are only a few templates, since
most of the rest of the process can be handled through customizing and modifying the item’s price. The basic weapon
templates are:

\textbf{melee weapon} \textit{[range c, dmg 1d6, 150¥]}

\textbf{light pistol} \textit{[range s/m, sa, dmg 1d6, ammo 3, 300¥]}

\textbf{heavy pistol} \textit{[range s/m, sa, dmg 1d8, ammo 2, 450¥]}

\textbf{submachine gun} \textit{[range s/m, sa/bf, dmg 1d8, ammo 3,
700¥]}

\textbf{longarm} \textit{[range s/m/l, sa, dmg 1d10, AP 1, obvious,
ammo 4, 600¥]}

\textbf{heavy weapon} \textit{[range m/l, fa, dmg 1d12, AP 2, loud,
obvious, stabilize, messy, ammo 4, 2,500¥]}

\subsubsection{DAMAGE EXPRESSIONS}
Damage expressions can be put in order from the smallest
damage die (1d4) through the largest (1d12), with modifications in between. Here’s how the damage options in Sixth
World progress:

\begin{center}
\begin{tabular}{c}
\toprule
Average Damage (low to high)\\
\midrule
1d4 \\
2d4b \\
1d4+1 \\
d6 \\
2d6b \\
1d6+1 \\
1d8 \\
1d8+1 \\
1d10 \\
2d8b \\
1d10+1 \\
1d12 \\
2d10b \\
1d12+1 \\
2d12b \\
\bottomrule
\end{tabular}
\end{center}

\textbf{Notes:}
\begin{dent}
1. No [w] rolls. The
``worst'' roll modifier is
a significant penalty,
especially as the die type
gets bigger. Save it for
broken gear and things
that interfere with the
characters.

2. The progression isn’t
nicely ordered, because
the [b] roll gets progressively better as the dice
type gets higher.
\end{dent}

\subsubsection{WEAPON TAGS}
Weapons use the following tags (in addition to the general
tags from the preceding page):
\begin{dent}

\textit{AP n:} this weapon ignores n points of armor.

\textit{Auto:} this weapon can fire in full auto mode (take +1 to
suppression fire). Treat as burst otherwise. Abbreviated fa.

\textit{Burst:} this weapon fires in burst mode (mark off 1 ammo to
deal +1 damage). Abbreviated bf.

\textit{Chem:} this weapon delivers a chemical agent of some kind
to the target; depending on the delivery mechanism, armor may be ignored.

\textit{Forceful:} when this weapon deals damage, it also deals 1
stun

\textit{Fuzed:} this weapon cannot be used at less than the shortest range increment listed

\textit{Reload:} after using this weapon, it takes more than a moment to reload it.

\textit{Semiauto:} this weapon fires one shot every time the trigger is pulled. Abbreviated sa.

\textit{Stabilized:} this weapon cannot be fired except from a bipod, tripod, or supported position.

\textit{Suppressed:} this weapon makes little to no noise when
fired

\textit{Thrown:} this item can be throw. If thrown, the range is
short.

\textit{Vented:} the weapon has recoil venting, granting +1 to
Suppression Fire
\end{dent}

\subsection{CUSTOMIZING WEAPONS}
To build a custom weapon, follow these steps:

\begin{dent}
1.	Choose base template.

2.	If creating the weapon during character creation, you
have 3 points to spend on customizations. If you’re buying it, the only limit is how much nuyen you’ve got on
your credstick.

3.	Modify the base template as you like: adjust damage,
rate of fire, ammo, and other tags by spending points or
adjusting the final price of the weapon.

4.	If you like, give your new weapon a name.
\end{dent}

\subsubsection{WEAPON CUSTOMIZATIONS}
\textbf{HI-POWER}

Increasing the power of a weapon raises the damage expression (and, if the damage expression becomes a [b] roll,
also increases the consistency of that damage somewhat, reflecting an ``in-world'' improvement in control). Up-gunning
a weapon raises the damage expression one step (use the
table on the preceding page to figure out the new damage).
You can increase a weapon by a maximum of 3 steps (e.g.,
1d6 to 1d8); each increment costs 1 point or adds 50¥ to the
base cost.

\textbf{LOW POWER}

The opposite of increasing power. You can reduce a weapon’s
damage expression by up to 2 steps to gain points for other
options, or to reduce the price. Each decrement provides 1
point or reduces the cost by 25\%.

\textbf{EMBEDDED}

The gun is built into an otherwise unremarkable non-cyberware object (such as a camera or briefcase). Doing so makes
it undetectable, but reduces accuracy. Subtract 1 from the
damage. Cost: -50\% / -1 point.

\textbf{CHANGING FIRE MODES}

You can add or remove firing modes from a weapon. Adding
a fire mode is a positive, while removing fire modes from a
weapon that already has them is negative. Note that if you
restrict a weapon to burst or full-auto mode, it always costs
ammunition to use, which can be a fairly significant penalty.

\textbf{PRICE REDUCTION}

When building a new weapon using the point by system, if
you have unused points you can use them to reduce the final
price of the weapon. Drop 50¥ from the price per point spent.

\subsubsection{MODIFYING TAGS}
You can add or remove tags from weapons, paying for (or
getting rebates back) depending on the tag. Positive tags cost
build points or more nuyen, while negative tags grant more
points or reduce the price of the weapon. The table below
lists the tags as well as their cost.

\textbf{Note:} positive and negative is relative to the tags the weapon already has. In other words, adding burst fire mode to
a pistol is a positive thing. If you removed it from an SMG
instead, then it would be a negative modification. The table
below simply indicates the value of the tag in points or nuyen added or subtracted when modifying a base template.

\begin{center}
\begin{tabular}{lp{5.5cm}}
\toprule
Tag Type& Tags\\
\midrule
1 pt / 50 ¥ & 2-hand, add/remove range increments, add/reduce ammo, additional fire modes, suppressed, vented, +bonus, subtle, stabilized, loud, messy, stun, chem, smart\\
2 pts / 100¥ & AP, forceful, ignores armor(e), 2-hand, fuzed, obvious, reload, conceal\\
\multicolumn{2}{p{5.5cm}}{e) - exceptional tag, twice the normal value

(m) - melee weapon tag
}\\
\bottomrule
\end{tabular}
\end{center}


\subsection{CREATING CYBERDECKS}
Cyberdecks are the essential tool of the hacker. They are the
Hacker’s connection to the Matrix, his weapon, his instrument, his toolbox, and his armor when he’s throwing down
with serious Matrix security.

\textbf{TAGS
}
\begin{dent}

\textit{CPU:} the raw processing power of the deck

\textit{Mask:} the stealthiness of a cyberdeck

\textit{Hardening:} the deck’s resistance to damage

\textit{Storage:} the deck’s capacity for loaded programs
\end{dent}

\subsubsection{DECK TEMPLATES}
Each template below provides a number of Gear Points (gp)
to distribute among the four tags listed above. Lower end
decks offer fewer points to play with, while the high-end
dream decks can be powerful rigs indeed. All decks start with
a base of 8 storage, and no deck can have a tag higher than 3.

\textbf{Entry Level} [\textit{3 gp, 25,000¥}]

\textbf{Mid-Range} [\textit{4 gp, 50,000¥}]

\textbf{High-End} [\textit{5 gp, 75,000¥}]

\textbf{Elite} [\textit{6 gp, 100,000¥}]

\subsection{CREATING VEHICLES}
Vehicles have the following tags describing their capabilities:

\begin{dent}

\textit{Power (pwr):} the vehicle’s horsepower, speed, and acceleration.

\textit{Armor (arm):} the vehicle or drone’s armor rating.

\textit{Frame (frm):} the vehicle’s or drone’s resilience. This is the
equivalent of the vehicle’s wounds. Remember that vehicles take half damage from small arms, and none from
melee weapons.

\textit{Sensors (ssr):} the quality of the vehicle’s sensors (used
when Checking the Situation while driving or piloting the
vehicle)

\textit{Seats n:} the number of people who can occupy the vehicle,
including the driver or pilot

\textit{Fuel:} fuel capacity
\end{dent}

\subsubsection{VEHICLE TEMPLATES}
When designing a vehicle, select a template below, distribute
the indicated Gear Points (gp) among the 4 core stats as desired. Power, armor and sensors may not have a value higher
than 3.

The base fuel and frame of the vehicle will be indicated in
each template. You can spend as many gear points as you
wish to increase those tags.

\textbf{BIKES}

\textbf{Scooter} [\textit{3 gp, 3 fuel, frm 3, seats 1, 1,800¥}]

\textbf{Street Bike} [\textit{5 gp, 3 fuel, frm 4, seats 2, 6,500¥}]

\textbf{Racer} [\textit{4 gp, 3 fuel, frame 3, +1 pwr, seats 1, 9,500¥}]

\textbf{Offroader} [\textit{5 gp, 3 fuel, frm 4, seats 2, 4,850¥}]

\textbf{Hog} [\textit{6 gp, 2 fuel, frm 5, seats 2, 17,500¥}]

\textbf{CARS}

\textbf{Economy} [\textit{4 gp, 3 fuel, frm 5, seats 3, 10,000¥}]

\textbf{Standard} [\textit{5 gp, 3 fuel, frm 6, seats 4 16,000¥}]

\textbf{Sports} [\textit{6 gp, +1 pwr, frm 5, 2 fuel, seats 2, 36,000¥}]

\textbf{Luxury} [\textit{6 gp, +1 ssr, frm 6, 3 fuel, seats 5, 85,000¥}]

\textbf{Exotic} [\textit{7 gp, +1 arm, frm 6, 2 fuel, seats 6, 200,000¥}]

\textbf{TRUCKS}

\textbf{Van} [\textit{6 gp, frm 8, 2 fuel, seats 8, 35,000¥}]

\textbf{Light Truck} [\textit{6 gp, +1 pwr, frm 10, 2 fuel, seats variable,48,000¥}]

\textbf{Heavy Truck} [\textit{7 gp, frm 12, +1 pwr, 2 fuel, seats variable,125,000¥}]

\textbf{ROTORCRAFT / VTOL}

\textbf{Helicopter} [\textit{6 gp, +1 ssr, frm 10, 3 fuel, seats 6, 100,000¥}]

\textbf{VTOL} [\textit{7 gp, +1 ssr, 4 fuel, frm 10, seats 8, 355,000¥}]

\subsection{CREATING DRONES}
Drones are built the same way as vehicles, and have most
of the same qualities. However, drones have the following
additional stats:

\begin{dent}
\textit{Tactical (tac):} the quality of the drone’s tactical expert system, which comes into play when the drone is in autonomous mode. Tac may not have a value higher than 3.
\end{dent}

\subsubsection{DRONE TEMPLATES}
\textbf{Ground Surveillance} [\textit{3 gp, +1 ssr, frm 4, 2 fuel, 1,800¥}]

\textbf{Ground Sentry} [\textit{4 gp, +1 arm, 1d6 dmg, frm 6, 2 fuel, 4,500¥}]

\textbf{Ground Combat} [\textit{4 gp, +1 tac, 2d6b dmg, frm 8, 3 fuel, 8,000¥}]

\textbf{Air Surveillance} [\textit{3 gp, +1 ssr, frm 3, 2 fuel, 2,500¥}]

\textbf{Air Sentry} [\textit{4 gp, +1 ssr, 2d4b dmg, frm 4, 2 fuel, 12,000¥}]

\textbf{Air Combat} [\textit{5 gp, +1 tac, 1d8 dmg, frm 6, 3 fuel, 22,000¥}]

\end{multicols}


\invisiblepart{CREATING CYBERWARE}
\section{CREATING CYBERWARE}
\begin{multicols}{2}
Cyberware, like other equipment in \SW/, can be described using a set of tags. Generally cyberware augments
a character either by providing capabilities that the character did not have (nor could have naturally) such as a direct
connection to a device or foot-long razors on their wrists, or
enhances an existing capability such as their reaction time or
toughness.

Since it’s possible to describe cyberware in terms of tags,
it is also possible to perform some customization of cyberware devices (although they’re usually pretty fixed in their
performance). The most typical customization possible is in
the cyberware’s grade, which indicates the general level of
enhancement it provides, and in its damage capability (for
cyberweapons).

\subsubsection{ACTIVATION CYBERWARE}
Cyberware is activated by spending Edge. By default, a cyberware system requires the user to spend 1 Edge to activate
it, each time they wish to use it (that is, each time the user
wishes to gain its benefits). The \textit{toggle} and \textit{always on} tags
modify this general rule, as described in the \textbf{Other Cyberware Tags} section.

\subsubsection{INSTALLING CYBERWARE}
Installation of cyberware is an advanced surgical procedure
that must be taken during downtime or legwork time due
to recovery time. There are two general types of cyberware.

\textbf{Implants} are cyberware that are installed inside the recipient’s body. The extent of the installation and the amount of
Essence lost varies; a datajack is a relatively trivial installation,
while wired reflexes involve an extensive whole-body procedure and a considerable amount of recovery time. Any cyberware item aside from, obviously, a replacement part can be
installed as an implant (for example, you don’t need to have
cybereyes to get cybernetic low-light vision).

\textbf{Full Replacements} are cyberware that fully replaces an
equivalent part of the recipient, such as eyes, ears, or limbs.
Like implants, the invasiveness of such a procedure varies,
but replacements are in general more invasive than implants.
By themselves, replacements offer no additional capability.
However, full replacements have the following benefits:

\begin{dent}

\tcirc{} full replacments can have optional components installed
into them with no further essence cost; instead, the
component takes up capacity equal to its essence cost.
\end{dent}
\subsubsection{ESSENCE COST}
Every time cyberware is installed in a metahuman, it costs a
bit of their essence. This loss depends on the invasiveness of
the surgery required, the biological systems modified, and
the grade of the cyberware. State of the art cyberware has a
significantly decreased essence cost, but is also significantly
more expensive. A character may not reduce their Essence
below 0.

The tag for Essence cost is simply essence n, where n is the
amount of essence the item costs to install.

\subsubsection{PRICE}
Unfortunately, there’s no ``generic'' piece of cyberware, so
there’s no ``standard price'' to start from when customizing
cyberware. The cost of the implant is based on a lot of factors:
how invasive it is, how technically complex, what exactly it
does, and how much the legal system and corporations frown
upon John Q. Citizen having something that does that. Your
standard datajack is an innocuous device, and might cost you
around a thousand nuyen. On the other hand, having a pistol
hidden inside your arm is probably going to cost a lot more,
because no matter how convincing you are, most people
won’t believe you when you say it’s just for target practice.

One of the jobs of the GM will be, if you use these customization rules, to figure out the base prices for different items.
Some very loose (essentially guesswork) guidelines are given
below:

\textbf{Common Legal Items}

\begin{dent}


\tcirc{} Minimally invasive: 1,000 - 5,000¥

\tcirc{} Moderately invasive: 7,500 - 40,000¥

\tcirc{} Highly invasive: 50,000 - 100,000¥
\end{dent}
Restricted or regulated items will be more pricey. A premium
of 25-50% over the cost of an equivalently invasive legal item
might be appropriate.

For flat-out illegal ‘ware, the sky’s the limit. It’s illegal to have
the augmentation in the first place, so the black market can
pretty much ask whatever it wants.

\subsection{CREATION RULES}
Although there is a list of ``typical'' cyberware in this section,
it is possible to create or customize cyberware items using
the rules in this sectionThe steps below describe how to create a new piece of cyberware (these are explained in more
detail below):

\begin{dent}


1.	Select either \textit{standard} or \textit{sota} grade.

2.	Select the item’s general function.

3.	Decide how invasive the augmentation is, noting the
base essence and price, as well as selecting the appropriate benefit based on function.

4.	Choose additional tags, adjusting the final essence and
nuyen cost as necessary.

5.	Write out the tags, and name the item.
\end{dent}
For items installed in full replacements, after you figure out
the final cost and stats, you may want to record the individual
components down, and simply note that they’re installed in
the containing implant, rather than jam everything into one
endless and unintelligible stream of tags.

\subsubsection{CYBERWARE QUALITIES}

\textbf{\large GRADE}

\textbf{Standard:} this is your basic ``off the shelf'' augmentation,
and is the default grade. Standard cyberware has the following characteristics
\begin{dent}

\tcirc{} Essence cost of 1, 2, or 3, depending on invasiveness

\tcirc{} Full replacements can have 2 add-on components
\end{dent}
\textbf{State of the Art:} state of the art (sota) cyberware uses the
latest technology to improve performance and customize
it to your specific physiology and genetic makeup, reducing its essence cost. SOTA cyberware has the following
characteristics:
\begin{dent}

\tcirc{} Essence cost is 0, 1, or 2 (for replacements with addons, add up the total cost for all components, then multiply). Yes, minimally invasive implants cost no essence.

\tcirc{} Base cost is multiplied by 3

\tcirc{} Full replacements can have 3 add-on components (these
must also be SOTA-grade)
\end{dent}
\textbf{\large FUNCTION}

Cyberware is highly varied, but has two general mechanical
functions in the game: \textbf{modify a move}, or \textbf{grant a new capability}. Therefore, a cyberware item may have one of the
following two tags:
\begin{dent}

\textit{modifies:} many enhancements affect a specific move or
moves; this tag describes the specific modification. For example, a smartlink alters the Rock \& Roll move, so the
tag list will contain modifies(Rock \& Roll), along with a
description of the specific benefit.

\textit{ability:} the implant adds a new ability the recipient did
not previously have (for example, armor, low-light vision,
sound damping, a gun hidden in their toe, etc.). The ability
added is usually evident from the name of the item (e.g.
``Thermographic Vision Implant''), but if not, put the specific ability in parenthesis after this tag. Use the \textit{special} tag to
describe specific effects, as needed.
\end{dent}

\textbf{\large INVASIVENESS}

The extent of the surgery required to install cyberware dictates both its base essence cost and its base cost in nuyen. In
general, the more substantial the augmentation or the more
fundamental or sensitive the systems being modified, the
more invasive the surgery.

\textbf{Level 0:} this level of cyberware is minimally invasive, requiring little essence loss. This type of cyberware has the following characteristics:
\begin{dent}

\tcirc{} Base essence cost of 1

\tcirc{} Typical Systems: device links, vision enhancement,
hearing enhancements, replacement eyes, replacement
ears, installed devices, small compartments, implanted
light blade, implanted holdout pistol

\tcirc{} \textbf{Benefit (choose 1):} new ability, 1d4 damage, special
effect
\end{dent}
\textbf{Level 1} cyberware requires a bit more surgical intervention
to install, a longer recovery time, and has more of an impact
on the recipients system. This level of augmentation has the
following characteristics:
\begin{dent}

\tcirc{} Base essence cost of 2

\tcirc{} Typical Systems: armor implants, hazard protection,
wired reflexes, skillwires, compartments, implanted
medium blade, implanted light pistol

\tcirc{} \textbf{Benefit (choose 1):} new ability, 1d6 damage, +1 Armor, Hold 1, special effect
\end{dent}
\textbf{Level 2} cyberware is highly invasive and complex, requiring
considerable modification of the recipient. It brings with it a
correspondingly high monetary and essence cost. This level
of augmentation has the following characteristics:
\begin{dent}

\tcirc{} Base essence cost of 3

\tcirc{} Typical Systems: replacement limbs, wired reflexes, armor implants, skillwires, move-by-wire system, cybertorso, hazard protection, implanted heavy pistol

\tcirc{} \textbf{Benefit (choose 1):} new ability, 1d8 damage, +2 Armor, Hold 2, special effect
\end{dent}
\subsubsection{OTHER CYBERWARE TAGS}
Cyberware can use many of the same general tags that other
equipment use, such as \textit{armor}, \textit{range}, or \textit{obvious}. The tags
below are unique to cyberware.
\begin{dent}

\textit{add-ons:} this is installed in an existing piece of cyberware,
instead of independently. The item takes up capacity equal
to its essence cost. \textbf{Note:} components \textbf{do not} inherit the
always on or toggle tags from the item in which they are
installed.

\textit{always on:} the implant remains on all the time. If adding
this tag to an item that modifies a move, multiply the cost
of the implant by 2. Full replacements always have this tag,
but their components do not inherit it.

\textit{n capacity (cap):} the cyberware item has capacity for n
add-on items. If add-ons are listed, this tag should show
the remaining capacity. Only full replacements can have
the cap tag. Capacity can be added in increments of 0.5 by
increasing the base cost of the item by 25%.

\textit{device:} this implant is a device of some sort (usually a
weapon or computing tool) that does not offer sensory
modification. If installed as an add-on, it must be installed
in a replacement with the device tag.

\textit{link (device):} this cyberware must be connected to the
proper kind of device to be effective (for example, a smartlink must be connected to a weapon with a smartgun system)

\textit{loaner:} this implant was given to you by an organization
lots of money, and they expect you to repay them somehow. This tag can reduce or eliminate the financial cost for
an implant, but it comes with a different sort of price tag.

\textit{resist (hazard):} the augmentation protects against particular environmental hazards such as toxins or electrocution

\textit{sealed:} a sealed implant requires at least an hour and the
proper tools to reload or refill. Reduce the base cost by
25%.

\textit{toggle:} this item is toggled on and off (that is, once activated, it stays on). For items that modifies a move, multiply
the cost of the item by 1.5.

\textit{used:} this implant started its life in someone else’s body,
and it shows. The first time you fail a move related to the
implant or are in a situation where the added capability
of the device comes into play, roll 1d6. On a 3 or better,
you’re fine. On a 2, the implant simply fails gracefully. On
a 1, the implant goes haywire:
\begin{dent}

\tcirc{} If the implant modifies a move, that move is glitched
until you get it fixed or shut down

\tcirc{} If the implant provides a capability, that ability becomes
a big problem (for example, if your used thermographic
vision goes haywire, you may be temporarily blinded)

\tcirc{} You can shut down a haywire implant by spending a
point of Edge.
\end{dent}
\end{dent}
\subsubsection{CHANGING MOVES}
When a cyberware item modifies a move, the basic version of
it always modifies a core or secondary moves, so it’s useful
to all of the different archetypes. However, if you want to
change the move modified by the item to one of your archetype moves, go right ahead. There’s only one rule: you can’t
double up. If you have an archetype move that grants a bonus
or grants Hold, you can’t change a cyberware item to grant
more Hold for that move. Just take the highest amount.

\subsubsection{MODIFYING TAGS}
If you add a beneficial tag, increase the cost of the item. If
you add a negative tag (such as obvious, or used), reduce the
overall price to reflect this.

\end{multicols}

\invisiblepart{CREATING PROGRAMS}
\section{CREATING PROGRAMS}
\begin{multicols}{2}
Programs act as a Hacker’s weapons, tools, and enhancements in the matrix. They may alter the stats of a cyberdeck,
or enhance your ability to damage enemy code, or help you
pull off moves. A program loaded into a cyberdeck’s storage
is assumed to be running. Changing programs is done by declaring it, or via a move, as the situation demands.

Now, no self-respecting codeslinger buys off-the-shelf software, for a couple reasons: one, there usually isn’t a shrinkwrapped program out there for the things the Hacker wants
to do; and two, if there was, you certainly don’t want anyone
to know you bought it.

So what is the Hacker to do? Well, write code, of course!
Here are rules for creating your own tools for bending the
matrix to your will.

\subsection{CODING}
Programs consist of one or more routines, which are appended to the program name as tags. Each routine offers a different effect or benefit; multiple routines can be combined into
a single piece of software.

Writing programs follows a simple procedure:
\begin{dent}

1.	Name the program (I encourage you to come up with
suitably Zero Cool names for programs)

2.	Add routines to the program, spending the required
time or money to develop them.

3.	Calculate the size of the program, which is how much
storage it occupies. A program’s size is equal to the
\textbf{number of routines x 2.}

\textbf{Example:} \textit{Blitz is writing a new program for her deck for
an upcoming run. She hopes to slip in, crack the datastore,
and get out. She calls the program NinjQk, and gives it
the routines analyze, stealth, and decrypt. This program
has size 6.}
\end{dent}
\subsubsection{PROGRAM ROUTINES}

\begin{dent}

\textit{Analyze:} this routine lets the hacker roll+Matrix to Check
the Situation while in VR.

\textit{Attack:} deal 1d6 damage to targeted node, program, or
hacker

\textit{Bounce:} temporarily relocate a hostile program to another
node in the system

\textit{Armor:} this routine increases a cyberdecks Hardening by 1

\textit{Stealth:} this routine increases a cyberdeck’s Mask rating
by 1

\textit{Scan:} this provides +1 ongoing to Awareness-based Stay
Frosty

\textit{Repair:} corrects errors and restores crashed code; heal 1
matrix damage

\textit{Interference:} slows hostile program alarm triggers

\textit{Decrypt:} take +1 to hacking Datastore nodes

\textit{Interface:} take +1 to hack or use Control nodes

\textit{Backdoor:} allows the hacker to automatically gain access
to a specific node at some point in the future.
\end{dent}
STACKING ROUTINES

You can add up to two copies of a single routine to a program. Doing so doubles its effect or the number of times you
can use the routine. For example, Harden can be stacked,
raising the bonus to hardening to +2. \textbf{Note:} when Attack is
doubled, it becomes 2d6b damage.

ON TIME, UNDER BUDGET
When creating programs (with the exception of during character creation), Hackers will need to devote time to writing,
debugging, and perfecting their code. Creating a program
requires the Hacker to spend one day per routine.

Of course, shadowrunners don’t always have the luxury of
time. If a hacker doesn’t have the time to write his or her own
code, he or she can work their contacts to purchase black
market bits. The average cost for a single routine is 250¥.
\begin{dent}

\textbf{Example:} \textit{Blitz’s new program, NinjQk, needs to be done
pretty quick. She has one day free, so she spends that
writing the analyze routine. However, she’s out of time
by then, so she calls up a couple buddies and snags some
stealth and decryption libraries from them. Since they were
friends, they cut her a break, and she scored the two routines for about 400¥.}
\end{dent}
\subsection{AGENTS}
As programs are assembled from multiple routines, it is possible to compile multiple programs into an autonomous expert system called an \textbf{agent}, virtual companions to a hacker
that act independently of the hacker but in accord with his or
her wishes.

Only one agent can be in operation at once. Agents have the
following characteristics:
\begin{dent}

\textbf{CPU:} this is the primary stat of the Agent, and is used
when executing its moves

\textbf{Wounds:} a Agent’s wounds are equal to the combined
size of its constituent programs

\textbf{Moves:} Agents use the Sling Code and Born Digital moves

\textbf{Other Stats:} any other stats an Agent depend on its constituent programs (e.g., if a constituent program has the
Armor routine, the Agent has Armor 1)
\end{dent}
To create an Agent:
\begin{dent}

1.	Choose up to 6 storage worth of programs already running on your cyberdeck to compile together.

2.	Allocate at least 1 point from your cyberdeck’s CPU to
the Agent’s CPU stat. A cyberdeck whose CPU is reduced to 0 in this fashion is not destroyed; it simply has
all of its primary power devoted to the agent, and CPU
cannot be added to the result of any Hacker moves.

3.	Determine the Agent’s wounds and other characteristics per the information above.
\end{dent}
\end{multicols}

\invisiblepart{CREATING SPELLS}
\section{CREATING SPELLS}
\begin{multicols}{2}
Spell creation in \SW/ is relatively simple, and requires
only that you name the spell, and then assign it the appropriate tags to describe how it works, based on the Spell Templates presented in the next section.

Every spell must have all core tags assigned; additional tags
may be assigned (see Other Spell Tags) as necessary (or when
required in the rules that follow).

\begin{dent}

\textbf{Example:} \textit{Lynn, playing the Mage, wants a spell that
shoots a jet of acid at the target. She calls it Acid Spray,
and gives it the following tags:} close/short/medium, creature, instant, Force 3, 1d8+EF dmg, element:acid,
obvious.
\end{dent}

\begin{dent}

\textbf{Example 2:} \textit{Lynn’s not all about hurting people; sometimes she needs to protect herself too! She creates a spell
she calls Astral Armor. It is a Manipulation spell affecting
only her, triggered by any incoming damage, and not obvious to casual observers. She starts with some basic tags:}
touch, self, triggered, Force 1, effect:+1 armor. \textit{Since it’s
a protective Manipulation spell, it gets the protection tag
as well. She wants it to be a bit more potent, so she decides to add the} exhausting \textit{tag to increase effect to +EF armor.
Finally, she wants to add the} subtle\textit{ tag, which requires an
extra point of Force. The final spell, then, is} Astral Armor
[touch, self, triggered, Force 2, protection, subtle, exhausting, effect:+EF armor against one attack]. \textit{It’s a costly
spell, but a nice way to have some low-profile protection
against surprise attacks.}
\end{dent}

\subsection{SPELL FORCE}
All spells start with a minimum Force of 1. This is the force at which the spell must be cast to gain any effect. Some tags increase this minimum. No combination of tags can reduce a spell's minimum below 1. When determining the effects of the spell, use the \textbf{Effective Force}, or \textbf{EF}, value which is the \textbf{(Force Cast - Minimum Force) + 1}. 

\begin{dent}
\textbf{Example:} \textit{Lynn wants to cast her Acid Spray spell which has a minimum Force of 3. Casting this spell at Force 3 would yield an EF of 1. Lynn wants more damage potential, so she decides to cast a more powerful Force 5 Acid Spray, which yields an EF of 3.}
\end{dent}


\subsubsection{SPELL TEMPLATES AND TAGS}

All spells share a core set of tags describing their \textbf{Range, Targets, Duration, Essence}, and \textbf{Effect}.

\textbf{Range} describes the effective range over which the spell can
be cast. Remember that most spells require line of sight to the
target. Combat spells, by default, have the \textit{LOS} tag. By default, a spell can only have one \textbf{Range} tag.
\begin{dent}

\textit{Touch:} the spellcaster must touch the target to cast the
spell.

\textit{LOS:} the spellcaster must be within line of sight of the target. Technological vision enhancements (aside from old fashioned optics) do not count for line of sight.

\textit{Linked:} the spellcaster must possess an object of high significance to the target, or a fresh (under 24 hours old) bodily sample. With an appropriate link, the spell has a range of \textbf{EF} kilometers.

\end{dent}
\textbf{Target} indicates the valid targets for the spell. Spells are by
default single target, though they may have multiple valid
target types.
\begin{dent}

\textit{Self:} the spell only affects the caster

\textit{Metahuman:} the spell only affects metahumans

\textit{Creature:} the spell affects any living creature

\textit{Spirit:} the spell affects only spirit beings

\textit{Object:} the spell affects inanimate objects

\textit{Device:} the spell affects technological devices
\end{dent}
\textbf{Duration} specifies how long the effects of a spell normally
last. \textbf{Note:} wound or stun damage removed by a spell does
not come back when the spell’s duration is up, unless that is
specified in the spell effect itself. For ease of play, those sorts
of effects are permanent.
\begin{dent}

\textit{Instant:} the spell occurs very quickly. \textbf{All Combat spells
have instant durations.}

\textit{Short:} the spell lasts long enough for the target to take one
move, more or less (this is common for spells that boost a
single move or enhance a Stat temporarily). Triggered (see
\textit{Other Spell Tags}) can replace this tag at the caster’s discretion. \textbf{All spells except Combat spells have a default
duration of short.}

\textit{Sustained:} the spell remains in effect for a period determined by the caster. Each sustained spell in effect inflicts a stacking -1 to future spellcasting moves to account for the split concentration of the caster. Common for spells that
grant ongoing bonuses.

\textit{Specified:} the spell lasts for a specific amount of time (e.g.
5 minutes, 30 minutes, 1 hour).
\end{dent}
\textbf{Force} indicates the minimum Force expenditure required to cast the spell. No customizations can reduce a
spell’s minimum Force below 1.

\textbf{Effect} describes the actual result of a successful casting of
the spell. Spell effects are extremely varied, but generally do
such things as enable previously impossible abilities (breathing underwater, or perceiving remote events), enhance existing abilities (offering bonuses or Boosts to moves or Stats), or
healing or inflicting damage. \textbf{Note:} the effect of combat spells
is almost always, of course, to inflict damage.

\subsection{CUSTOMIZING SPELLS}
Using the basic tags as well as tags specific to certain spell
categories (if any), spells can be modified in order to meet the
caster’s needs. Most modifications simply require the caster
to commit more essence to power the spell.

The following modifications are common:
\begin{dent}

\textit{More Targets:} additional valid target types or additional
targets can be added to a spell. For each target type added, increase the minimum Force by 1.

\textit{Discreet Casting:} all spells are assumed to have the obvious tag, indicating that you can’t miss the mage going
through the motions to cast the spell. To add the subtle tag
to hide the casting process, increase the minimum Force by 1.

\textit{Increased Range:} to add an additional range tag, increase
the minimum Force by 1. By default, Combat spells start with
LOS; Health spells start
with touch, and other spells with touch, LOS.

\textit{Decreased Range:} in some cases you may wish to decrease
the effective range of a spell in order to decrease its minimum Force. Remove the longest range increment and either reduce minimum Force, or (for damaging spells) stage
the damage die type down one step.

\textit{Potent Effect:} you may double the potency of a non-combat spell’s effect, by adding the exhausting tag (modifying
the effect of combat spells is described in that section).

\textit{Increase Duration:} some spells (usually Health, Illusion,
and Detection spells) have durations longer than instant.
Increasing the duration of the spell by one step increases the minimum Force by 1.
\end{dent}

\subsubsection{TYPE-SPECIFIC SPELL TAGS}

\paragraph{COMBAT}

Combat spells have the following specific customization options:
\begin{dent}

\textit{Damage:} instead of an effect tag, combat spells deal damage (similar to weapons). All combat spells start with a
base damage value of \textbf{1d6}. Spell damage can be upgraded
in a couple ways, each with a cost:

\textit{Scaling (add EF to damage):} either remove the highest range increment
from the spell, or add the obvious tag

\textit{Upgrade damage die:} increase the minimum Force
of the spell by 1, and add the obvious tag

\textit{Downgrade damage die:} reduce the damage die by 1 step to reduce minimum Force by 1 

\textit{Modify the damage to a ``best'' roll:} add the \textit{exhausting} tag

\textit{Ignore Armor:} reduce the damage die by 1 step

\end{dent}
\paragraph{DETECTION}

Detection spells have the following specific tags:
\begin{dent}

\textit{Analysis:} the spell is designed to analyze the workings of
an object, device, or similar target

\textit{Perception:} the spell enhances the target’s perceptive capability or to enable otherwise impossible feats of perception (such as clairvoyance)

\textit{Telepathy:} the spell affects the target’s mind, allowing the
caster to read surface thoughts or intentions, or glean other information
\end{dent}
\paragraph{ILLUSION}

Illusion spells have the following specific tags:
\begin{dent}

\textit{Concealment:} the spell’s purpose is to conceal its targets
from detection by others

\textit{Distraction:} the spell creates illusions that distract and confuse the target, enhancing your actions or hampering theirs
\end{dent}
\paragraph{MANIPULATION}

Manipulation spells have the following specific tags.
\begin{dent}

\textit{Protection:} the spell’s focus is protecting the target(s)
against threats

\textit{Telekinesis:} the spell enables the caster to move physical
objects

\textit{Energy:} the spell manipulates energy to create effects
(such as igniting material or generating light)

\textit{Mental:} the spell manipulates the mind of the target
through direct magical force
\end{dent}
\paragraph{HEALTH}

Health spells have the following specific tags:
\begin{dent}

\textit{Heal:} the spell mends wounds and eases trauma

\textit{Cure:} the spell counteracts the effects of disease, toxins,
and similar threats.

\textit{Enhance:} the spell enhances the physiology of the target
in some, such as increasing a Stat or enabling otherwise
impossible feats
\end{dent}
\subsubsection{OTHER SPELL TAGS}
\begin{dent}

\textit{Area:} the spell covers an area of effect, within its specified
range, and affects all valid targets in the area. Adding the
area tag to a combat spell reduces the damage die by 1
step (to a minimum of 1d4); adding this to another kind of
spell increases its minimum Force by 1.

\textit{Element:} this spell has an elemental aspect (e.g. acid, fire,
ice, electricity, water, air) with corresponding additional
effects; increase its minimum Force by 1.

\textit{Scaling (add EF to an effect):} add the \textit{exhausting} tag and add the \textit{obvious} tag. Only available for non-Combat spells.

\textit{Exhausting:} this spell is especially draining; the caster must
take at least 1 stun damage (in addition to any other drain damage incurred) when casting this spell (this stun ignores armor, although it can otherwise be healed normally).

\textit{Subtle:} this tag means much the same as it does with other
activities, except that for spells, it indicates that the preparations to cast the spell are subtle; the spell effect itself
may or may not be (for example, a fireball can be subtle,
but only insofar as nobody notices the mage forming the
spell; once it goes off, it’s certainly obvious).

\textit{Triggered:} this spell is triggered by a particular event (often
a move); it remains in effect until the individual in question makes the triggering move or action. This tag is a
replacement for the Short duration tag at the spellcaster’s
discretion.
\end{dent}
\subsubsection{THE MAGE'S SANCTUM}
Mages, unfortunately, cannot simply borrow another mage’s
spell to use. The creation of a spell is a very personal event,
and you wouldn’t want to have someone else’s formulas ``go
down the wrong pipe,'' as it were. As a result, it requires time
(and sometimes money) to develop a spell.

Mechanically, development of a new spell requires the Mage
to spend at least 72 hours in study, preparing reagents,
studying tomes, and inscribing strange symbols. Once done,
of course, the spell is added to the mage’s repertoire; a Mage
never forgets her spells.

It is possible to shorten this process somewhat by obtaining
help from outside sources. Talismongers, for instance, might
be able to locate items or suggest pronunciations; other mages may be able to explain certain concepts to the uninitiated;
and spending time in pure study (using the Initiate move) can
reduce the time required.

\end{multicols}


\invisiblepart{CREATING SPIRITS}
\section{CREATING SPIRITS}
\begin{multicols}{2}
Instead of crafting spells like mages, shamans familiarize
themselves with the denizens of Astral Space, learning to
make bargains and offer wagers in order to secure the aid
and services of these ethereal beings. A practiced shaman is
adept at ``wheeling and dealing'' with spirits and elementals.
There is a dizzying array of different spirits in the astral world.
\SW/ lets the Shaman create the spirits they wish to
summon.

\subsection{SPIRIT BONDING}
Although the rules here provide a mechanical way to make
your own custom spirits, remember that spirits are independent entities, not ``on the fly'' creations of the Shaman. In
the game world, the shaman has met, negotiated with, and
bonded with a spirit, developing a relationship (the \textbf{spirit
bond}) with the entity.

\subsubsection{JUST BUSINESS}
It is important to recognize that the relationship between the
Shaman and the spirits to whom he or she has bonded is
not necessarily (or even \textit{usually}) one of friendship or altruism.
Rather, the relationship is more akin to a contract or pact—it
is a business relationship, with consideration promised and
mutually agreeable terms established. Spirits do not, as a
rule, love being randomly yanked out of the astral plane to
perform work for people, and if uncontrolled, are as likely to
turn on their summoner as they are to simply vanish back into
Astral Space.

\subsubsection{RULES}
Use the following procedure to develop the spirits with which
you’ve formed a Spirit Bond.
\begin{dent}

1.	Choose the spirit’s \textbf{Type:} elemental or natural.

2.	Choose the spirit’s \textbf{Domain}, and record the base Armor
and Wounds.

3.	Choose the spirit’s \textbf{Nature}, and modify the basic spirit
tags as needed.

4.	Distribute 4 spirit points among spirit’s Moves, adjusting for the spirit’s purpose. No spirt move may have a
modifier higher than +3.

5.	Add additional tags if desired (see \textit{Other Spirit Tags}).

6.	Name your spirit.

\textbf{Example:} \textit{Pam is playing a Shaman named Chert, and is
developing the initial three spirits Chert can summon. Pam
decides the first one will be a natural forest spirit, a protector of the dwindling unspoiled lands.}

\textit{With those decisions made, the spirit’s qualities so far are}
natural, forest, protector, armor 1, wounds 10, dmg 1d8,
guard 1, enthrall -1.

\textit{Pam also wants the spirit to blend in with the forest, and
to an excellent guardian of its inhabitants. She spends
one spirit point (out of 4) to gain the} aspect \textit{tag, and then
spends the remaining three to boost the Guard move
twice, and the Harm move once. The final spirit looks like
this:} natural, forest, protector, harm 2, guard 3, search 0,
enthrall -1, mentor 0, armor 1, wounds 10.
\end{dent}
\subsubsection{SPIRIT TYPES}
\begin{dent}

\textbf{Elemental:} these spirits represent the basic four elements,
air, earth, fire, and water, and can be summoned anywhere.

\textbf{Natural:} natural spirits are spirits associated with particular domains (such as ``city spirits'' or ``mountain spirits'').
Natural spirits may enter other domains freely, but they
can only be summoned within their own, and if they cross
domains, there’s always a chance they attract unwanted
attention from other spirits who don’t like intruders.
\end{dent}
\subsubsection{BASIC SPIRIT TAGS}
\textbf{Domain} represents the spirit’s preferred environment or the
area in which it may be summoned. A natural spirit summoned in its domain always has the generous tag. The domain of an elemental is considered to be the same as its element (though they gain no benefit from being within their
domain).
\begin{dent}

\textit{Urban:} spirits that dwell in urban or developed lands, especially cities

\textit{Plains:} spirits that dwell in open plains, grasslands, open
fields, and farms

\textit{Forest:} spirits that dwell in forests, woods, and similar areas

\textit{Mountain:} spirits that dwell in foothills, crags, ridges, and
other mountainous terrain

\textit{Earth:} spirits that dwell underground or in caves; the domains of earth spirits are widespread.

\textit{Deserts:} spirits that dwell in the sere, forbidding landscape
of the deserts

\textit{Sky:} spirits dwelling in the open skies.

\textit{Storm:} spirits of storm and disruption

\textit{Swamps:} spirits who dwell where earth and water are one

\textit{Water:} spirits of the water, be it lakes, rivers, or the open
sea
\end{dent}
There are two things to be aware of regarding domains. First,
domains are relatively confined—a mountain spirit’s domain
is not all mountains, nor even all of a specific mountain. Rather, it is usually a region with a radius of around 500 meters,
within a mountainous region. Overlap among domains is possible, and the byzantine negotiations that take place between
spirits defy understanding even by the most gifted shamans.

Also remember that multiple domains may exist within a
larger area that seems uniform. In other words, city spirits
(for example) are the only kind of spirit you’ll run across in a
city—a park within a city may be the home of a forest spirit,
and you may find a river spirit fighting to protect it’s home
from polluted runoff in some industrial area.

\textbf{Armor} represents the spirit’s innate magical resistance to
damage; spirit armor cannot be ignored, nor reduced by
weapons with the AP tag. All spirits have 1 armor.

\textbf{Wounds} simply represent the spirit’s innate health; all spirits,
by default, have 8 wounds.

\subsubsection{SPIRIT NATURE}
Every spirit has a \textbf{nature}, which indicates its sense of purpose
and the activities to which it is drawn. A spirit’s nature also
affects its basic tags and moves (see Spirit Moves, below) in
various ways.

\textbf{Watcher} spirits observe, find, and note. They are incapable of
dealing harm to anyone or anything. Watcher spirits have the
following modifiers: \textit{Search +2, Wounds -2, may not Harm.}

\textbf{Teacher} spirits wish to inform and instruct, and find it difficult to inflict damage upon those they could otherwise teach.
Teacher spirits have the following modifiers: \textit{Mentor +2,
Harm -2, dmg 1d4.}

\textbf{Protector} spirits preserve, defend, and support their domain.
They are unconcerned with influencing intruders, preferring to
throw them out instead. Protector spirits have the following
modifiers:\textit{ Guard +1, Enthrall -1, Wounds +2, dmg 1d8.}

\textbf{Destroyer} spirits are warrior spirits who revel in combat and
bloodletting. They are fearsome enemies, though somewhat
limited in imagination. Destroyer spirits have the following
modifiers:\textit{ Harm +2, Mentor -1, Search -2, Wounds +1, Armor +1, dmg 1d10.}

\textbf{Seducer} spirits wish to influence, to inspire love, andto acquire servants, though they do not typically enjoy directly
harming others. Seducer spirits have the following modifiers:
\textit{Enthrall +2, Harm -1, Wounds -1, dmg 1d4.}

\subsubsection{SPIRIT MOVES}
Spirits and elementals summoned by player characters are
individual beings that have their own set of moves. While
summoned, spirits may perform a number of moves equal to
their Force. Each use of a move below counts
toward that limit.

When creating a spirit, the Shaman may spend up to 4 spirit
points to increase the value of a spirit’s moves. However,
remember that some additional tags cost spirit points, so use
them wisely!

\textbf{HARM:} when a spirit \textbf{attacks someone or something,}
roll+Harm. On 10+, the spirit deals its damage. On 7-9, the
spirit deals damage, but also takes damage.

\textbf{SEARCH:} when the spirit \textbf{attempts to locate individuals or
items within its domain,} roll+Search. On 10+, the spirit locates the item and can tell the Shaman where it is. On 7-9,
the spirit can tell the shaman whether the item or person is
within its domain, but not it’s specific location. Note: the GM
and player should determine the search range for elementals.

\textbf{GUARD:} when a spirit \textbf{stands in defense of its domain or
inhabitants thereof,} roll+Guard. On 10+, the spirit prevents
damage or hostile effects from occurring. On 7-9, the spirit
halves damage or the potency of a hostile effect.

\textbf{ENTHRALL:} when a spirit \textbf{attempts to control someone’s
actions or thoughts,} roll+Enthrall. If the target is a:
\begin{dent}

\tcirc{} An NPC: On a 10+, the spirit issues two instructions
that the NPC must follow, or take 3 damage. On 7-9,
the spirit may issue one instruction.

\tcirc{} A PC: On a 10+, both of the following apply. On 7-9,
only 1 applies:
\begin{dent}

\tcirc{} If the character complies, they mark XP

\tcirc{} If the characer refuses, they must Stay Frosty
\end{dent}
\end{dent}
\textbf{MENTOR:} when a spirit \textbf{imparts knowledge or truth}, roll+Mentor. On 10+, the GM provides, in secrete, a useful or
interesting piece of information to the target. On 7-9, the GM
provides an interesting piece of information.

\subsubsection{OTHER SPIRIT TAGS}
\begin{dent}


\textit{Robust:} the spirit is particularly resistant to damage; all
damage rolls against it are \textbf{[w]}. Adding this tag costs 1
spirit point.

\textit{Aspect:} the spirit takes on the appearance of their domain,
and is invisible in their domain unless it chooses to be
seen. All spirits have this tag.

\textit{Generous:} the spirit will perform one extra move; adding
this tag costs 1 spirit point.

\textit{Insubstantial:} damage dealt and taken is halved

\textit{Weakness (specify):} the spirit has a weakness to a particular
material or element which ignores insubstantiality, armor,
and robustness. Adding this tag allows the free addition of
another tag.

\textit{Engulf:} the spirit may enclose a target in the ubstance of its
domain, typically (but not always) dealing damage.

\textit{Wild:} this spirit has an extra spirit point, but the shaman
must take -2 whenever he or she conjures it.
\end{dent}
\subsection{MAKE NEW BONDS}
As with weaponry, spells, or programs, it takes time and effort to develop a relationship with a spirit. The spirit creation
rules here are, as already said, not intended for ``on the fly
summoning,'' rather they are intended to help Shaman players create a list of spirits that the shaman is accustomed to
summoning, and that fit the player’s desired concept for their
character.

If the Shaman wants to develop a relationship with a new
spirit, the character must spend at least two full days of
downtime meditating and communing, meeting and negotiating with spirits in the Astral realm. At the conclusion of this
time, the Shaman’s player may create a new spirit with whom
the Shaman has formed a bond.

\subsubsection{INTRODUCTIONS}
A shaman can reduce the time spent in bargaining with a
new spirit in a very simple way—have another spirit ``make
introductions.'' To do so, a Shaman must be mentored by another spirit (one he or she has summoned). If the mentoring
is successful (use the Mentor move), reduce the time required
by one day.

\end{multicols}

\invisiblepart{TOTEMS}
\section{TOTEMS}
\begin{multicols}{2}
Shaman characters must select a totem, representing their
connection to one of the great spirits.

\paragraph{BEAR}
\begin{dent}
\textbf{Boons:} +1 to summoning Protector spirits.

\textbf{Flaw:} when injured, roll 1d6. On 1 or 2, the shaman goes
berserk).
\end{dent}
\paragraph{CAT}
\begin{dent}
\textbf{Boons:} gain low-light vision; you cannot be surprised

\textbf{Flaw:} you cannot deal lethal damage to your enemy
\end{dent}


\paragraph{COYOTE}
\begin{dent}
\textbf{Boons:} take +1 to conjure Teacher spirits

\textbf{Flaws:} destroyer spirits summoned lose 1 spirit point
\end{dent}

\paragraph{DARK KING}
\begin{dent}
\textbf{Boons:} when you check the situation, you are boosted

\textbf{Flaw:} take -1 ongoing to gut checks
\end{dent}

\paragraph{DOG}
\begin{dent}
\textbf{Boons:} and take +1 to conjure protector spirits or city
spirits

\textbf{Flaw:} your moves are glitched if you have left an ally
behind or in danger
\end{dent}


\paragraph{DRAGONSLAYER}
\begin{dent}
\textbf{Boons:} take +1 to stay frosty

\textbf{Flaw:} if you break a promise, your moves are glitched until you fulfill the promise or otherwise atone
\end{dent}

\paragraph{GATOR}
\begin{dent}
\textbf{Boons:}take +1 to conjure water spirits.

\textbf{Flaw:} You are exceptionally greedy
\end{dent}


\paragraph{EAGLE}
\begin{dent}
\textbf{Boons:} take +1 to conjure watcher spirits or air elementals

\textbf{Flaw:} you have an allergy to something relatively common, and take -1 ongoing when exposed
\end{dent}

\paragraph{FIRE-BRINGER}
\begin{dent}
\textbf{Boons:} take +1 to conjure fire spirits.

\textbf{Flaw:} when facing a difficult choice, you invariably choose to aid others, even at your own expense
\end{dent}


\paragraph{LION}
\begin{dent}
\textbf{Boons:} take +1 to conjure protector or plains spirits

\textbf{Flaw:} Take -1 on Gut Checks
\end{dent}

\paragraph{MOON MAIDEN}
\begin{dent}
\textbf{Boons:} take +1 ongoing to manipulate others

\textbf{Flaw:} take -1 to casting combat spells
\end{dent}

\paragraph{MOUNTAIN}
\begin{dent}
\textbf{Boons:} take +1 to conjure earth spirits

\textbf{Flaw:} once you've decided on a plan of action, you stick to it - even it means going alone
\end{dent}


\paragraph{OWL}
\begin{dent}
\textbf{Boons:} gain low-light vision, take +1 to conjure teacher
spirits

\textbf{Flaw:} During the day, the minimum force to cast spells is increased by 1
\end{dent}


\paragraph{RACCOON}
\begin{dent}
\textbf{Boons:} and take +1 to conjure watcher spirits

\textbf{Flaw:} must Stay Frosty to avoid letting his curiosity get to
him
\end{dent}


\paragraph{RAT}
\begin{dent}
\textbf{Boons:} take +1 to conjure city spirits

\textbf{Flaw:} when combat starts, you must Stay Frosty, or flee
\end{dent}


\paragraph{RAVEN}
\begin{dent}
\textbf{Boons:} take +1 to conjure watcher spirits

\textbf{Flaw:} you must take advantage of others’ misfortune
when you can
\end{dent}

\paragraph{SEDUCTRESS}
\begin{dent}
\textbf{Boons:} take +2 when manipulating someone

\textbf{Flaw:} when given the opportunity to indulge in a vice, roll 1d6: on 1, 2, or 3, the shaman gives into the vice (drugs, btls, etc...)
\end{dent}


\paragraph{SHARK}
\begin{dent}
\textbf{Boons:} take +1 to conjure destroyer spirits

\textbf{Flaw:} when injured or when you injure another, roll 1d6: on 1, 2, or 3, the shaman
goes into a frenzy. The shaman may continue to attack their last victim instead of moving on to nearby opponents. 
\end{dent}


\paragraph{SNAKE}
\begin{dent}
\textbf{Boons:} take +1 to conjure seducer spirits

\textbf{Flaw:} take -1 ongoing to Rock \& Roll
\end{dent}

\paragraph{THUNDERBIRD}
\begin{dent}
\textbf{Boons:} take +1 on when making them sweat

\textbf{Flaw:} when insulted, roll 1d6: on 1, 2, or 3, the shaman can't resist but to respond in kind
\end{dent}

\paragraph{WISE WARRIOR}
\begin{dent}
\textbf{Boons:} take +1 ongoing to Rock \& Roll

\textbf{Flaw:} when you have acted dishonorably, you are glitched until you are able to atone
\end{dent}


\paragraph{WOLF}
\begin{dent}
\textbf{Boons:} take +1 to conjure protector spirits

\textbf{Flaw:} you must Stay Frosty to retreat from combat
\end{dent}

\end{multicols}

\invisiblepart{Compendium Classes}
\section{Compendium Classes}
\begin{multicols}{2}

In \SW/, a \textbf{compendium class} is an mixin class that augments your existing class. Similar to cross-archetype moves, compendium classes provide a set of moves a player may take for their character upon advancement. However, compendium classes differ in two respects:

\begin{dent}

1. All compendium classes possess a requirement. The nature of the requirement varies, but in generally they represent a special milestone in a character's life which marks a significant change in attitude, social standing, or another context.

2. There are no restrictions on the number of moves that be chosen from a compendium class. Typically the number of moves per class is low and the existence of the entrance requirement helps to focus the character down a particular path.

\end{dent}

Compendium classes typically fulfill two roles in \SW/. Most commonly, compendium classes allow characters to explore niches that are not fully represented by standard archetypes. A good example would be the \textit{Assassin} class - while the \textit{Covert Ops} archetype could be played as an assassin, for example, their class moves aren't especially specialized for wetwork jobs. At the same time, within the world of \SW/, assassins are not common enough to be a general character archetype -  they certainly exist, but those individuals that market themselves as professional assassins are rare.

The other role fulfilled by compendium classes are campaign specializations. Every setting has its special groups and a compendium class is an excellent method of describing membership. Whether that means a character is part of a setting's organization, where class moves represent perks of the position, or has undergone an experience unique to the setting, where the moves represent the effects of the experience.

Several example compendium classes are listed below:

\end{multicols}

\newpage

\invisiblepart{Compendium Class: Assassin}
\section{Assassin}
\begin{multicols}{2}


\texttt{>>> I get it. No one ever wants to be the guy pulling the trigger. Maybe they think they have a conscious, or whatever. Or maybe they just don't want to deal with the mess. I don't really care. Because one thing is for certain: no matter their reasons for not doing the job themselves, there is a never ending stream of people who have no problem hiring me to do it for them. <<<}

\textbf{The Assassin} is a cold blooded killer for hire. That is the thin line that separates the assassin from the common murderer. As long as the money is good, it doesn't matter much who has to die. The assassin is a master of eliminating targets with minimal fuss, whether at range or up close. 

\subsection{Assassin Moves \& Requirements}

When you have \textbf{accepted a kill contract and successfully completed the job by eliminating your target and only your target}, the next time you advance, you may take this move:

\textbf{CONTRACT KILLER:} when you \textbf{have downtime and put out word that you’re looking to take on a contract}, roll+Presence. On 10+, someone approaches you with a job - they’ll give you a name and maybe a description. Roll 2d6b: that’s what the job is worth to them, in thousands of nuyen; take it or leave it. On a 7-9, the job has strings attached - they want you to kill the target in a specific way or place, by a specific time, etcetera.

 Either way, once the job is done, they'll find a way to pay you. If you fail to complete a contract, take -1 ongoing to \textbf{Contract Killer} until you prove yourself again.

If you have taken the \textbf{Contract Killer}
move, you are eligible to take any of the
following moves next time you advance:

\textbf{PROFESSIONAL:} when you \textbf{are
  approached for a kill contract}, instead roll
1d10 to determine the price of the contract in
thousands of nuyen. 

\textbf{MEASURE TWICE, KILL ONCE:} when you
\textbf{have time to study the environment of your
  target beforehand for at least an hour}, you are
never considered surprised when operating within
that environment for the next 24 hours.

\textbf{HEARTSEEKER:} when you \textbf{Rock \&
  Roll}, on a 10+ you may spend 1 Edge to specifically target and
destroy a vital organ of your choice.

\textbf{GRIM EXPERIENCE:} when you \textbf{take damage
from combat}, you may optionally take 1 additional
point of damage. If you do so, mark off an
additional point of XP.

\textbf{SERIES OF IMPROBABLE EVENTS:} when you
\textbf{doctor a scene to hide your involvement},
roll+Craft. On 10+, you can create an impression
that the murder was an one in a million
accident. On 7-9, you hide your tracks, but
careful investigation will reveal the murder.


\end{multicols}

\invisiblepart{Compendium Class: Smuggler}
\section{Smuggler}
\begin{multicols}{2}

\texttt{>>> Everyone wants something. And when
  there is want, someone will be selling. That's
  capitalism for you. Of course, selling requires
  supply and that's where I come in. Guns, drugs,
  whores, even the occasional creepy magic
  artifact, I don't much care. Just give me the
  cash, the cargo, and the destination - I'll get
  it there in record time, no questions asked. <<<}

\textbf{The Smuggler} is an expert at moving past
borders and checkpoints all while keeping their
cargo hidden and safe. Good smugglers know all the
tricks to avoid border patrols and where to hide
their potentially illicit goods. Its dangerous
work, but for those who enjoy the adrenaline
rush, a smuggler's life pays well. There's always
someone who wants something moved.

\subsection{Smuggler Moves \& Requirements}

When you have \textbf{successfully transported a
  significant amount of contraband past a border
  checkpoint at great personal risk}, the next
time you advance, you may take this move:

\textbf{STICK IT WHERE?!:} when you \textbf{hide a
  small item (the size of your fist or smaller) on
  your person}, roll+Craft. On 10+ the item is
undetectable. On 7-9, you successfully hide the
item, but it is very awkward for you.

If you have taken the \textbf{Stick it Where?!}
move, you are eligible to take any of the
following moves next time you advance:

\textbf{ITS BIGGER ON THE INSIDE:} when you \textbf{have
  downtime or legwork time}, you may modify a
vehicle or drone you own to contain a hidden
compartment. The vehicle or drone cannot contain
more hidden compartments than half its Frame,
rounded down.

\textbf{I DON'T HAVE ANY:} when you \textbf{Stay
  Frosty and lie
  about what you are carrying}, if you have not
looked at looked at your cargo's contents, you are
boosted for the move.

\textbf{EAT MY DUST:} when you \textbf{attempt to
  flee from authorities in a vehicle},
roll+Craft. On 10+, you can execute some quick
maneuvers to throw them off your trail. On 7-9,
you succeed temporarily, but your pursuers will pick up your
trail again soon.

\textbf{TRANSPORT SPECIALIST:} you are an expert
at transporting a particular kind of cargo. Choose
a type of cargo from below - when carrying that
cargo, take +1 ongoing to all rolls.
\begin{moveoptions}
\moveoption{Weapons}
 
\moveoption{Drugs}

\moveoption{People}

\moveoption{Animals}

\moveoption{Magical}

\moveoption{Very Large Objects}

\end{moveoptions}

\textbf{HIDDEN SHORTCUT:} when \textbf{traveling
  in a vehicle or by foot}, roll+Awareness. On
10+, you know of a shortcut that cuts your normal travel
time in half. On 7-9, your shortcut works, but you
encounter trouble on the way.

\end{multicols}

\invisiblepart{Compendium Class: Gang Lord}
\section{Gang Lord}
\begin{multicols}{2}

\texttt{>>> The streets are a hard place. Full of
  dangerous, nasty people. And I'm the worst of
  all. By blood, guile, and ruthless
  determination, I've climbed to the top of this
  social ladder of human trash. But there's no
  rest for the wicked - every week there seems to
  be a new challenger. But I'd be lying if I told
  you I was tired of crushing those bitches skulls. <<<}

\textbf{The Gang Lord} is the ruler of his own little
corner of hell. Outside of the major cities, slums
and barrens crawl with roaming gangs of thugs. And
each gang has its leader. Gang lords, while often
short lived, command respect in their local
communities and have their fingers in all the
local crime. 

\subsection{Gang Lord Moves \& Requirements}

When you \textbf{personally depose an existing
  gang lord in a fashion that indisputably
  demonstrates your power}, the next
time you advance, you may take this move:

\textbf{LEADER OF THE PACK:} you gain command of a
local gang containing 5 members. They follow your commands and, while
you don't lead your gang into hard times, are
loyal to you. Your gang's turf, or sphere of
influence, is a number of kilometers equal to the
number of members in your gang. Choose a gang attribute from the
following list:
\begin{dent}
\textbf{Big} Your gang is rather large. Roll 2d6
and add that number of extra members to your gang.

\textbf{Savage} Your gang is known to be extra
savage in combat. Gain +1 towards \textbf{Make
  Them Sweat} if your target knows of your gang.

\textbf{Bikers} Your gang is a biker gang. Your
gang's turf is now a number of kilometers equal to
twice the number of members in your gang.

\textbf{Wizkids} Your gang has Awakened
members. For every five gang members, choose one
member to be a mage, shaman, or adept.

\textbf{Contraband} Your gang traffics in some
form of contraband. Gain d6 hundred nuyen a week
as your cut as leader.

\textbf{Tight} Your gang is tight-knit by a shared
bond. Your members are more loyal than usual and
will go the extra mile for you.

\textbf{Dug In} Your gang possesses a hidden and
fortified safehouse. 

\textbf{Chromed} Your gang has numerous cyberware
enhancements. When they join you in combat, you
gain +1 towards \textbf{Rock \& Roll} moves.

\textbf{Techheads} You gang has hackers and
riggers. For every five gang members, choose one
member to be a hacker or rigger.

\end{dent}

\critterspec
{GANG MEMBER}
{group, intelligent, medium}
{Spiked bat (1d6+1 dmg, c), cheap but powerful pistol (2d8w dmg, s/m)}
{9 Wounds / 1 Armor}
{A typical member of your gang. They're not
  especially dangerous to a well armed group, but
  in the local neighborhood they are a force to
  reckoned with.}
{to guard their turf.}
{}

If you have taken the \textbf{Leader of the Pack}
move, you are eligible to take any of the
following moves next time you advance:

\textbf{STREET LESSONS:} Choose an additional gang attribute.

\textbf{NEIGHBORHOOD SNITCH:} Your gang had eyes
and ears everywhere within your gang's turf. When
operating within your turf, you cannot be
surprised. 

\textbf{GOT YOUR BACK:} when you \textbf{encounter
  trouble within your gang's turf}, you can call
on your gang and home turf advantage for help and roll+Presence. On 10+,
choose 2. On 7-9, choose one.
\begin{moveoptions}
\moveoption{A gang member is close by and appears
  instantly to aid you}
 
\moveoption{d6 (or total number of gang members,
  whichever is smaller) arrive in 5 minutes to
  aid you. }

\moveoption{Your knowledge of your turf grants you
 +3 hold for subsequent \textbf{Rock \& Roll} and
 \textbf{Check the Situation} moves.}

\end{moveoptions}

\textbf{ARMS SHIPMENT:} you arrange underworld
deals to upgrade your gang's gear. Increase your
gang members' armor by 1 and replace their pistols
with either a SMG (range s/m, sa/fa, dmg 1d8, AP
1) or a shotgun (range s/m, sa, dmg 1d10+1, obvious, loud,
forceful).

\textbf{LEGION:} when you \textbf{defeat a rival
  gang member in combat} you may spare their life
and roll+Presence. On 10+, you convert them into
your gang as a new loyal member. On 7-9, you convert
them, but their loyalty is suspect. 

\end{multicols}


\end{document}
%%% Local Variables: 
%%% mode: latex
%%% TeX-master: t
%%% End: 
