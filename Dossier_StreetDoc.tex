\invisiblepart{DOSSIER : STREET DOCTOR}

\section{THE STREET DOC}
\begin{multicols}{3}
\setlength{\parskip}{.05cm}

\texttt{>>>Medicine, they say, is a calling. You’re in it to help peo-
ple. Well, that’s true, as far as it goes. I liked what I did, un-
til one day I realized I just couldn’t do it anymore. It had
changed, or maybe I did.}

\texttt{But when you’ve spent your time doing it, that’s what you
know. And remember that thing I said about wanting to help
people? Well there’s a whole lot of people who need help,
and they live just below our noses, right where we can’t see.
I set out to help them - street medicine. These days, street
medicine will get you tied up in ugly business sooner or later.
I ended up crossing some people. I needed money. I found
out about shadowrunning. I also found out that plenty of
teams love a good scalpel.}

\texttt{It’s not always fun, combat medicine. In fact, “fun” is not even
in the top 10 words I’d use to describe it. But I figure it’s bet-
ter than leaving someone to see whether blood loss or the
waste management crew gets to them first. So I’m still help-
ing people. They’re not always good people. In fact, they’re
usually career criminals.}

\texttt{Hey, nobody’s perfect.<<<}

\textbf{The Street Doc} brings medical expertise to the shad-
ows, helping their team survive and recover from the
inevitable injuries that they will incur in their particular
line of work. Modern technology might make basic
first aid a matter of a slap patch and a pain pill, but
when you get caught by a frag grenade, basic first aid
is not what you need. You need the Doc.


\subsection{CREATING A STREET DOC}

\paragraph{1.  Choose your Metatype}

You may choose \textbf{Human}, \textbf{Dwarf}, \textbf{Elf}, \textbf{Ork}, or
\textbf{Troll}. Each metatype offers a selection of meta-
type moves. Choose one metatype move from
the options presented.

\paragraph{2.  Choose your look}

\textit{Clear eyes, old eyes, sharp eyes}

\textit{Close cut hair, stylish hairdo, bandana}

\textit{Fit body, heavy body, compact body}

\textit{Business attire, street clothes, EMT jumpsuit}

\paragraph{3.  Choose your name and street name}

Make up a name and street name or pick a real
name and street name from the lists and name
generators starting in the \textbf{GM Resources} section.

\paragraph{4.  Assign your stats}

You have 5 stats: Awareness, Combat, Stamina,
Craft, and Presence. Important stats for you are
Craft and Presence.

You have 4 \textbf{Build Points} to distribute among
your stats. To increase a stat by 1 point costs 1
Build Point. You may increase a stat to a maxi-
mum of +2 as a starting character. If you wish,
you may lower 1 stat to -1 in order to have an
additional point to spend.

\paragraph{5.  Choose your Equipment}

Choose from the lists below, or customize your
own gear using the rules in \textbf{Creating Gear} on
page 60.

\textbf{Armor:} \textit{ballistic vest, armor jacket}

\textbf{Weapon:} \textit{Narcoject rifle, Browning
Max Power, HK227, stun baton, combat knife}

\textbf{Med Kit:} \textit{You have a medkit with 6 Supply.}


\paragraph{6.  Choose your cyberware}

You may start with one of the following cyber-
ware kits (descriptions of these items are on
page 45):

\textbf{Kit 1 (3 essence):} \textit{cyberears with ultrasound
and radio, level 1 skillwires}

\textbf{Kit 2 (3 essence):} \textit{obvious cyberarm with
ReadiMed and shocktrodes}


\paragraph{7.  Set your Essence and Edge.}

To determine your starting Essence, subtract the
essence cost of your cyberware (if any) from 6.

You start with 3 Edge.

\paragraph{8.  Choose 2 Contacts}

ER doctor, morgue staffer, medical examiner,
DocWagon driver, organlegger, black market or-
gan dealer, blood bank worker, pharmacist

\paragraph{9.  Establish debts and favors}

Place one of your fellow runners’ names in at
least one of the blanks in the \textbf{Debts \& Favors}
section of your playbook. Each time a name
appears in a debt or favor, it counts as 1 Bond
with that character. The more people you have
Bond with, the better.

\paragraph{10.  Starting Funds}

You start play with 3d6 x 400¥ immediately
available.

\paragraph{11.  Starting Moves}

You know all the Core and Secondary Moves.
You also know the \textbf{Combat Medic} and
\textbf{Stay With Me} moves.

\end{multicols}

\newpage

\begin{dossier}
\dossierstatbar{THE STREET DOC}
\hspace{.5cm}%
\vrule width 2pt
\hspace{.3cm}%
\begin{dossiermovebar}
\fontsize{9pt}{1em}\selectfont
\setlength{\parskip}{.2cm}


\selectedMove{  Combat Medic:} when you provide medical aid to a person, roll+Craft and mark off 1
Supply from your kit. On 10+, the patient heals 2d4b damage. On 7-9, the patient heals 1d4
damage.

\selectedMove{  Stay With Me:} when you attempt to stabilize a teammate who is bleeding out, roll+Craft
and mark off 2 supply from your kit. On 10+, choose 3. On 7-9, choose 2:
\begin{moveoptions}
\moveoption{they can be moved without a stretcher}

\moveoption{it takes fewer supplies than expected - mark off only 1 supply}

\moveoption{you do not expose yourself to danger to help them.}

\moveoption{they will not have a chronic injury}
\end{moveoptions}
Your patient does not die if you fail this move, and you may take -1 and try again. A second
failure, however, results in the death of the patient.

\unselectedMove{ Grace Under Fire:} when you are working on a patient during a fight but not actively
fighting, you have +1 armor.

\unselectedMove{ First Do No Harm:} if you refuse to do harm (you never deal lethal damage), then your
Grace Under Fire move grants +2 armor instead.

\unselectedMove{ We All Bleed Red:} when you take time to treat an injured enemy, mark off 1 supply
and roll+Presence. On 10+, they’re stable, and you can ask two questions which they will
answer truthfully. On 7-9, you can ask only one question.

\unselectedMove{ Good Drugs:} when a patient contracts a disease or is poisoned, roll+Awareness. On
10+, you have the correct antidote or antitoxin on hand, and can halt the progress of the
disease or poison. Mark off 1 supply from your medkit. On 7-9, you are only able to slow
the effects. Mark off 1 supply from your medkit.

\unselectedMove{ Pharmacy Is Open:} when you use a contact to obtain medical supplies (amounting to
+1 supply), and roll+Presence. On 10+, choose 2. On 7-9, choose 1:
\begin{moveoptions}
\moveoption{you get +2 supply instead of +1}

\moveoption{it takes 1 day to get the supplies instead of 2}

\moveoption{nobody notices the supplies are missing}

\moveoption{you receive an interesting piece of information as well}
\end{moveoptions}
\unselectedMove{ Mobile Surgery:} you own a vehicle that contains a small but complete surgical suite,
capable of treating serious injuries. It has a base supply value of 10. Supplies from the mo-
bile surgery can be used to replenish your Med Kit.

\unselectedMove{ You Got This:} whenever you walk someone through a medical procedure (such as first
aid), roll+Presence. On 10+, they are boosted. On 7-9, they take +1.

\unselectedMove{ Megalexicosis:} when you spout a stream of medical technobabble to confuse, intimi-
date, convince, or distract someone, you may roll+Craft instead of +Presence.



\end{dossiermovebar}%
\end{dossier}

%%% Local Variables: 
%%% mode: latex
%%% TeX-master: "sixth_world"
%%% End: 
